\usepackage[top=2cm,bottom=2cm,left=2cm,right=2cm]{geometry}% On peut changer en cours de route avec la commande \newgeometry

\usepackage[utf8]{inputenc}
\usepackage[french]{babel}
\usepackage{amsmath,amsfonts,amssymb,graphicx}
%\usepackage{amsthm}
\usepackage{framed}
\usepackage[amsthm,thmmarks,framed]{ntheorem}
\usepackage[dvipsnames]{pstricks}
%\usepackage{pstricks-add,pst-plot,pst-node}
\usepackage{pstricks,pst-plot,pst-text,pst-tree}%,pst-eps,pst-fill,pst-node,pst-math,pstricks-add,pst-xkey}
\usepackage{epic,eepic}
% Versions propos\'ees par Frédéric Junier
\usepackage[inline]{asymptote}


%\usepackage{color} %(black, white, red, green, blue, yellow, magenta et cyan)
\usepackage[dvipsnames]{xcolor}

\usepackage{verbatim}

\usepackage{enumitem}
\frenchbsetup{StandardLists=true}

\usepackage[all]{xy}

\usepackage{multicol}				%%%	Jusqu' \`a nouvel ordre pas de multicol, pour cause de compatibilit\'e avec des fonctions graphiques de TeX.
%\begin{multicols}[titre]{nb colonnes}
%\setlength{\columnseprule}{0.25pt}

\usepackage{parskip}
\setlength{\parindent}{0cm}
%\setlength{\parskip}{0.5cm}

%Je veux une mise en page pour les théorèmes lemmes preuves, définitions etc, suffisamment aérée. Est-il possible de commander un encadrement systématique ?

\newcommand{\g}[1]{\`#1}%\g{lettre}
\newcommand{\Rom}[1]{\uppercase\expandafter{\romannumeral #1\relax}}%\uppercase permet d'avoir des lettres majuscules

\providecommand{\abs}[1]{\lvert#1\rvert}

%Pour tout le m\'emoire !
\newcommand{\Fiy}{$(P,X,\mathcal{P},Y)$}
\newcommand{\Fig}{$(P,X,\mathcal{P},G)$}


%Annotations d'un m\'emoire en cours de r\'edaction :
\newcommand{\tr}{\textbf{\textcolor{red}{D\'emonstration \g{a} trouver }}}
\newcommand{\re}{\textbf{\textcolor{NavyBlue}{D\'emonstration \g{a} recopier }}}
\newcommand{\es}{\textbf{\textcolor{OliveGreen}{Esquisse de d\'emonstration }}}

%Pour les formules alg\'ebriques casse-pieds :
\newcommand{\Sk}[1]{\mathcal{S}(\mathbb{#1}^{n+1},\|\|_2)}
\newcommand{\Qr}{\frac{\mathbb{K}^{n+1}\setminus\{0\}}{\mathbb{R}_+^{\ast}}}
\newcommand{\mcd}[4]{\left(\begin{array}{cc}#1&#2\\#3&#4\end{array}\right)}

%Pour les formules concernant les Ufibrés :
\newcommand{\ec}[1]{e^{\mathbf{i}#1}}
\newcommand{\st}[3]{\left(\cos\dfrac{#1}{2}\ec{#2},\sin\dfrac{#1}{2}\ec{#3}\right)}
\newcommand{\se}[2]{s_{U_{#1}}^{#2}}
\newcommand{\lt}[1][k]{\dfrac{\mathbb{S}^3}{\mathbb{U}_{#1}}}
\newcommand{\pl}[3][\omega]{\text{Im}\left(\left(#2\cdot #1 ,#3\cdot #1\right)\right)_{#1\in\mathbb{U}_k}}
\newcommand{\pla}[3][m]{\text{Im}\left(\left(#2\cdot \ec{\frac{2#1\pi}{k}} ,#3\cdot \ec{\frac{2#1\pi}{k}}\right)\right)_{#1\in[\![1,k]\!]}}
\newcommand{\gk}[1][k]{\underline{\gamma}_{#1}}

%Autres formules pour le m\'emoire :
\newcommand{\tc}[1]{\text{#1}}
\DeclareMathOperator{\sinc}{sinc}


\newcommand*{\etoile}
{
\begin{center}
\hspace{1pt}\par
*\hspace{5pt}*\hspace{5pt}*
\end{center}
}


\newcommand*{\ligneinter}
{
\begin{center}
\vspace{2pt}
\hfill\rule{0.5\linewidth}{0.1pt}\hfill\null
\end{center}
\vspace{7pt}
}

{
\theoremstyle{break}
\theoremprework{\vspace{0.2cm}\begin{minipage}{\textwidth}\setlength{\parskip}{0.2cm}} % Pris en compte uniquement pour le premier newtheorem. parskip seulement à l'intérieur de la définition.
\theorempostwork{\ligneinter\end{minipage}} %Même chose
\theoremheaderfont{\scshape}
\theorembodyfont{\itshape}
\theoremseparator{ :\newline\vspace{0.2cm}}
\newtheorem{defi}{D\'efinition}%[section]
%\newtheorem{prop}{Proposition}[section]
%\newtheorem{lemm}{Lemme}[section]
} 

{%
\theoremstyle{break}
\theoremprework{\vspace{0.2cm}\begin{minipage}{\textwidth}\setlength{\parskip}{0.15cm}} %Pris en compte uniquement pour le premier newtheorem
\theorempostwork{\ligneinter\end{minipage}} %Même chose
\theoremheaderfont{\bfseries}
\theorembodyfont{\itshape}
\theoremseparator{ :\newline\vspace{0.2cm}}
%\newtheorem{lemm}{Lemme}[section]
\newtheorem{prop}{Proposition}[section]
}

{%
\theoremstyle{break}
\theoremprework{\vspace{0.2cm}\begin{minipage}{\textwidth}} %Pris en compte uniquement pour le premier newtheorem
\theorempostwork{\etoile\end{minipage}} %Même chose
\theoremheaderfont{\bfseries}
\theorembodyfont{\itshape}
\theoremseparator{ :\newline\vspace{0.2cm}}
%\newtheorem{lemm}{Lemme}[section]
\newtheorem{prefi}{Proposition-d\'efinition}[section]
}

{%
\theoremstyle{break}
%\theoremprework{\begin{tabular}{|p{\textwidth}|}\hline}
%\theorempostwork{\\ \hline\end{tabular}}
\theoremheaderfont{\scshape}
\theorembodyfont{\normalfont}
\theoremseparator{ :\newline\vspace{0.2cm}}
%\newtheorem{lemm}{Lemme}[section]
\newframedtheorem{theo}{Th\'eor\`eme}[section]
}

{%
\theoremstyle{break}
\theoremprework{\vspace{0.2cm}\begin{minipage}{\textwidth}\setlength{\parskip}{0.1cm}}
\theorempostwork{\ligneinter\end{minipage}}
\theoremheaderfont{\scshape}
\theorembodyfont{\itshape}
\theoremseparator{ :\newline\vspace{0.2cm}}
\newtheorem{lemm}{Lemme}[theo]
}
  
{%
\theoremstyle{break}
\theoremprework{\setlength{\parskip}{0.1cm}}%\begin{tabular}{|p{\textwidth}|}\hline}
%\theorempostwork{\\ \hline\end{tabular}}
\theoremheaderfont{\scshape}
\theorembodyfont{\itshape}
\theoremseparator{ :\newline\vspace{0.2cm}}
%\newtheorem{lemm}{Lemme}[section]
\newframedtheorem{coro}{Corollaire}[theo]
}



{
\theoremstyle{break}
\theoremprework{\vspace{0.5cm}}
\theorempostwork{\vspace{0.5cm}\ligneinter}
\theoremheaderfont{\scshape}
\theorembodyfont{\normalfont\small}
\theoremseparator{ :\newline\vspace{0.2cm}}
\newtheorem{exem}{Exemple}[section]
} 

{
\theoremstyle{break}
\theoremprework{\vspace{0.5cm}\begin{minipage}{\textwidth}}
\theorempostwork{\end{minipage}\ligneinter}
\theoremheaderfont{\scshape}
\theorembodyfont{\small}
\theoremseparator{ :\newline\vspace{0.2cm}}
\newtheorem{rema}{Remarque}[section]
} 