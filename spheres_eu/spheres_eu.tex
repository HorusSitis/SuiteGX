\chapter{Cas des sph\`eres %euclidiennes 
$\mathbb{S}^n$}

\emph{%
Dans ce paragraphe, on classifie compl\g{e}tement, pour tout entier positif $n$, les $G-$fibr\'es au-dessus de $\mathbb{S}^n$ %
\g{a} l'aide de la topologie du groupe de structure $G$, connexe par arcs, choisi. %
Le cas o\g{u} $G=\mathbb{U}$ et $n=3$, particuli\g{e}rement \'eclairant, sera ensuite \'etudi\'e de fa\c con d\'etaill\'ee.%
}

%\setcounter{subsection}{-1}

\section{Pr\'eliminaires topologiques : homotopie}

\subsection{Rel\`evement homotopiques}

La d\'efinition suivante est issue de \cite{NaberF}.

\begin{defi}[Rel\g{e}vement homotopique sur un fibr\'e localement trivial]
Soit \Fiy un fibr\'e localement trivial. Soit de plus $Z$ un espace topologique.

\par
On dit que \Fiy poss\g{e}de la propri\'et\'e de rel\g{e}vement homotopique relativement \g{a} $Z$ si et seulement si, %
quelles que soient les application continues $f$ et $F$, de $Z$ dans $X$ et de $Z\times [0,1]$ dans $X$ respectivement, v\'erifiant : 

{\em%
\begin{description}
\item[Homotopie :] il existe une homotopie $F$ dans $Z$ \`a valeurs dans $X$, telle que :
\[\forall y \in Z , F(y,0) = f(y)\]
%
\item[Rel\g{e}vement :] il existe une application $\tilde{f}$ de $Z$ dans $P$ telle que le diagramme suivant soit commutatif
\[\xymatrix{ &P \ar[d]^{\mathcal{P}} \\ Z \ar[r]^{f} \ar[ru]^{\tilde{f}} & X}\]
\end{description}
}

$F$ admet un rel\`evement $\tilde{F}$ sur $P$, autrement dit une homotopie $\tilde{F}$ d\'efinie sur $[0,1]^n \times [0,1]$ et \`a valeurs dans $P$, %
telle que $F = \mathcal{P} \circ \tilde{F}$, qui rend commutatif le diagramme :
\[\xymatrix{ &P \ar[d]^{\mathcal{P}} \\ Z \times [0,1] \ar[r]^{F} \ar[ru]^{\tilde{F}} & X}\]
\end{defi}

\begin{exem}
\begin{enumerate}
\item Dans l'annexe \ref{an_y}, on montre -th\'eor\g{e}me \ref{rchefiy}- qu'un fibr\'e \Fiy %
poss\g{e}de la propri\'et\'e de rel\g{e}vement homotopique relativement \g{a} un singleton.
%
\item Plus g\'en\'eralement, on montre \cite{NaberF} qu'un fibr\'e localement trivial \Fiy %
poss\g{e}de la propri\'et\'e de rel\g{e}vement homotopique relativement \g{a} tout cube $[0,1]^n$.
\end{enumerate}
\end{exem}

La d\'emonstration de la deuxi\g{e}me propri\'et\'e, valable pour tout fibr\'e localement trivial \Fiy , %
utilise les deux propri\'et\'es de l'espace $[0,1]^n$ muni de sa topologie usuelle :

\begin{enumerate}[label=(A\arabic *)]
\item $[0,1]^n$ est compact ;
\item Quel que soit l'ensemble ferm\'e $A$ de $[0,1]^n$, et quelque soit le voisinage ouvert $V$ de $A$, %
il existe un voisinage $W$ et $A$ tel que $\overline{W}\subseteq V$ ;
\item Deux ferm\'es $F_1$ et $F_2$ de $[0,1]^n$ peuvent \^etre s\'epar\'es par une fonction num\'erique autremant dit il existe, %
quels que soient les ferm\'es $F_1$ et $F_2$ une application continue $\mu$ de $[0,1]^n$ dans $[0,1]$, %
\'egale \g{a} $0$ en tout point de $F_1$ et \g{a} $1$ en tout point de $F_2$.
\end{enumerate}

D'apr\g{e}s \cite{BouTG}, chapitre IX, les deux derniers axiomes sont \'equivalents \g{a} : \og{} $[0,1]^n$ est un espace normal \fg{}. %
D'apr\g{e}s la m\^eme r\'ef\'erence, tout espace topologique compact ou m\'etrisable est normal.

\par
On prouve donc, dans \cite{NaberF} :

\begin{theo}
Soit \Fiy un fibr\'e localement trivial.

\par
Alors, quel que soit l'espace topologique compact $Z$, \Fiy poss\g{e}de la propri\'et\'e de rel\g{e}vement topologique relativement \g{a} $Z$.
\end{theo}

%\subsubsection{Construction de sections globales}

\subsection{Sph\g{e}res et boules}

\begin{prop}\label{hsd}
Soient $n$ un entier naturel non nul, $Y$ un espace topologique, et $f$ une application continue de $\mathbb{S}^{n-1}$ dans $Y$.

\par
Alors $f$ est homotope à une application constante de $\mathbb{S}^{n-1}$ dans $Y$ si et seulement si %
elle se prolonge continûment en une application continue $\tilde{F}$ du disque unité $D^n$ de l'espace euclidien de dimension $n$, vers $Y$.
\end{prop}

\section{Classification des $G-$fibr\'es au-dessus de $\mathbb{S}^n$}

Dans la sous-section qui suit, $n$ est un entier positif et $G$ un groupe topologique connexe par arcs.

\subsection{Fonction caract\'eristique}
%Soit $n$ un entier naturel non nul.
%
%Venons-en maintenant à la classification des fibrés principaux au-dessus des sphères euclidiennes.\\

On note, dans la suite de ce paragraphe, $n$ un entier naturel non nul, et on se fixe un groupe topologique $G$ connexe par arcs. %
On s'intéresse aux $G$-fibrés principaux au-dessus de $\mathbb{S}^n$, nous allons construire un invariant complet pour ces objets.

\par
Soit $P \xrightarrow{\mathcal{P}} \mathbb{S}^n$ un tel fibré. On définit, pour un réel strictement positif $\varepsilon$ :
\[
V_1^{\varepsilon} := \{(x_i) \in \mathbb{S}^n | x_{n+1} \in ]-\varepsilon , 1]\}%
\]
%
\[
V_2^{\varepsilon} := \{(x_i) \in \mathbb{S}^n | x_{n+1} \in [-1,\varepsilon [ \}%
\]
avec le repérage canonique de $\mathbb{S}^n$ dans $\mathbb{R}^{n+1}$.

\par
Ces ouverts sont deux calottes de la sphère, nord et sud respectivement, %
chacune incluant l'équateur $\mathbb{S}^{n-1}$ et empiétant légèrement sur l'hémisphère opposé. %
En particulier ces deux ouverts s'intersectent sur la bande $\mathcal{B}$ de $\mathbb{S}^n$, %
dont les points sont repérés dans $\mathbb{R}^{n+1}$ avec une dernière coordonnée comprise, strictement, entre $-\varepsilon$ et $\varepsilon$.

\par
On remarque que la projection stéréographique $\varphi_S$ de $\mathbb{S}^n$, depuis son pôle nord, envoie $\mathcal{B}$ dans la couronne :
\[
\left\{ X \in \mathbb{R}^n | \|X\| \in \left] \left(\frac{1-\varepsilon}{1+\varepsilon}\right)^{\frac{1}{2}} , %
\left(\frac{1+\varepsilon}{1-\varepsilon}\right)^{\frac{1}{2}} \right[ \right\}
\]

Par convexité cette couronne se rétracte, de manière homotopique, sur la sphère-unité de $\mathbb{R}^n$, %
il existe donc une rétraction continue $r$ de $\mathcal{B}$ sur l'équateur $\mathbb{S}^{n-1}$ de $\mathbb{S}^n$.

\par
Les adhérences respectives de $V_1^{\varepsilon}$ et $V_2^{\varepsilon}$ sont, via les projections stéréographiques $\varphi_S$ et $\varphi_N$ respectivement, %
homéomorphes à une boule fermée de l'espace euclidien $\mathbb{R}^n$, donc également au cube $[0,1]^n$.

\par
On en déduit, pour les fibrés induits par $G \hookrightarrow P \xrightarrow{\mathcal{P}_{\Box}} \mathbb{S}^n$ sur ces deux fermés, %
l'existence de deux sections globales $\tilde{s_1}$ et $\tilde{s_2}$.

\par
On note encore $\tilde{s_1}$ et $\tilde{s_2}$ les restrictions aux ouverts $V_1^{\varepsilon}$ et $V_2^{\varepsilon}$ de ces sections de fibré. On note :
\[(V_1^{\varepsilon},(\mathcal{P},\tilde{\psi_1}))\text{ et }(V_2^{\varepsilon},(\mathcal{P},\tilde{\psi_2}))\]

les deux trivialisations ainsi définies pour notre fibré, elles recouvrent la base $\mathbb{S}^n$ comme $\varepsilon$ est strictement positif.

\par
Intéressons-nous aux fonctions de transition pour ce fibré, associées à cette famille de trivialisations. On pose par définition :
\[
\forall x \in V_1^{\varepsilon} \cap V_2^{\varepsilon} , \tilde{\psi_2}(\tilde{s_2}(x)) = \tilde{g_{21}}(x) \tilde{\psi_1}(\tilde{s_1}(x))
\]
autrement dit par $G$-équivariance à droite :

\[\forall p \in \mathcal{P}^{-1}(V_1^{\varepsilon} \cap V_2^{\varepsilon}) , \tilde{\psi_2}(p) = \tilde{g_{21}}(x) \tilde{\psi_1}(p)\]
avec la notation habituelle pour $x$ et $p$.

\par
La propriété de cocycle permet de réduire notre connaissance de $(\tilde{g_{11}},\tilde{g_{12}},\tilde{g_{21}},\tilde{g_{22}}$ à la seule connaissance de $\tilde{g_{21}}$.

\par
On normalise maintenant la famille de fonctions de transitions associée aux deux trivialisations qui précèdent, %
au-dessus d'un point $x_0$ choisi arbitrairement sur l'équateur. On pose :
\[\tilde{g_{21}}(x_0) = a\text{ Déphasage : }\psi_1 = a \tilde{\psi_1}\]

Ainsi : $\tilde{\psi_2}(x_0) = \psi_1(x_0)$.

La famille $(g_{ji})$ des fonctions de transition définie par $(\psi_1,\tilde{\psi_2})$, %
d'après l'égalité qui précède et par propriété de cocycle vérifie : $g_{ji}(x_0) = e_G$, pour chacun de ses termes.

\par
On définit maintenant l'application :
\[T : (\mathbb{S}^{n-1},x_0) \rightarrow (G,e_G) : x \mapsto g_{21}(r(x))\]
que nous appellerons fonctions caractéristique pour le fibré.

L'existence de la rétraction $r$ permet de remonter, depuis une fonction continue définie à l'équateur, %
à une application continue de la bande $\mathcal{B}$ qui prend globalement les mêmes valeurs.

\par
D'après le théorème de reconstruction des fibrés principaux, %
on en déduit que toute application $T$ définie comme précédemment est caractéristique, pour un $G$-fibré au-dessus de $\mathbb{S}^n$.

\par
Cette construction, non univoque a priori, va nous permettre de définir un invariant d'équivalence, pour des $G$-fibrés au-dessus de $\mathbb{S}^n$.

\subsection{Un invariant complet pour la classification}

\begin{theo}\label{tinv}
Soient $G \hookrightarrow P_\tc{\Rom{1}} \xrightarrow{\mathcal{P}_{\Rom{1}}} \mathbb{S}^n$ et $G \hookrightarrow P_{\Rom{2}} \xrightarrow{\mathcal{P}_{\Rom{2}}} \mathbb{S}^n$ %
deux fibrés principaux sur un groupe $G$ connexe par arcs. %
On construit comme précédemment deux fonctions caractéristiques $T_{\Rom{1}}$ et $T_{\Rom{2}}$.
\par
Si les deux fibrés sont équivalents alors $T_{\Rom{1}}$ et $T_{\Rom{2}}$ sont homotopes relativement à $\{x_0\}$.
\end{theo}

%\begin{center}
%\emph{Nous avons ainsi construit un invariant d'\'equivalence pour les fibr\'es au-dessus des sph\`eres.}
%\end{center}

\begin{proof}
On adopte les notations usuelles pour les trivialisations des deux fibrés en écrivant leurs indices, $I$ et $II$, en exposant, %
on définit $\lambda_1$, $\lambda_2$ à la manière du paragraphe sur l'équivalence de fibrés, par la correspondance entre les deux familles de trivialisations.

\par
Par définition :
\[\forall x \in \mathcal{B} , g_{21}^{\Rom{2}}(x) = (\lambda_2(x))^{-1} g_{21}^{\Rom{1}}(x) \lambda_1(x)\]
ce qui entraîne :
\[\forall x \in \mathbb{S}^{n-1} , T_{\Rom{2}}(x) = (\lambda_2(x))^{-1} T_{\Rom{1}}(x) \lambda_1(x)\]

\etoile
On veut maintenant se débarasser des facteurs en $\lambda_i$, nous allons pour cela utiliser des homotopies sur $\mathbb{S}^{n-1}$ relativement à $x_0$.

\par
On remarque que $\lambda_1$ et $\lambda_2$ coïncident en $x_0$, nous noterons $a_0$ la valeur commune de ces applications en ce point.

\par
Afin de réaliser une homotopie de $T_{\Rom{2}}$ vers $T_{\Rom{1}}$ relativement à $x_0$ on plonge l'espace d'arrivée $\mathbb{S}^{n-1}$, %
respectivement dans les demi-sphères fermées, supérieure $D_1$ et inférieure $D_2$, de $\mathbb{S}^n$, %
dans lesquelles vivent resectivement $\lambda_1$ et $\lambda_2$.

\par
Chacune de ces demi-sphères fermées, que l'on peut envover sur une partie étoilée de $\mathbb{R}^n$ par projection stéréographique, %
est contractile donc d'après l'annexe \ref{an_y} : l'inclusion de $\mathbb{S}^{n-1}$ dans $D_1$, %
et l'application qui envoie chaque point de l'équateur sur $x_0$, toujours dans cet hémisphère, %
sont homotopes relativement à $x_0$, via une homotopie que nous noterons $H_1$.

\par
$\lambda_1 \circ H_1$ réalise une homotopie, sur $D_1$, entre $\lambda_1$ et l'application constante $\tilde{a_0}$ relativement à $x_0$, %
pointée en $a_0$ dans $G$. Nous notons $K_1$ cette homotopie.

\par
On définit, de même, une homotopie $K_2$ entre $\lambda_2$ et $\tilde{a_0}$ relativement à $x_0$.

\[\mathbb{S}^{n-1} \times [0,1] : (x,t) \mapsto (K_2(x,t))^{-1} T_{\Rom{1}}(x) K_1(x,t)\]
réalise une homotopie de $T_{\Rom{2}}$ vers $a_0^{-1} T_{\Rom{1}} a_0$, relativement à $x_0$.

\par
Enfin par hypothèse, il existe un chemin $\alpha$ de $a_0$ vers $e_G$,
\[(x,t) \mapsto \alpha(t)^{-1} T_{\Rom{1}}(x) \alpha(t)\]
réalise une homotopie de $a_0^{-1} T_{\Rom{1}} a_0$ vers $T_{\Rom{1}}$, relativement à $x_0$.
\end{proof}

\begin{theo}\label{tinvc}
L'invariant d\'efini au th\'eor\g{e}me \ref{tinv} est complet.
\end{theo}

\begin{proof}
On suppose construites, pour deux fibrés $G \hookrightarrow P_{\Rom{1}} \xrightarrow{\mathcal{P}_{\Rom{1}}} \mathbb{S}^n$ %
et $G \hookrightarrow P_{\Rom{2}} \xrightarrow{\mathcal{P}_{\Rom{2}}} \mathbb{S}^n$, %
les deux fonctions caractéristiques $T_{\Rom{1}}$ et $T_{\Rom{2}}$ selon la notation précédente, %
et on émet l'hypothèse que ces deux fonctions sont homotopes, sur $\mathbb{S}^{n-1}$, relativement à $x_0$.

\par
Montrons que nos deux fibrés sont équivalents, en utilisant le critère du théorème \ref{fbl}.

\par
On remarque que l'application $T_{\Rom{1}} T_{\Rom{2}}^{-1}$ de l'équateur $\mathbb{S}^{n-1}$ dans $G$ est homotope à l'application constante sur $\mathbb{S}^{n-1}$, %
de valeur le neutre $e_G$ de $G$ -on ne s'intéresse pas aux points-base $x_0$ et $e_G$.

\par
On montre avec la proposition \ref{hsd} et en composant par la projection stéréographique $\varphi_N$ %
que cette application se prolonge sur $D_1$ en une fonction $\nu$ continue, à valeurs dans $G$.

\par
On construit par recollement, sur $V_1^{\varepsilon}$, une application continue $\lambda_1$ et à valeurs dans $G$, telle que :
\[\forall x \in D_i , \lambda_1(x) = \nu(x)\text{ et }\forall x \in V_1^{\varepsilon} \cap U_2 , \lambda_1 (x) = g^{\Rom{1}}_{21}(x) (g^{\Rom{2}}_{21}(x))^{-1}\]
où $U_2$ est la demi-sphère ouverte sud de $\mathbb{S}^n$.

On munit alors les fibrés des trivialisations :

\[(V_1^{\varepsilon},(\mathcal{P}_{\Rom{1}},\psi_1^{\varepsilon ,\Rom{1}}))\text{ et }(U_2,(\mathcal{P}_1,\psi_2^{\Rom{1}}))\]
pour le premier,

\[(V_1^{\varepsilon},(\mathcal{P}_{\Rom{2}},\psi_1^{\varepsilon ,\Rom{2}}))\text{ et }(V_1^{\varepsilon},(\mathcal{P}_2,\psi_2^{\Rom{2}}))\]
pour le deuxième, où $\psi_2^{\Rom{1}}$ et $\psi_2^{\Rom{2}}$ sont les phases induites sur $U_2$, %
par les trivialisations des deux fibrés au-dessus de $V_2^{\varepsilon}$ que nous supposons construites préalablement à cette démonstration.

\par
On définit, de manière équivalente à ces familles de trivialisations, %
les deux fonctions de transition $g_{21}^{\Rom{1}}$ et $g_{21}^{\Rom{2}}$, sur l'intersection $V_1^{\varepsilon} \cap U_2$, par :

\[\forall p \in \mathcal{P}_{\Rom{1}}^{-1}(V_1^{\varepsilon}) , \psi_2^{\Rom{1}}(p) (\psi_1^{\varepsilon ,\Rom{1}}(p))^{-1} = g_{21}^{\Rom{1}}(\mathcal{P}_{\Rom{1}}(p))\]
et la formule symétrique pour le fibré $\Rom{2}$.

\par
Construisons enfin une équivalence $\tilde{f}$ du fibré $\Rom{1}$ vers le fibré $\Rom{2}$, à l'aide des deux couples de trivialisations locales qui précèdent :
\[\forall p \in \mathcal{P}_{\Rom{1}}^{-1}(V_1^{\varepsilon}) , \psi_1^{\varepsilon ,\Rom{2}}(\tilde{f}(p) = \lambda_1(\mathcal{P}_{\Rom{1}}(p)) \psi_1^{\varepsilon , \Rom{1}}(p)%
\text{ et }%
\forall p \in \mathcal{P}_{\Rom{1}}^{-1}(U_2) , \psi_2^{\Rom{2}}(\tilde{f}(p) = \lambda_1(\mathcal{P}_I(p)) \psi_2^{\Rom{1}}(p)\]

Montrons que cette définition est cohérente : il s'agit de vérifier que les deux formules qui précèdent sont équivalentes, au-dessus de $V_1^{\varepsilon} \cap U_2$.

\par
Par définition de $\tilde{f}$:
\[\forall p \in V_1^{\varepsilon} , \psi_2^{\Rom{2}}(\tilde{f}(p))(\psi_1^{\varepsilon , \Rom{2}}(\tilde{f}(p)))^{-1} = \psi_2^{\Rom{1}}(p) (\psi_1^{\varepsilon ,\Rom{1}}(p))^{-1} (\lambda_1(x))^{-1}\]

On en d\'eduit : $\forall p \in V_1^{\varepsilon} , \psi_2^{\Rom{2}}(\tilde{f}(p))(\psi_1^{\varepsilon ,\Rom{2}}(\tilde{f}(p)))^{-1} = g_{21}^{\Rom{1}}(\mathcal{P}_{\Rom{1}}(p))(\lambda_1(x))^{-1}$, puis :

\[\forall p \in V_1^{\varepsilon} , \psi_2^{\Rom{2}}(\tilde{f}(p))(\psi_1^{\varepsilon ,\Rom{2}}(\tilde{f}(p)))^{-1} = g_{21}^{\Rom{2}}(\mathcal{P}_{\Rom{2}}(\tilde{f}(p)))\]

comme les fibres au-dessus de $\mathcal{P_{\Rom{1}}}$ et $\mathcal{P}_{\Rom{2}}$ se confondent.
\end{proof}

\section{$U(1)\hookrightarrow\mathbb{S}^3\twoheadrightarrow\mathbb{S}^2$ et $SU(2)\hookrightarrow\mathbb{S}^7\twoheadrightarrow\mathbb{S}^4$}

\subsection{Une famille de $\mathbb{U}-$fibr\'es au-dessus de $\mathbb{S}^2$}

\subsubsection{Pr\'eambule : quotients par le groupe $\mathbb{U}_k$ des racines $k-$i\`emes de l'unit\'e}\label{ltk}

\textit{%
Dans cette section, on construit, \`a l'aide de chacune des projections $\mathcal{P}^+$ et $\mathcal{P}^-$ et en quotientant par le groupe $\mathbb{U}_k$, %
une famille de $\mathbb{U}-$ fibr\'es non isomorphes entre eux au-dessus de $\mathbb{S}^2$, index\'ee par l'ensemble des entiers naturels non nuls. %
Ici encore, nous nous cantonnerons \`a la projection $\mathcal{P}^+$ comme point de d\'epart, le cas \og{}n\'egatif\fg{} se traite de la m\^eme mani\`ere.%
}

\emph{Dans cette section et la suivante, $k$ d\'esigne un entier naturel non nul et $\mathbb{U}_k$ le groupe des racines $k-$i\`emes de l'unit\'e dans $\mathbb{C}$.}

\paragraph{Des endomorphismes du groupe topologique $\mathbb{U}$ :}~\\

\par
Voici un petit r\'esultat facile, mais int\'eressant pour la suite.
\par
L'endomorphisme $\omega\mapsto\omega^k$ du groupe topologique $\mathbb{U}$ se factorise canoniquement :
\[\xymatrix{%
\mathbb{U} \ar[r]^{\omega\mapsto\omega^k} \ar[d]_{:\mathbb{U}_k}&\mathbb{U}\\%
\dfrac{\mathbb{U}}{\mathbb{U}_k} \ar[ru]_{h_k}&%
}\]%
Puisque $\mathbb{U}$ est compact, c'est \'egalement le cas du quotient $\frac{\mathbb{U}}{\mathbb{U}_k}$ sur lequel ce groupe topologique se projette. %
$h_k$ est donc une bijection continue et ferm\'ee qui relie $\frac{\mathbb{U}}{\mathbb{U}_k}$ et $\mathbb{U}$, %
image de lui-m\^eme par le morphisme $\omega\mapsto\omega^k$ - pour v\'erifier sa surjectivit\'e et donc celle de $h_k$, il suffit de repr\'esenter un \'el\'ement de l'ensemble image par une exponentielle complexe, %
puis de diviser son argument par $k$.
\par
On peut en conclure que $h_k$ est un isomorphisme de groupes topologiques. En particulier, %
\emph{%
les groupes topologiques $\mathbb{U}$ et $\dfrac{\mathbb{U}}{\mathbb{U}_k}$ sont isomorphes.%
}
\ligneinter
\dots enrouler le groupe $\mathbb{U}$, en $k$ tours, sur lui-m\^eme.

\begin{rema}
On peut utiliser le th\'eor\`eme de rel\`evement et la monog\'en\'eit\'e des sous-groupes ferm\'es de $\mathbb{R}$ pour montrer que les \emph{seuls} %
endomorphismes non triviaux de $\mathbb{U}$ sont de la forme $\omega\mapsto\omega^k$ o\`u $k$ est un entier relatif.
\end{rema}

\paragraph{Cas de la sph\`ere $\mathbb{S}^3$}~\\

\par
L'action, par multiplication scalaire, de $\mathbb{U}$ sur $\mathbb{S}^3$ induit \'evidemment une action du groupe discret $\mathbb{U}_k$ sur cette sph\`ere. %
On peut donc quotienter $\mathbb{S}^3$ par $\mathbb{U}_k$, on note $[\; ]_k$ la projection canonique de la shp\`ere de dimension $3$ sur l'espace $\lt$ ainsi d\'efini, %
par analogie avec $\mathbb{S}^3\xrightarrow{[\; ]}\mathbb{P}^1(\mathbb{C})$.
\par
L'espace topologique, d\'efini \`a hom\'eomorphisme pr\`es par le quotient $\lt$ est un cas d'\emph{espace lenticulaire}, not\'e parfois $\mathcal{L}(k,1)$. %
Par exemple, $SO(3)$ ou $\lt[2]$, est hom\'eomorphe \`a $\mathcal{L}(2,1)$. Sch\'ematiquement, les complexes lenticulaires sont des sortes de \og{}sph\`eres tordues\fg{}.

\subsubsection{Construction d'une famille de $\mathbb{U}-$ fibr\'es au-dessus de $\mathbb{S}^2$}\label{lt1}

\paragraph{Actions de groupes sur le quotient $\lt$ :}~\\

\par
Nous allons construire une \emph{action-quotient}, simple, de $\dfrac{\mathbb{U}}{\mathbb{U}_k}$ sur $\lt$. Soit tout d'abord $\gamma$ l'action de $\mathbb{U}$, %
que nous noterons \`a droite par coh\'erence avec \dots , image de la multiplication scalaire par la projection $[\; ]_k$. %
Cette action poss\`ede un noyau non trivial : $\mathbb{U}_k$.

\par
En effet, soit : $\alpha\in\mathbb{U}_k$. Soit de plus $(z_1,z_2)$ un \'el\'ement de $\mathbb{S}^3$, qui d\'efinit un \'el\'ement $[z_1,z_2]_k$ de $\lt$. %
On note que $[z_1,z_2]_k=\text{Im}((z_1\cdot\omega ,z_2\cdot\omega ))_{\omega\in\mathbb{U}_k}$. %
De m\^eme, $\gamma ([z_1,z_2]_k,\alpha)=\text{Im}((z_1\alpha\cdot\omega ,z_2\alpha\cdot\omega ))_{\omega\in\mathbb{U}_k}$. %
Puisque $\mathbb{U}_k$ est globalement invariant par la translation \`a droite par son \'el\'ement $\alpha$, les deux images \'ecrites pr\'ec\'edemment sont identiques.

\par
R\'eciproquement, soit $\alpha$ un \'el\'ement de $\ker\gamma$. On remarque que : $\gamma([1,1]_k,\alpha)=[1,1]_k$ soit
\[\pl{\alpha}{\alpha}=\pl{1}{1}\]
En particulier : \[\exists\omega\in\mathbb{U}_k|\alpha\cdot\omega=1\]
Il s'ensuit que $\alpha=\omega^{-1}$, donc $\alpha\in\mathbb{U}_k$.
\par
On note $\gk$ l'action-quotient de $\gamma$ par son noyau $\mathbb{U}_k$, elle est simple et v\'erifie :
\[\gk([z_1,z_2]_k,\omega\cdot\mathbb{U}_k)=\gamma([z_1,z_2]_k,\omega)\]
autrement dit
\[\gk([z_1,z_2]_k,\omega\cdot\mathbb{U}_k)=\pl[\omega ']{z_1\omega}{z_2\omega}\]
L'action $\gk$ fait donc commuter le diagramme suivant :
\[%
\xymatrixrowsep{1in}              %%% Attention avec Quickpreview
\xymatrixcolsep{1in}
\xymatrix%
{%
\mathbb{U}\times\mathbb{S}^3\ar[d]_{\left(\cdot\mathbb{U}_k,\cdot\mathbb{U}_k\right)}\ar[r]_{\text{scalaire}}^{\text{multiplication}}&\mathbb{S}^3\ar[d]^{[\; ]}\\%
\dfrac{\mathbb{U}}{\mathbb{U}_k}\times\lt\ar@{-->}[r]_{\gk}&\lt%
}%
\]
%%% Modèle pris sur l'Internet
%\[
%\xymatrixrowsep{1in}
%\xymatrixcolsep{2in}
%\xymatrix
%{
%A\ar[r]\ar[d] & B\ar[d]\\
%C\ar[r] & D
%}
%\]
On en d\'eduit que $\gk\circ \left(\cdot\mathbb{U}_k,\cdot\mathbb{U}_k\right)$ est continue. %
Par ailleurs, $\left(\cdot\mathbb{U}_k,\cdot\mathbb{U}_k\right)$ est une application ouverte de $\mathbb{U}\times\mathbb{S}^3$ vers $\dfrac{\mathbb{U}}{\mathbb{U}_k}$ %
comme les projections canoniques de $\mathbb{U}$ et $\mathbb{S}^3$ sur leurs quotients respectifs par $\mathbb{U}_k$.
\par
Soit maitenant $O$ un ouvert de $\lt$. On note que
\[\left(\gk\circ\left(\cdot\mathbb{U}_k,\cdot\mathbb{U}_k\right)\right)^{-1}(O)=%
\left(\cdot\mathbb{U}_k,\cdot\mathbb{U}_k\right){-1}\left(\gk^{-1}(O)\right)\]
cet ouvert de $\mathbb{U}\times\mathbb{S}^3$ sera not\'e $O'$ .%
La surjectivit\'e de $\left(\cdot\mathbb{U}_k,\cdot\mathbb{U}_k\right)$ permet d'en d\'eduire que
\[\left(\cdot\mathbb{U}_k,\cdot\mathbb{U}_k\right)(O')=\gk^{-1}(O)\]
il s'ensuit que $\gk^{-1}(O)$ est un ouvert de $\dfrac{\mathbb{U}}{\mathbb{U}_k}\times\lt$ car $\left(\cdot\mathbb{U}_k,\cdot\mathbb{U}_k\right)$ est une application ouverte.

\par
Conclusion : $\gk$ est une action simple, \textbf{continue} de $\dfrac{\mathbb{U}}{\mathbb{U}_k}$ sur $\lt$.

\par
On peut enfin construire, \`a l'aide de $\gk$ et du paragraphe pr\'ec\'edent, une action simple $\delta_k$ de $\mathbb{U}$ sur $\lt$ en posant :
\[\forall ([z_1,z_2]_k,\omega)\in\lt\times\mathbb{U},\delta_k([z_1,z_2]_k,\omega)=\gk([z_1,z_2]_k,h_k^{-1}(\omega))\]
On remarque que :
\[\delta_k\left([z_1,z_2]_k,\omega^k\right)=\gamma([z_1,z_2]_k,\omega)\text{ soit }\delta_k\left([z_1,z_2]_k,\omega^k\right)=[z_1\cdot\omega,z_2\cdot\omega]_k\]
quels que soient $(z_1,z_2)$, $\omega$, dans $\mathbb{S}^3$ ou $\mathbb{U}$.

\par
\emph{Nous avons ainsi construit \textbf{deux} actions continues $\gk$ et $\delta_k$, \textbf{simples}, respectivement de $\dfrac{\mathbb{U}}{\mathbb{U}_k}$ et $\mathbb{U}$, %
\textbf{sur} $\mathbf{\lt}$.}

\par
\emph{L'action $\mathbf{\gk}$ agit \`a droite par \textbf{classes \`a gauche} modulo $\mathbb{U}_k$. %
Cela est possible car $\mathbb{U}_k$ est un sous-groupe \textbf{distingu\'e} de $\mathbb{U}$.}

\paragraph{D\'efinition, \`a l'aide de $\mathcal{P}^+$, d'une projection de $\lt$ sur $\mathbb{S}^2$ :}~\\

%\par
On remarque que $\mathbb{U}_k$ est un sous-groupe de $\mathbb{U}$ : les projections $\mathcal{P}^+$ et $\mathcal{P}^-$ se factorisent \`a droite par $[\; ]$ selon les diagrammes suivants :
\[%
\xymatrix{%
\mathbb{S}^3\ar[d]_{[\; ]_k}^{:\mathbb{U}_k}\ar[r]^{\mathcal{P}^+}_{:\mathbb{U}}&\mathbb{S}^2\\%
\lt\ar[ru]_{\mathcal{P}^{+k}}&%
}%
\qquad\text{et}\qquad%
\xymatrix{%
\mathbb{S}^3\ar[d]_{[\; ]_k}^{:\mathbb{U}_k}\ar[r]^{\mathcal{P}^-}_{:\mathbb{U}}&\mathbb{S}^2\\%
\lt\ar[ru]_{\mathcal{P}^{-k}}&%
}%
\]
Nous venons de d\'efinir deux projections, $\mathcal{P}^+$ et $\mathcal{P}^-$, de $\lt$ sur $\mathbb{S}^2$.

\par
Par ailleurs, la d\'efinition de $\gk$ d'une part, et la relation :
\[[z_1,z_2]_k\in\left(\mathcal{P}^{+k}\right)^{-1}(p)\Rightarrow (z_1,z_2)\in\left(\mathcal{P}^+\right)^{-1}(p)\]\label{inc}
valable pour tout point $p$ de $\mathbb{S}^2$ d'autre part, entra\^inent que les fibres de la projection $\mathcal{P}^{+k}$ %
sont exactement les orbites, dans $\lt$, de l'action $\gk$.

\par
La m\^eme chose est vraie, bien s\^ur, pour la projection $\mathcal{P}^{-k}$ ; d'ailleurs les fibres de deux projections sont aussi les orbites de $\lt$ sous l'action $\delta$.

\paragraph{Structures de fibr\'es principaux :}~\\

%\par
Nous allons encore construire une structure de fibr\'e principal, de groupe de structure isomorphe \`a $\mathbb{U}$, au-dessus de $\mathbb{S}^2$. %
On utilise pour cela la construction pr\'ec\'edente, d\'etaill\'ee ici seulement avec $\mathcal{P}^{+k}$ ; %
le cas de $\mathcal{P}^{-k}$ se traite de la m\^eme man\`ere et donne un autre fibr\'e de m\^eme espace de phases et m\^eme groupe de structure.

\par
Il suffit, pour notre construction, d'exhiber deux sections $\se{S}{+k}$ et $\se{N}{+k}$, %
les ouverts de trivialisation $U_S$ et $U_N$ que pour la structude de $\mathbb{U}-$fibr\'e d\'efinie \`a partir de $\mathcal{P}^+$.

\par
On note, tout d'abord, $\mathcal{U}_S^k$ et $\mathcal{U}_N^k$ les images r\'eciproques respectives, par $\mathcal{P}^{+k}$, des ouverts $U_S$ et $U_N$ de $\mathbb {S}^2$. %
%D'apr\`es la remarque pr\'ec\'edente concernant les fibres de $\mathcal{P}^{+k}$ et celles de $\mathcal{P}^+$, on note que $\mathcal{U}_s^k=[\mathcal{U}_S]_k$ et $\mathcal{U}_n^k=[\mathcal{U}_N^k]_k$.
Soit maintenant $\underline{q}\in\mathcal{U}_S$. On note d'apr\`es \ref{inc} que pour tout ant\'ec\'edent $q$ de $\underline{q}$ par la projection $[\; ]_k$, %
$q\in\mathcal{U}_S$. Nous venons de montrer que $\mathcal{U}_S^k\subseteq [\mathcal{U}_S]_k$. R\'eciproquement, soit $q\in\mathcal{U}_S$. %
Par d\'efinition de $\mathcal{P}^{+k}$, $\mathcal{P}^{+k}([q]_k)=\mathcal{P}^+(q)$ donc $\mathcal{P}^{+k}([q]_k)\in U_S$. Ainsi $[\mathcal{U}_S]_k\subseteq\mathcal{U}_S^k$.

\par
On \'etablit de la m\^eme mani\`ere que $\mathcal{U}_N^k=[\mathcal{U}_N]_k$.

\par
On pose maintenant, pour tout poit $p$ de $U_S$ : 
\[%
\left\{\begin{array}{rccl}%
\forall (\phi,\theta )\in U_S&\se{S}{+k}(p)&=&\pl{\cos\frac{\phi}{2}\ec{\theta}}{\sin\frac{\phi}{2}}\\%
\forall (\phi,\theta )\in U_N&\se{N}{+k}(p)&=&\pl{\cos\frac{\phi}{2}}{\sin\frac{\phi}{2}\ec{(-\theta )}}%
\end{array}\right.%
\]
Ainsi, les sections $\se{S}{+k}$ et $\se{N}{+k}$ sont d\'efinies de sorte que les diagrammes :
\[%
\xymatrix{%
\mathcal{U}_S\ar[d]_{[\; ]_k}&U_S\ar[l]_{\se{S}{+}}\ar[ld]^{\se{S}{+k}}\\%
\mathcal{U}_S^k&%
}%
%\qquad%
\text{\hspace{1.5cm}et\hspace{1.5cm}}%\qquad%
\xymatrix{%
\mathcal{U}_N\ar[d]_{[\; ]_k}&U_N\ar[l]_{\se{N}{+}}\ar[ld]^{\se{N}{+k}}\\%
\mathcal{U}_N^k&%
}%
\]
sont commutatifs.

\par
On d\'efinit alors naturellement les fonctions r\'eciproques de trivialisation $\Phi_S^{+k}$ et $\Phi_N^{+k}$. %
Tout d'abord, d'apr\`es l'observation qui pr\'ec\`ede concernant les fibres de $\mathcal{P}^{+k}$, on munit $\lt$ de l'action $\gk$. %
On pose ensuite, pour tout r\'eel $\xi$ :
\[\Phi_S^{+k}\left((\phi,\theta);\ec{\xi}\mathbb{U}_k\right)=\pla{\cos\frac{\phi}{2}\ec{\theta+\xi}}{\sin\frac{\phi}{2}\ec{\xi}}\]
ce qui s'\'ecrit encore :
\[\Phi_S^{+k}\left((\phi,\theta);\ec{\xi}\mathbb{U}_k\right)=\left[\st{\phi}{\theta+\xi}{\xi}\right]_k\]
soit :
\[\Phi_S^{+k}\left((\phi,\theta),\ec{\xi}\mathbb{U}_k\right)=\left[\Phi_S^+\left((\phi,\theta),\ec{\xi}\right)\right]_k\]
On remarque que $\se{S}{+k}=[\; ]_k\circ\se{S}{+}$ donc la section $\se{S}{+k}$ est continue. %
La continuit\'e de $\gk$ permet d'en d\'eduire que $\Phi_S^{+k}$ est continue ; le cas de $\Phi_N^{+k}$ se traite de la m\^eme mani\`ere.

\par
Il reste \`a prouver la continuit\'e des fonctions de phase $\psi_S^{+k}$ et $\psi_N^{+k}$, %
qui vont de $\mathcal{U}_S^k$ et $\mathcal{U}_N^k$ respectivement, vers $\dfrac{\mathbb{U}}{\mathbb{U}_k}$.

\par
On remarque que : \[\forall (z_1,z_2)\in\mathbb{S}^3 , %
\left\{\begin{array}{lcr}%
\psi_S^{+k}\left(\left[z_1,z_2\right]_k\right)&=&\psi_S^+(z_1,z_2)\cdot\mathbb{U}_k\\%
\psi_N^{+k}\left(\left[z_1,z_2\right]_k\right)&=&\psi_N^+(z_1,z_2)\cdot\mathbb{U}_k%
\end{array}\right\}\]
Autrement dit les deux diagrammes :
\[%
\xymatrix{\mathcal{U}_S\ar[d]_{[\; ]_k}\ar[r]^{\psi_S^+}&\mathbb{U}\ar[d]^{\cdot\mathbb{U}_k}\\%
\mathcal{U}_S^k\ar[r]_{\psi_S^{+k}}&\dfrac{\mathbb{U}}{\mathbb{U}_k}%
}%
\text{ et }%
\xymatrix{\mathcal{U}_N\ar[d]_{[\; ]_k}\ar[r]^{\psi_N^+}&\mathbb{U}\ar[d]^{\cdot\mathbb{U}_k}\\%
\mathcal{U}_N^k\ar[r]_{\psi_N^{+k}}&\dfrac{\mathbb{U}}{\mathbb{U}_k}%
}%
\]
sont commutatifs.

\par
Or les applications $\psi_S^+$ et $\psi_N^+$ sont continues d'apr\`es la sous-section \ref{fis}, %
de m\^eme que la projection canonique de $\mathbb{U}$ sur $\dfrac{\mathbb{U}}{\mathbb{U}_k}$ ; %
$[\; ]_k$ est une application ouverte.
\par
Le raisonnement utilis\'e pour d\'emontrer la continuit\'e de l'action $\gk$ permet de conculre que les fonctions $\psi_S^{+k}$ et $\psi_N^{+k}$ sont continues.
\etoile
\emph{Nous avons construit un fibr\'e principal de base $\mathbb{S}^2$ et de groupe de structure $\dfrac{\mathbb{U}}{\mathbb{U}_k}$, %
\`a l'aide des deux ouverts de trivialisation $U_S$ et $U_N$, l'espace de phases est $\lt$, muni de l'action $\gk$}.

\par
Pour construire un $\mathbb{U}-$fibr\'e \`a l'aide de la structure qui pr\'ed\`ede en utilisant l'identification entre $\dfrac{\mathbb{U}}{\mathbb{U}_k}$ et $\mathbb{U}$ \'etabli au d\'ebut de la section, %
on consid\`ere l'action $\delta_k$ de $\mathbb{U}$ sur $\lt$ et on compose les fonctions de phase $\psi_S^{+k}$ et $\psi_N^{+k}$ \`a gauche par $h_k$ ; %
on note $\left(\lt ,\mathbb{S}^2,\mathcal{P}^{+k},\mathbb{U}\right)$ le fibr\'e ainsi d\'efini.

\ligneinter
Bien s\^ur, la construction qui pr\'ec\`ede est possible \`a partir de la projection $\mathcal{P}^-$. %
On note que $\se{S}{-k}(\phi,\theta)=\pla{\cos\frac{\phi}{2}\ec{(-\theta)}}{\sin\frac{\phi}{2}}$ %
et $\se{N}{-k}(\phi,\theta)=\pla{\cos\frac{\phi}{2}}{\sin\frac{\phi}{2}\ec{\theta}}$.

\par
\emph{Il existe ainsi une famille $\left(\lt[\abs{k}],\mathbb{S}^2,\mathcal{P}^k,\mathbb{U}\right)_k$, index\'ee par $\mathbb{Z}\setminus\{0\}$, de fibr\'es principaux de base $\mathbb{S}^2$ et de groupe de structure $\mathbb{U}$. %
De plus, nous allons voir que ces fibr\'es sont \emph{non isomorphes} deux \`a deux.} -on identifie $\mathbb{S}^3$ \`a $\lt[1]$.

\paragraph{Fonctions de transition pour les fibr\'es $\left(\lt[\abs{k}],\mathbb{S}^2,\mathcal{P}^k,\mathbb{U}\right)_k$ : actions $\gk$ et $\delta_k$.}~\\

%\par
Les formules qui pr\'ec\g{e}dent concernant $\se{S}{+k}$, $\se{N}{+k}$, $\se{S}{-k}$ et $\se{N}{-k}$ nous donnent, avec la notation idoine pour $g_{SN}^{+k}$ et $g_{SN}^{-k}$ :
\[g_{SN}^{+k}(\phi ,\theta)=\ec{\theta}\cdot\mathbb{U}_k\text{ et }g_{SN}^{-k}(\phi ,\theta)=\ec{(-\theta)}\cdot\mathbb{U}_k\]
quels que soient les r\'eels $\phi$ et $\theta$ tels que : $\phi\in ]0,\pi[$.

\ligneinter
Les fonctions de transition $g_{SN}^{k,\delta_{\abs{k}}}$ des fibr\'es de la famille $\left(\lt[\abs{k}],\mathbb{S}^2,\mathcal{P}^k,\mathbb{U}\right)_{k\in\mathbb{Z}\setminus\{0\}}$ %
s'en d\'eduisent via le morphisme de groupes topologiques $h_k$ :
\[\boxed{\forall (\phi,\theta)\in ]0,\pi [,g_{SN}^{k,\delta_{\abs{k}}}(\phi,\theta)=\ec{k\theta}}\]

\begin{rema}
Les espaces topologiques de la famille $(\mathcal{L}(k,1))_{k\in\mathbb{N}^{\ast}}$, d\'efinis \g{a} hom\'eomorphisme pr\g{e}s, %
sont \'etudi\'es dans \cite{Lens}.
\end{rema}

\subsubsection{Rev\^etements et espaces lenticulaires :}

\subsection{$SU(2)-$fibr\'es au-dessus de $\mathbb{S}^4$}

%Document \'eponyme.
