\chapter{Fibr\'es localement triviaux}

\begin{defi}
Soient $X$ et $Y$ deux espaces topologiques s\'epar\'es.\\
Un fibr\'e localement trivial sur $X$, de fibre $Y$, est d\'efini par :
\begin{itemize}
\item un espace topologique $P$, appel\'e espace total,% et une action \`a droite, continue, de $G$ sur $P$,
\item une surjection continue $\mathcal{P}$ de $P$ sur $X$, %$G$-invariante autrement dit : $\forall (p,g) \in P \times G , \mathcal{P} (p \ast g) = \mathcal{P} (p)$,
\end{itemize}
tels que pour tout point $x$ de la base $X$ %
il existe un voisinage ouvert de $x$ et un hom\'eomorphisme $\Phi$ de $V\times Y$ sur $\mathcal{P}^{-1}(V)$ v\'erifiant :
%soit muni d'un voisinage ouvert $V$, associé à un homéomorphisme $\Phi$ de $\mathcal{P}^{-1}(V)$ sur $V \times G$ tel que:
\[\forall (v,y)\in v\times Y , \mathcal{P}(\Phi (v,y))=v\]
Un tel couple $(V, \Phi)$ est appel\'e trivialisation locale du fibr\'e -botte de fibres. %Nous étudierons explicitement le terme de phase $\psi$ dans les problÚmes de physique que nous rencontrerons.\\
On sh\'ematise par $Y \hookrightarrow P \overset{\mathcal{P}}{\twoheadrightarrow} X$ le fibr\'e ainsi d\'efini, %
que nous noterons formellement \Fiy.\\
On adoptera souvent la notation $\Phi$, $\Psi$ pour l'homorphisme de trivialisation et sa r\'eciproque.
\end{defi}

\begin{exem}[Rev\^etements]
Un rev\^etement est d\'efini par une projection $\mathcal{P}$, continue, d'un espace connexe $\tilde{X}$ sur un espace topologique $X$, %
telle que tout point $x$ de l'espace $X$ admet un voisinage $U_x$ v\'erifiant :
\[\mathcal{P}^{-1}(\{U_x\})=\underset{i}{\bigcup}U_x^i\]
o\`u $(U_x^i)_i$ est une famille, disjointe, d'ouverts de $\tilde{X}$ telle que pour tout $i$, $\mathcal{P}$ induit un hom\'eomorphisme de $U_x^i$ sur $U_x$.

On peut montrer qu'un rev\^etement d\'efinit un fibr\'e localement trivial, dont la fibre est discr\`ete.
\end{exem}

%\newpage

\section{Un peu de topologie}

\subsection{Rel\`evement d'applications}

Il est ais\'e de montrer que, lorsque \Fiy est un fibr\'e localement trivial, tel que $X$ et $Y$ sont connexe, alors l'espace total $P$ est connnexe. %
La d\'emonstration de la m\^eme propri\'et\'e concernant la connexit\'e par arcs est la premi\`ere occasion de relever une application continue d'un espace connexe, \`a valeurs dans $X$, %
par une fonction \`a valeurs dnas $P$.

\begin{prop}\label{rchefiy}
Soient \Fiy un fibr\'e localement trivial de fibre $Y$ connexe par arcs, et $\alpha$ un chemin de $X$, %
c'est- \`a-dire une application continue de $[0,1]$ dans $X$. %
Soit de plus $\tilde{x_0}$ un \'el\'ement de la fibre de $\alpha (0)$.

Alors il existe un chemin $\tilde{\alpha}$ de $P$ tel que :
\begin{itemize}
\item $\tilde{\alpha}(0)=\tilde{x_0}$
\item $\forall t \in [0,1] , \mathcal{P}(\tilde{\alpha}(t))=\alpha (t)$
\end{itemize}
\end{prop}

\begin{proof}[\es]
$V_i$ trivialisation au-dessus des \'el\'ements de $\alpha ([0,1])$, %
$V_j$ fini Borel-lebesgue puis $U_j$ images r\'eciproques, %
$(I_k)$ avec les composantes connexes et encore Borel-Lebesgue, famille simple quitte \`a restreindre,
puis on construit $(J_l)_l$ intervalles ouverts, au plus deux se rencontrent simultan\'ement ...
\end{proof}

Si \Fiy est un rev\^etement, on d\'emontre que ce rel\`evement est unique. On peut m\^eme facilement d\'emontrer un peu plus :

\begin{prop}\label{urc}
Soient $\tilde{X} \overset{\mathcal{P}}{\longrightarrow} X$ un rev\^etement, $Z$ un espace topologique connexe, %
$z_0$, $x_0$ et $\tilde{x_0}$ des \'el\'ements de $Z$, $X$ et $\tilde{X}$ respectivement. On suppose de plus que : $\tilde{x_0}\in\mathcal{P}^{-1}(\{x_0\})$.

Si $f$ est une application continue de $Z$ dans $X$, qui envoie $z_0$ sur $x_0$, et si $f$ admet un rel\`evement vers $P$ qui vaut $\tilde{x_0}$ en $z_0$, %
alors ce rel\`evement est unique.
\end{prop}

\begin{rema}
Une application continue $f$ entre deux espaces topologiques $X$ et $Y$, telle que $f(x_0)=y_0$ %
pour deux \'el\'ements de $X$ et $Y$ respectivement, est appel\'ee application continue entre les espaces point\'es $(X,x_0)$ et $(Y,y_0)$.

On peut reformuler la proposition ci-dessus : %
soient $\tilde{X}\overset{\mathcal{P}}{\longrightarrow} X$ un rev\^etement, $Z$ un espace topologique connexe, $z_0$, $x_0$ et $\tilde{x_0}$ des \'el\'ements de $Z$, $X$ et $\tilde{X}$ respectivement. %
Si $f$ est une application continue entre les espaces point\'es $(Z,z_0)$ et $(X,x_0)$ qui admet un rel\`evement $\tilde{f}$, d\'efini entre les espaces point\'es $(\tilde{X},\tilde{x_0})$, alors ce rel\`evement est unique.
\end{rema}

L'existence s'\'etablit avec le lemme de Zorn :

\begin{prop}
Avec les hypoth\`eses de la proposition qui pr\'ec\`ede concernant $f$, $z_0$ et $x_0$, il existe (au moins) une application continue $\tilde{f}$ de $Z$ vers $P$, %
qui vaut $\tilde{x_0}$ en $z_0$, qui rel\`eve $f$ au sens o\`u :
\[\forall z\in Z, \mathcal{P}(\tilde{f}(z))=f(z)\]
\end{prop}

\begin{proof}[\es]
Rel\g{e}vements locaux, un rel\g{e}vement d\'efini sur un ouvert strictement inclus dans $Z$ n'est pas maximal.
\end{proof}

%\begin{rema}
%Dans la d\'emonstration qui concerne un chemin de la base d'un fibr\'e, %
%de m\^eme que dans une autre d\'emonstration que nous verrons pour le rel\`evement d'une homotopie dans un rev\^etement, %
%le recours au lemme de Zorn est \'evit\'e gr\^ace \`a un argument de compacit\'e.
%\end{rema}

La proposition suivante sera utile pour la sous-sous section \ref{gs1}.% Un argument de compacit\'e permet d'\'eviter, dans sa d\'emonstration, le recours au lemme de Zorn.

\begin{prop}\label{red2}
Soient $\tilde{X}\overset{\mathcal{P}}{\longrightarrow} X$ un rev\^etement, tel que $\mathcal{P}(\tilde{x_0})=x_0$ por deux \'el\'ements de $\tilde{X}$ et $X$. %
Soit de plus $F$ une application continue entre les espaces point\'es $([0,1]^2,(0,0))$ et $(X,x_0)$.

Alors il existe un rel\g{e}vement $\tilde{F}$ de $F$ \g{a} valeurs dans $\tilde{X}$, tel que : $\tilde{F}((0,0))=\tilde{x_0}$.
\end{prop}

\begin{proof}[\re]
\end{proof}

\begin{rema}
\begin{enumerate}
\item L'argument qui pr\'ec\g{e}de, o\g{u} le recours au lemme de Zorn est \'evit\'e \g{a} l'aide de la construction de $(V_n)_{n\in [\![1,N]\!]}$, %
est en fait valable si l'on prend n'importe quel espace topologique connexe et compact $Z$ au lieu de $[0,1]^2$.
\item D'apr\g{e}s la proposition \ref{urc}, le rel\g{e}vement ainsi construit est unique.
\end{enumerate}
\end{rema}

\bigskip
Mentionnons mainenant une cons\'equence connue de la proposition \ref{rchefiy} :

\begin{prop}[Th\'eor\`eme de rel\`evement, exponentielle complexe]
Soit $f$ une application continue, d\'efinie sur un un intervalle r\'eel $I$, qui prend ses valeurs dans $\mathbb{U}$. Soient \'egalement $t_0$ un \'el\'ement de $I$, et $\theta_{t_0}$ un argument de $f(t_0)$.

Il existe au moins une application continue $\theta$ de $I$ dans $\mathbb{R}$, telle que :
\[\forall t \in I, f(t)=\exp (i\theta (t))\]
\end{prop}

Pour m\'emoire, l'exponentielle complexe $\theta \mapsto \exp (i\theta )$ d\'efinit un rev\^etement de $\mathbb{U}$ par $\mathbb{R}$, %
la fibre est ici hom'eoorphe \`a $\mathbb{Z}$. C'est m\^eme un fibr\'e principal, de groupe de structure $2\pi\mathbb{Z}$.

\begin{proof}
Nous connaissons d\'ej\`a le cas o\`u $I$ est un segment.

Supposons : $I=[a,b[$, o\`u $a$ est un nombre r\'eel, $b$ un nombre sup\'erieur \`a $a$, ou $+\infty$. Soit $\theta_a$ un argument de $f(a)$\\
Soit $(b_n)_{n\in\mathbb{N}^{\ast}}$ la suite r\'eelle, strictement croissante, d\'efinie pour tout $n$ strictement positif par :
\begin{itemize}
\item $b_n=b-\frac{b-a}{n+1}$ si $b$ est fini,
\item $b_n=a+n$ si $b$ est infini.
\end{itemize}
Dans les deux cas : $[a,b[=\underset{n\in\mathbb{N}^{\ast}}{\bigcup}[a,b_n[$.

Il existe un rel\`evement $\tilde{f}_1$ de $f$ sur $[a,b_1]$,  tel que : $\tilde{f}_1(a)=\theta_a$.\\
Supposons construit, pour un entier naturel $n$ non nul, un rel\`evement $\tilde{f}_n$ de $f$, d\'efini sur $[a,b_n]$, tel que : $\tilde{f}_n(a)=\theta_a$.\\
Toujours d'apr\`es la proposition \ref{rchefiy}, %
il existe une application $\tilde{f}_{b{n+1}}$ de $[b_n,b_{n+1}]$ dans $\mathbb{R}$, continue, telle que :
\[\forall t \in [b_n,b_{n+1}], \exp (i\tilde{f}_{b_{n+1}}(t))=f(t) \text{ et }\tilde{f}_{b_{n+1}}(b_n)=\tilde{f}_n(b_n)\]
On peut ainsi d\'efinir, par recollement sur $[a,b_{n+1}]$, un rel\`evement continu $\tilde{f}_{n+1}$ de $f$, qui vaut $\theta_a$ en $a$.

Nous avons ainsi construit, par r\'ecurrence, une suite $(\tilde{f}_n)$ de rel\`evements de $f$ qui valent $\theta_a$ en $a$, et qui v\'erifie de plus :
\[\forall n \in\mathbb{N}^{\ast}, \forall t \in [a,b-n], \tilde{f}_n(t)=\tilde{f}_{n+1}(t)\]
Cette propri\'et\'e de coh\'erence se g\'en\'eralise ais\'ement entre deux fonctions $\tilde{f}_m$ et $\tilde{f}_n$, %
d\`es que $m$ est inf\'erieur \`a $n$, sur le domaine de d\'efinition de $\tilde{f}_m$.

On peut donc d\'efinir, sur l'intervalle $[a,b[$, le prolongement commun $\tilde{f}$ de tous les termes de $(\tilde{f}_n)_n$, %
ce qui nous donne le rel\`evement souhait\'e pour $f$.

Un raisonnement similaire permet de r\'esoudre le cas d'un intervale ferm\'e \`a gauche, et ouvert \`a droite; %
ceci permet enfin de r\'egler le cas d'un intervalle ouvert de $\mathbb{R}$.
\end{proof}

\subsection{Homotopie}

Voici quelques g\'en\'eralit\'es qui concernent l'homotopie entre des fonctions continues, la d\'efinition qui suit est utile dans toutet la suite du paragraphe.

\begin{defi}
Soient $X$ et $Y$ deux espaces topologiques, soit de plus $A$ une partie de $X$, \'eventuellement vide.

Une \textbf{homotopie}, relativement \`a $A$, entre deux applications coninues $f$ et $g$ de $X$ vers $Y$, %
qui co\"incident sur $A$, %
est une application $H$ de $X\times [0,1]$ dans $Y$, continue, telle que :
\begin{itemize}
\item $\forall x \in X, H(x,0)=f(x) \wedge H(x,1)=g(x)$
\item $\forall t \in [0,1] , \forall a \in A , H(a,t)=f(a)$
\end{itemize}
Si une telle homotopie existe, on dit que $f$ et $g$ sont homotopes relativement \`a $A$, %
et on note : $f\simeq g \text{rel} A$.
\end{defi}

\begin{prop}
Avec la notation qui pr\'ec\`ede, l'homotopie relativement \`a $A$ d\'efinit une relation d'\'equivalence entre les applications continues de $X$ vers $Y$.
\end{prop}

Les deux exemples qui suivent sont importants :

\begin{exem}[Homotopie libre]
Lorsque $A$ est vide, l'homotopie est dite \textbf{libre}, on dit simplement que $f$ et $g$ sont homotopes et on note : $f\simeq g$.

L'ensemble des clases d'\'equivalence d'applications continues entre $X$ et $Y$ est not\'e $[X,Y]$.
\end{exem}

\begin{exem}[Homotopie de chemins]
Cette fois-ci $X$ est le segment $[0,1]$, $\alpha$ et $\alpha '$ sont deux chemins de $Y$.
%$A$ est choisi comme \'etant l'ensemble $\{0,1\}$.\\

Sauf mention explicite du contraire, une homotopie entre $\alpha$ et $\alpha '$ d\'esigne toujours une homotopie, relativement \`a $\{0,1\}$, %
entre les applications continues $\alpha$ et $\alpha '$ de $[0,1]$ vers $Y$. On note encore $\alpha\simeq\alpha ' \text{rel}\{0,1\}$.

Pour qu'une telle homotopie existe, il faut en particulier que $\alpha(0)=\alpha '(0)$ et $\alpha(1)=\alpha '(1)$.
\end{exem}

\begin{defi}
Une application continue entre deux espaces topologiques $X$ et $Y$, qui est homotope \`a une application constante de $X$ vers $Y$, %
est dite homotope \`a z\'ero.
\end{defi}

Les espaces qui suivent sont courants en topologie alg\'ebrique, nous les retrouvons dans la classification des fibr\'e principaux.

\begin{prefi}[Espace contractile]
Un espace topologique $Y$ est dit contractile lorqu'il v\'erifie l'une des deux caract\'erisations, \'equivalentes, qui suivent :
\begin{itemize}
\item[C1] $Id_Y$ est homotope \`a z\'ero ;
\item[C2] deux applications continues d'un espace topologique $X$ dans $Y$ sont toujours homotopes entre elles.
\end{itemize}
\end{prefi}

La deuxi\`eme caract\'erisation entra\^ine notamment qu'un espace contractile est connexe par arcs.

L'homotopie est \'egalement une relation entre les espaces topologiques :

\begin{defi}
Une application continue $h$ entre deux espaces topologiques $X$ et $Y$ est appel\'ee \'equivalence homotopique si et seulement si %
il existe une application continue $h'$ de $Y$ vers $X$ telle que :
\begin{itemize}
\item $h'\circ h\simeq Id_X$ ;
\item $h\circ h'\simeq Id_Y$.
\end{itemize}
La relation ainsi d\'efinie sur tout ensemble d'espaces topologiques, appel\'ee \'equivalence homotopique, est une relation d'\'equivalence.
\end{defi}

\begin{exem}
Un hom\'eomorphisme entre deux espaces topologiques $X$ et $Y$ est une \'equivalence homotopique entre $X$ et $Y$.
\end{exem}

Bien s\^ur, deux espaces topologiques homotopiquement \'equivalents ne sont pas n\'ecessairement hom\'eomorphes, c'est l'int\'er\^et de la d\'efinition qui pr\'ec\`ede. %
Voici un exemple important qui illustre ce fait :

\begin{prop}
Un espace contractile est homotopiquement \'equivalent \`a un singleton, muni de sa -seule- topologie discr\`ete.
\end{prop}

\subsection{Le groupe fondamental}

Dans ce paragraphe, on \'etablit des propri\'et\'es alg\'ebriques sur les lacets dans des espaces topologiques. La structure de groupe fondamental sur un espace point\'e est particuli\g{e}remant int\'eressante.

\begin{defi}
Soit $\alpha$ un chemin qui prend ses valeurs dans un espace topologique $X$. $\alpha '(0)$ et $\alpha (1)$ sont respectivement appel\'es origine et extr\'emit\'e.
\end{defi}

\begin{exem}
Soient $\alpha$ et $\beta$ deux chemins homotopes relativement \g{a} $\{0,1\}$, via une homotopie de chemins $H$ \g{a} valeurs dans un espace topologique $X$.

Alors les chemins interm\'ediaires $(s\mapsto H(s,t)_t$ ont tous les m\^emes origines et extr\'emit\'es.
\end{exem}

Lorsque $\alpha$ et $\beta$ sont deux chemins tels que $\alpha (1)=\beta (0)$, on peut les composer, par concat\'enation : le chemin $\alpha \beta$ est d\'efini par les formules
\[\forall s\in \left[0,\frac{1}{2}\right] , \alpha \beta (s) = \alpha (2s)\]% \text{ et } 
et
\[\forall s \in \left[\frac{1}{2} , 1\right] , \alpha \beta (s) = \beta (2s-1)\]

On peut aussi d\'efinir un inverse $\alpha^{\leftarrow}$ pour un chemin $\alpha$ :
\[\forall s \in [0,1] , \alpha^{\leftarrow} (s) = \alpha (1-s)\]

Les cons\'equences en termes d'homotopie de ces deux d\'efinitions sont illustr\'ees dans la proposition suivante :

\begin{prop}
Soient $\alpha$ et $\beta$ deux chemins sunr un espace topologique $X$. On suppose $\alpha (1)$ \'egal \g{a} $\beta (0)$.
\begin{itemize}
\item Si $\alpha '$ et $\beta '$ sont deux chemins %
respectivement homotopes $\alpha$ et $\beta$, alors $\alpha \beta$ est homotope \g{a} $\alpha '\beta '$.
\item Si $\alpha '$ est un chemin homotope \g{a} $\alpha$, alors $\alpha '^{\leftarrow}$ est homotope \g{a} $\alpha^{\leftarrow}$.
\end{itemize}
\end{prop}

Pour tout chemin $\alpha$ d'un espace topologique $X$, %
on note $[\alpha ]$ sa classe d'homotopie relativement \g{a} $\{0,1\}$. %
La proposition qui pr\'ec\g{e}de permet de d\'efinir une loi de composition entre les classes d'homotpie de chemins. %
Le produit $[\alpha ][\alpha^{\leftarrow}]$ est ainsi la classe d'homotopie du chemin constant $\tilde{\alpha (0)}$.

Les lacets sont des chemins qui b\'en\'eficient de propri\'et\'es alg\'ebriques encore plus riches.

\begin{defi}[Lacets]
Soit $(X,x_0)$ un espace topologique point\'e.

Un chemin de $X$, dont l'origine et l'extr\'emit\'e se situent en $x_0$ est appel\'e lacet, ou boucle, en $x_0$.
\end{defi}

\begin{rema}
Avec la notation qui pr\'ec\g{e}de, les chemins interm\'ediaires r\'ealis\'es par une homotopie entre $\alpha$ et un autre lacet $\alpha '$ %
sont encore des lacets en $x_0$. C'est en particulier le cas de $\alpha '$.
\end{rema}

\begin{prefi}[Groupe fondamental]
Soit $(X,x_0)$ un espace topologique point\'e. %
Alors l'ensemble $[([0,1],\{0,1\});(X,x_0)]$ des classes d'homotopie de lacets de $X$ en $x_0$, %
muni de la loi de composition des lacets -concat\'enation-, est un groupe, appel\'e %
\textbf{groupe fondamental de l'espace $X$ en $x_0$.} Son \'el\'ement neutre est le lacet constant dont la valeur est $x_0$.

On le note $\pi_1(X,x_0)$
\end{prefi}

\es Il suffit maintenant d'\'etablir l'associativit\'e de la loi de concat\'enation, \'etendue aux classes d'homotopie.

\begin{rema}
La classe $[s\mapsto x_0]$ est l'ensemble des lacets de $X$ en $x_0$, homotopes \g{a} z\'ero.
\end{rema}

\begin{exem}
Soit un espace topologique contenant un seul point $x_0$. Alors le groupe fondamental $\pi_1(\{x_0\},x_0)$ est trivial, %
il admet comme seul \'el\'ement la classe d'hmotopie $[s\mapsto x_0]$, elle-m\^eme constitu\'ee du seul lacet constant de $\{x_0\}$.
\end{exem}

Nous allons voir d'autres exemples de groupes fondamentaux; d'une facon g\'en\'erale, un groupe fondamental est difficle \g{a} calculer, surtout si ce groupe est non trivial.
%cédille ?

Avec la notation qui pr\'ec\g{e}de, si $x_0$ est un autre \'el\'ement de $X$ reli\'e, dans $X$, \g{a} $x_0$ par un chemin $\sigma$, %
alors l'application $[\alpha]\mapsto [\sigma^{-1}][\alpha][\sigma]$ r\'ealise un isomorphisme entre les groupes $\pi_1(X,x_0)$ et $\pi_1(X,x_1)$. %
Ainsi, lorsque $X$ est connexe par arcs, tous les groupes fondamentaux d\'efinis sur $X$ sont isomorphes entre eux. %
On parle alors du groupe fondamental de $X$, d\'efini \g{a} isomorphisme pr\g{e}s, ce groupe se note $\pi_1(X)$. %
Toutefois, l'isomorphisme qui pr\'ec\g{e}de n'est pas canonique : il d\'epend de la classe d'homotopie de $\sigma$.

\etoile

On peut montrer que toute application continue $f$ entre deux espaces point\'es $(X,x_0)$ et $(Y,y_0)$ d\'efinit, %
avec la formule $[\alpha]\mapsto [f\circ\alpha]$, un morphisme $f_{\sharp}$ entre les groupes $\pi_1 (X,x_0)$ et $\pi_1 (Y,y_0)$. %
La correspondance entre $f$ et $f_{\sharp}$ respecte la composition des applications et envoie l'application identit\'e d'un espace $(X,x_0)$ sur l'identit\'e de son groupe fondamental. %
On \'etablit notamment la proposition suivante :

\begin{prop}
Soient $(X,x_0)$ et $(Y,y_0)$ deux espaces point\'es tel qu'il existe un hom\'eomorphisme $h$ entre $X$ et $Y$ qui pr\'eserve les points de base. %
Alors : $\pi_1(X,x_0)$ et $\pi_1(Y,y_0)$ sont isomorphes.

Si de plus $X$ et $Y$ sont connexes par arcs, alors les groupes $\pi_1 (X)$ et $\pi_1 (Y)$, d\'efinis \g{a} isomorphisme pr\`es, sont isomorphes.
\end{prop}

\etoile

Voici maintenant un premier exemple g\'en\'eral pour le calcul du groupe fondamental.

\begin{theo}
Soit $(X,x_0)$ un espace topologique point\'e. On suppose que $X$ est contractile : cet espace est donc, en particulier, connexe ar arcs.

Alors : $\pi_1(X,x_0)$ est trivial. De plus, si $x_1$ est un autre \'el\'ement de $X$, alors $\pi_1(X,x_0)$ et $\pi_1(X,x_1)$ sont \textit{canoniquement} isomorphes.

Le groupe $\pi_1(X)$, d\'efini \g{a} isomorphisme pr\g{e}s, est trivial.
\end{theo}

La d\'emonstration de ce th\'eor\g{e}me demande un peu de travail en termes de construction d'homotopies. Voici un r\'esultat interm\'ediaire, qui nous resservira ensuite :

\begin{lemm}\label{abcd}
$\alpha\beta\gamma\delta H$
\end{lemm}

\begin{proof}{\re}
\end{proof}

\begin{proof}[Th\'eor\g{e}me : \tr]
\end{proof}

\begin{defi}[Simple connexit\'e]
Un espace topologique $X$, connexe par arcs, et dont le groupe fondamental $\pi_1(X)$, d\'efini \g{a} isomorphisme pr\g{e}s, est trivial, est dit simplement connexe.
\end{defi}

\begin{exem}
Un espace contractile est simplement connexe.
\end{exem}

\subsubsection{Classes d'homotopie de lacets}

La caract\'erisation qui va suivre pour les classes d'homotopie de lacets est particuli\g{e}remant int\'eressante du point de vue topologique, %
m\^eme si elle oublie la structure alg\'ebrique qui \'emane de la concat\'enation des boucles.

Soit $\mathcal{Q}^{[0,1]}$ l'application : $t\mapsto \exp (2i\pi t)$. %
$\mathcal{P}^{[0,1]}$ est une surjection continue de $[0,1]$ vers $\mathbb{U}$, ferm\'ee car $[0,1]$ est compact. %
Il s'ensuit notamment que $\mathcal{Q}^{[0,1]}$ encoie un ouvert satur\'e de $[0,1]$ sur un ouvert de $\mathbb{U}$.
De plus, cette application identifie $0$ et $1$, et seulement ces deux points. %

\begin{prop}
Soit $(X,x_0)$ un espace topologique point\'e, et $p_0$ un point de la sph\g{e}re $\mathbb{S}^1$ de l'espace euclidien $\mathbb{R}^2$.

Il existe une bijection entre $\pi_1(X,x_0)$ et $[(\mathbb{S}^1,p_0);(X,x_0)]$.
\end{prop}

\begin{proof}
Prenons ici pour $(\mathbb{S}^1,p_0)$ l'espace point\'e $(\mathbb{U},1)$ afin de simplifier les formules de cette d\'emonstration, %
la preuve est exactement la m\^eme pour la sph\g{e}re de $\mathbb{R}^2$ point\'ee en un $p_0$ quelconque.

Soit maintenant $\alpha$ une boucle de $(X,x_0)$. %
Puisque $0$ et $1$ sont envoy\'es en $1$ par l'application $\mathcal{Q}^{[0,1]}$, %
et puisque cette application envoie injectivement $]0,1[$ sur $\mathbb{U}\setminus\{1\}$, %
on peut d\'efinir de mani\g{e}re unique l'application $\tilde{\alpha}$ de $(\mathbb{U},1)$ vers $(X,x_0)$ par la formule :
\[\forall s \in [0,1],\tilde{\alpha}\left(\mathcal{Q}^{[0,1]}(s)\right)=\alpha (s)\]
De plus, cette application est continue car $\mathcal{Q}^{[0,1]}$ envoie un ouvert satur\'e de $[0,1]$ sur un ouvert de $\mathbb{U}$.

R\'eciproquement, on associe, pour toute application continue $\tilde{\alpha}$ entre les espaces point\'es $(\mathbb{U},1)$ et $(X,x_0)$ %
une boucle $\alpha$ d\'efinie par :
\[\forall s\in [0,1],\alpha(s)=\tilde{\alpha}\left(\mathcal{Q}^{[0,1]}(s)\right)\]

Ces deux applications, $\alpha\mapsto\tilde{\alpha}$ et $\tilde{\alpha}\mapsto\alpha$ sont inverses l'une de l'autre, elles sont donc bijectives.

Montrons que, si $\alpha$ et $\alpha '$ sont deux lacets homotopes de $(X,x_0)$,%
alors $\tilde{\alpha}$ et $\tilde{\alpha '}$ sont \'egalement homotopiquements \'equivalemntes du point de vue des applications continues de $(\mathbb{U},1)$ vers $(X,x_0)$. %
Soit en effet $H$ une homotopie entre deux boucles $\alpha$ et $\alpha '$ de $(X,x_0)$. De m\^eme que dans le cas des lacets, les propri\'etes ensemblistes de $\mathcal{Q}^{[0,1]}$ permettent de d\'efinir, sur $\mathbb{U}\times [0,1]$, une application $\tilde{H}$ par :
\[\forall (s,t)\in[0,1]\times [0,1],\tilde{H}\left(\mathcal{Q}^{[0,1]}(s),t\right)=H(s,t)\]
La satur\'ee-ouvertitude de $\mathcal{Q}^{[0,1]}$ entra\^ine celle de $\left(\mathcal{Q}^{[0,1]},Id_{[0,1]}\right)$ -on identifie les $(0,t)$ et de $(1,t)$-, %
il s'ensuit, comme en  ce qui concernait la d\'efinition de $\tilde{\alpha}$, que l'application $\tilde{H}$ est continue sur $\mathbb{U}\times[0,1]$. %
V\'erifions que $\tilde{H}$ est une homotopie, relativement \g{a} $1$, entre les applications $\tilde{\alpha}$ et $\tilde{\alpha '}$.
Remarquons d'abord que, pout tout $t$ r\'eel compris entre $0$ et $1$ :
\[\tilde{H}(1,t)=H(0,t)\text{ et de m\^eme }\tilde{H}(1,t)=H(1,t)\]
Puisque $H$ est une homotopie de lacets, on en d\'eduit que la fonction de $t$ ci-dessus prend une valeur unique, %
autrement dit $\tilde{H}$ est une homotopie relativement \g{a} $\{1\}$.\\
Soit par ailleurs : $z\in\mathbb{U}$. Il existe au moins un \'el\'ement $s$ de $[0,1]$ tel que $\mathcal{Q}^{[0,1]}(s)$ vaut $z$. %
Ainsi : $\tilde{H}(z,1)=H(s,0)$ donc $\tilde{H}(z,0)=\alpha (s)$ par d\'efinition de l'homotopie $H$; %
$\alpha (s)=\tilde{\alpha}\left(\mathcal{Q}^{[0,1]}\right)$ autrement dit $\alpha (s)=\tilde{\alpha}(z)$, on en conclut :
\[\tilde{H}(z,0)=\tilde{\alpha}(z)\]
On d\'emontre de la m\^eme mani\g{e}re que : $\forall z \in\mathbb{U},\tilde{H}(z,1)=\tilde{\alpha '}(z)$.

Enfin, si $\tilde{\alpha}$ et $\tilde{\alpha '}$ sont deux applications continues entre les espaces point\'es $(\mathbb{U},1)$ et $(X,x_0)$, %
alors les lacets $\alpha$ et $\alpha '$, d\'efinis avec la convention de langage qui pr\'ec\g{e}de, sont homotopes. %
Pour s'en convaincre, il suffit de construire une homotopie entre ces deux lacets en composant des applications continues.
\end{proof}

\begin{rema}
Cette proposition semble naturelle du point de vue topologique : %
en effet, $\mathbb{S}^1$ et $[0,1]$ peuvent tous les deux \^etre vus comme des compactifi\'es de l'espace $]0,1[$, %
le premier, par un point \g{a} l'infini, le deuxi\g{e}me par les points $0$ et $1$, %
qui sont justement identifi\'es par la projection qui relie $[0,1]$ \g{a} $\mathbb{S}^1$, %3
ou plut\^ot la bijection qui lui est canoniquement associ\'ee.
\end{rema}

\subsubsection{Un exemple important : $\pi_1(\mathbb{S}^1)$}\label{gs1}

Puisque deux espaces topologiques connexes par arcs, hom\'eomorphes entre eux, ont m\^eme groupe fondamental, d\'efini \g{a} isomorphisme pr\`es, on peut \'enoncer le th\'eor\g{e}me suivant :

\begin{theo}
Le groupe $\pi_1(\mathbb{S}^1)$, d\'efini \g{a} hom\'eomorphisme pr\`es, v\'erifie :

\[\pi_1(\mathbb{S}^1)\cong\mathbb{Z}\]
\end{theo}

\begin{proof}
Voici un r\'esultat interm\'ediaire pour l'\'etude de $[([0,1],\{0,1\}),(\mathbb{U},1)]$ :

\begin{lemm}[Rel\g{e}vement pour une homotopie de chemins]
Soient $\alpha$ et $\beta$ deux lacets de $\mathbb{U}$ bas\'es en $1$. On suppose qu'il existe une homotopie $F$ entre ces deux chemins.

Alors : il existe une application continue $\tilde{F}$ de $[0,1]^2$ dans $\mathbb{R}$ telle que :
\[\forall (s,t)\in[0,1]^2 , \exp(2i\pi \tilde{F}(s,t))=F(s,t)\]
Si l'on demande aussi : $\tilde{F}(0,0)=0$, alors $\tilde{F}$ existe et est unique, %
c'est une homotopie entre les chemins $\tilde{\alpha}$ et $\tilde{\beta}$, respectivement uniques rel\g{e}vements de $\alpha$ et $\beta$ qui valent $0$ en $0$.
\end{lemm}

\begin{proof}
L'existence de l'application $\tilde{F}$ v\'erifiant la formule ci-dessus nous est donn\'ee par la proposition \ref{red2}, l'unicit\'e vient de \ref{urc}.

On remarque que $s\mapsto \tilde{F}(s,0)$ est un rel\`evement de $\alpha$ qui vaut $0$ en $0$ : par unicit\'e dans la d\'efinition de $\tilde{\alpha}$, on en d\'eduit que
\[\forall s\in [0,1],\tilde{F}(s,0)=\tilde{\alpha}(s)\]

Par ailleurs :
\[\forall t\in [0,1],\mathcal{P}(\tilde{F}(0,t))=F(0,t)\]
o\g{u} $\mathcal{P}=a\mapsto\exp (2i\pi a)$.
Puisque $F$ est une homotopie de chemins : $\forall t\in[0,1], F(0,1)=F(0,0)$, %
donc $[0,1]\rightarrow\mathbb{R}:t\mapsto\tilde{F}(0,t)$ parcourt la fibre de $F(0,0)$ pour la projection $\mathcal{P}$. %
On cette fibre est discr\g{e}te : la continuit\'e de $t\mapsto \tilde{F}(0,t)$ nous donne :
\[\forall t\in[0,1],\tilde{F}(0,t)=\tilde{F}(0,0)\text{ soit }\forall t\in[0,1],\tilde{F}(0,t)=0\]
Puisque $\tilde{\beta}(0)=0$, il s'ensuit que : $\tilde{F}(0,1)=\tilde{\beta}(0)$.

Par ailleurs, l'\'egalit\'e : $\forall s\in[0,1],F(s,1)=\beta (s)$ entra\^ine $\forall s\in[0,1], \mathcal{P}(\tilde{F}(s,1))=\beta (s)$ donc d'apr\g{e}s \ref{urc} et l'\'egalit\'e d\'emontr\'ee ci-dessus :
\[\forall s\in[0,1], \tilde{F}(s,1)=\tilde{\beta}(s)\]
Enfin : $t\mapsto\tilde{F}(1,t)$ est une application continue sur $[0,1]$ qui parcourt $\mathcal{P}^{-1}(\alpha (1))$ puisque $F(t,1)$ vaut $\alpha (1)$ pour tout $t$ de $[0,1]$. %
Il s'ensuit que : $\forall t\in[0,1] , \tilde{F}(1,t)=\tilde{F}(1,0)$ puisque cette fibre est discr\`ete, donc :
\[\forall t\in[0,1],\tilde{F}(1,t)=\tilde{\alpha}(1)\]
En particulier : $\tilde{\alpha}(1)=\tilde{\beta}(1)$, ce qui nous permet de conclure : $\tilde{F}$ est une homotopie entre les chemins $\tilde{\alpha}$ et $\tilde{\beta}$, qui prennent leurs valeurs dans $\mathbb{R}$.
\end{proof}

On peut donc associer sans ambig\"uit\'e, \g{a} toute classe d'homotopie $[\alpha]$ des lacets de $\mathbb{U}$ bas\'es en $1$, %
la classe d'homotopie de chemins $[\tilde{\alpha}]$, o\`u $\tilde{\alpha}$ est l'unique rel\`evement de $\alpha$ qui vaut $0$ en $0$. %
Toujours avec le lemme pr\'ec\'edent, on d\'efinit, de mani\g{e}re unique, pour toute classe $[\tilde{\alpha}]$ de chemins homotopes de $\mathbb{R}$, le r\'eel : $\tilde{\alpha}(1)$. %
Enfin, pour tout lacet $\alpha$ de $(\mathbb{U},1)$ : %
$\mathcal{P}^{-1}(\alpha (1))=\mathbb{Z}$ o\g{u} $\mathcal{P}$ d\'esigne toujours l'exponentielle complexe $a\mapsto\exp (2i\pi a)$.

Il existe donc une unique application
\[\deg :\pi_1(\mathbb{U},1)\rightarrow \mathbb{Z}:[\alpha]\mapsto \tilde{\alpha}(1)\]
$\deg ([\alpha]$ sera appel\'e le degr\'e de la classe d'homotpie $[\alpha]$, pour tout lacet $\alpha$ de $(\mathbb{U},1)$.

\begin{lemm}\label{isodeg}
L'application $\deg :\pi_1(\mathbb{U},1)\rightarrow\mathbb{Z}:[\alpha]\mapsto\tilde{\alpha}(1)$ r\'ealise un isomorphisme de groupes.
\end{lemm}

\begin{proof}
On proc\g{e}de en trois \'etapes :
\begin{itemize}
\item[\textit{Propri\'et\'e de morphisme :}]
Soient $\alpha$ et $\beta$ deux lacets de $\mathbb{U}$ bas\'es en $1$. Nous allons calculer : $\tilde{\alpha\cdot\beta} (1)$.

Rappelons que le lacet $\alpha\cdot\beta$ est d\'efini par :
\[
%\left\{
\begin{array}{rcccl}
\forall t\in&\left[0,\frac{1}{2}\right]&\alpha\cdot\beta (t)&=&\alpha (2t)\\[1ex]
\forall t\in&\left[\frac{1}{2},1\right]&\alpha\cdot\beta (t)&=&\beta (2t-1)
\end{array}
%\right.
\]
Soient respectivement, $\tilde{alpha}$ et $\tilde{\beta}$ les rel\g{e}vements de $\alpha$ et $\beta$ qui valent $0$ en $0$. Soit de plus $\tilde{\gamma}$ le chemin d\'efini par :
\[
%\left\{
\begin{array}{rcccl}
\forall t\in&\left[0,\frac{1}{2}\right]&\tilde{\gamma} (t)&=&\tilde{\alpha} (2t)\\[1ex]
\forall t\in&\left[\frac{1}{2},1\right]&\tilde{\gamma} (t)&=&\tilde{\alpha}(1)+\tilde{\beta} (2t-1)
\end{array}
%\right.
\]
Pa recollement, $\tilde{\gamma}$ est continu. On remarque de plus que : $\tilde{\gamma}(1)=\tilde{\alpha}(1)+\tilde{\beta}(1)$.

Montrons maintenant que $\tilde{\gamma}$ est un rel\`evement dans $\mathbb{R}$ de $\alpha\cdot\beta$, qui prend la valeur $0$ en $0$.

La deuxi\g{e}me proposition est \'evidente.

Par d\'efinition de $\tilde{\gamma}$ et $\tilde{\alpha}$ : $\forall t\in\left[0,\frac{1}{2}\right],\exp (2i\pi\tilde{\gamma}(t))=\alpha (2t)$ donc %
\fbox{$\forall t\in\left[0,\frac{1}{2}\right],\exp (2i\pi \tilde{\gamma}(t))=\alpha\cdot\beta (t)$}.

Par ailleurs, puisque $\exp (2i\pi\tilde{\alpha}(1))=1$,

$\tilde{\alpha}(1)\in\mathbb{Z}$ donc : %
\[\forall t\in\left[\frac{1}{2},1\right]\exp (2i\pi(\tilde{\beta}(2t-1)+\tilde{\alpha}(1)))=\exp(2i\pi\tilde{\beta}(2t-1))\]
donc : $\forall t\in\left[\frac{1}{2},1\right],\exp (2i\pi \tilde{\gamma}(t))=\beta (2t-1)$
autrement dit : \fbox{$\forall t\in\left[\frac{1}{2},1\right],\exp (2i\pi\tilde{\gamma}(t))=\alpha\cdot\beta (t)$}

Nous avons montr\'e la propri\'et\'e de rel\`evement pour $\tilde{\gamma}$.

Ainsi : $\tilde{\gamma}=\tilde{\alpha\cdot\beta}$, ce qui entra\^ine : $\tilde{\alpha\cdot\beta} (1)=\tilde{\alpha}(1)+\tilde{\beta}(1)$. Il s'ensuit que :
\[\deg [\alpha\cdot\beta ]=\deg [\alpha ]+\deg [\beta ]\]
\item[\textit{Surjectitiv\'e :}]
Pour tout entier relatif $n$ : $\alpha :[0,1]\rightarrow \mathbb{U}:s\mapsto \exp{2in\pi s}$ admet comme rel\g{e}vement, nul en $0$, $s\mapsto ns$. %
$[\alpha]$ est donc de degr\'e $n$.
\item[\textit{Injectivit\'e :}]
Soit : $[\alpha]\in\ker\deg$.

On peut \'ecrire $\tilde{\alpha}(1)=0$. Ainsi, $\tilde{\alpha}$ est un lacet de $\mathbb{R}$ bas\'e en $0$. Soit $\tilde{H}$ l'homotopie de chemins dans $\mathbb{R}$, d\'efinie par :
\[\forall (s,t)\in [0,1]^2,\tilde{H}(s,t)=(1-t)\tilde{\alpha (s)}+t\cdot 0\]
On note : $\forall (s,t)\in[0,1]^2,H(s,t)=\exp(2i\pi \tilde{H}(s,t))$

Comme $\tilde{\alpha}$ est un rel\g{e}vement de $\alpha$ : $\forall s\in[0,1],H(s,0)=\alpha (s)$. 

De plus :
\[\forall t\in[0,1],\left(\tilde{H}(0,t)=0\right)\wedge\left(\tilde{H}(1,t)=0\right)\]
donc
\[\forall t\in[0,1],\left(H(0,t)=1\right)\wedge\left(H(1,t)=1\right)\]
car $H$ est continue et car car la fibre de l'exponentielle complexe au-dessus de $1$ est discr\`ete.

Enfin : $\forall s\in[0,1],\tilde{H}(s,1)=0$ donc $\forall s\in[0,1],H(s,1)=1$.

Conclusion : $H$ est une homotopie entre $\alpha$ et le chemin constant, de valeur $1$. Il s'ensuit que : $[\alpha]$ est l'\'el\'ement neutre de $\pi_1(\mathbb{U},1)$, ce qui ach\g{e}ve la prueve pour l'innjectivit\'e.
\end{itemize}
\end{proof}

On peut maintenant terminer la preuve du th\'eor\g{e}me. $\mathbb{U}$ est connexe par arcs : on d\'eduit du lemme \ref{isodeg} que le groupe $\pi_1(\mathbb{U}$, d\'efini \g{a} isomorphisme pr\g{e}s, est $\mathbb{Z}$.

Enfin : $\mathbb{S}^1\simeq\mathbb{U}$ \g{a} hom\'eomorphisme pr\g{e}s, il s'ensuit que :\[\pi_1(\mathbb{S}^1)\cong\mathbb{Z}\]
\end{proof}

\subsubsection{Pour finir ...}

On peut aller plus loin et constater que le groupe fondamental est un \textbf{invariant homotopique}, c'est-\g{a}-dire qu'il peut \^etre transport\'e par une \'equivalence homotopique entre deux espaces point\'es.

\begin{prop}
Soient $(X,x_0)$ un espace topologique point\'e, $Y$ un autre espace topologique, on suppose qu'il existe une \'equivalence homotopique $h$ de $X$ vers $Y$.

Alors : $\pi_1(X,x_0)$ est isomorphe \g{a} $\pi_1(Y,h(x_0))$.
\end{prop}

Ici encore, une d\'emonstration est possible avec les connaissances qui pr\'ec\g{e}dent, %elle requiert un peu de virtuosit\'e en mati\g{e}re de construction d'homotopie.
on peut utiliser le lemme \ref{abcd}

A l'aide de ce r\'esultat, on retrouve facilement la simple connexit\'e des espaces contractiles, %
et on montre, par exemple, que l'espace $\mathbb{R}^2\setminus\{(0,0)\}$ admet comme groupe fondamental, d\'efini \g{a} isomorphisme pr\g{e}s, $\mathbb{Z}$. %
C'est aussi le cas de n'importe quelle couronne de $\mathbb{R}^2$.

\subsection{Un th\'eor\`eme de rel\`evement homotopique}

\begin{theo}[Th\'eor\`eme de rel\`evement homotopique]
Soit \Fiy un fibr\'e localement trivial.%, où $Y$ n'est pas nécessairement un groupe de structure.\\
On fixe un entier naturel non nul $n$.\\
On suppose qu'il existe une application continue $f$ de $[0,1]^n$ dans $X$ qui se rel\`eve sur $P$ en une application $\tilde{f}$ de $[0,1]^n$ dans $P$, %
de sorte que le diagrame ci-dessous soit commutatif :
\[\xymatrix{ &P \ar[d]^{\mathcal{P}} \\ [0,1]^n \ar[r]^{f} \ar[ru]^{\tilde{f}} & X}\]
On suppose de plus l'existence d'une homotopie $F$ dans $[0,1]^n$ \`a valeurs dans $X$, telle que :
\[\forall y \in [0,1]^n , F(y,0) = f(y)\]
Alors $F$ admet un rel\`evement $\tilde{F}$ sur $P$, autrement dit une homotopie $\tilde{F}$ d\'efinie sur $[0,1]^n \times [0,1]$ et \`a valeurs dans $P$, telle que $F = \mathcal{P} \circ \tilde{F}$, qui v\'erifie encore :
\[\forall y \in [0,1]^n , \tilde{F}(y,0) = \tilde{f}(y)\]
En particulier le diagrammme suivant est commutatif :
\[\xymatrix{ &P \ar[d]^{\mathcal{P}} \\ [0,1]^n \times [0,1] \ar[r]^{F} \ar[ru]^{\tilde{F}} & X}\]
\end{theo}

\begin{proof}[\tr]
%\dots \null
\end{proof}

\subsection{Groupes d'homotopie sup\'erieurs}




%Exemples : $z\mapsto z^2$ etc ?



\section{Sph\g{e}res euclidiennes}