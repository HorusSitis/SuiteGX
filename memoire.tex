\documentclass[a4paper,10pt]{report}
%\usepackage[utf8]{inputenc}

\usepackage[top=2cm,bottom=2cm,left=2cm,right=2cm]{geometry}% On peut changer en cours de route avec la commande \newgeometry

\usepackage[utf8]{inputenc}
\usepackage[french]{babel}
\usepackage{amsmath,amsfonts,amssymb,graphicx}
%\usepackage{amsthm}
\usepackage{framed}
\usepackage[amsthm,thmmarks,framed]{ntheorem}
\usepackage[dvipsnames]{pstricks}
%\usepackage{pstricks-add,pst-plot,pst-node}
\usepackage{pstricks,pst-plot,pst-text,pst-tree}%,pst-eps,pst-fill,pst-node,pst-math,pstricks-add,pst-xkey}
\usepackage{epic,eepic}
% Versions propos\'ees par Frédéric Junier
\usepackage[inline]{asymptote}


%\usepackage{color} %(black, white, red, green, blue, yellow, magenta et cyan)
\usepackage[dvipsnames]{xcolor}

\usepackage{verbatim}

\usepackage{enumitem}
\frenchbsetup{StandardLists=true}

\usepackage[all]{xy}

\usepackage{multicol}				%%%	Jusqu' \`a nouvel ordre pas de multicol, pour cause de compatibilit\'e avec des fonctions graphiques de TeX.
%\begin{multicols}[titre]{nb colonnes}
%\setlength{\columnseprule}{0.25pt}

\usepackage{parskip}
\setlength{\parindent}{0cm}

%Je veux une mise en page pour les théorèmes lemmes preuves, définitions etc, suffisamment aérée. Est-il possible de commander un encadrement systématique ?

\newcommand{\g}[1]{\`#1}%\g{lettre}
\newcommand{\Rom}[1]{\uppercase\expandafter{\romannumeral #1\relax}}%\uppercase permet d'avoir des lettres majuscules

\providecommand{\abs}[1]{\lvert#1\rvert}

%Pour tout le m\'emoire !
\newcommand{\Fiy}{$(P,X,\mathcal{P},Y)$}
\newcommand{\Fig}{$(P,X,\mathcal{P},G)$}


%Annotations d'un m\'emoire en cours de r\'edaction :
\newcommand{\tr}{\textbf{\textcolor{red}{D\'emonstration \g{a} trouver }}}
\newcommand{\re}{\textbf{\textcolor{NavyBlue}{D\'emonstration \g{a} recopier }}}
\newcommand{\es}{\textbf{\textcolor{OliveGreen}{Esquisse de d\'emonstration }}}

%Pour les formules alg\'ebriques casse-pieds :
\newcommand{\Sk}[1]{\mathcal{S}(\mathbb{#1}^{n+1},\|\|_2)}
\newcommand{\Qr}{\frac{\mathbb{K}^{n+1}\setminus\{0\}}{\mathbb{R}_+^{\ast}}}
\newcommand{\mcd}[4]{\left(\begin{array}{cc}#1&#2\\#3&#4\end{array}\right)}

%Pour les formules concernant les Ufibrés :
\newcommand{\ec}[1]{e^{\mathbf{i}#1}}
\newcommand{\st}[3]{\left(\cos\dfrac{#1}{2}\ec{#2},\sin\dfrac{#1}{2}\ec{#3}\right)}
\newcommand{\se}[2]{s_{U_{#1}}^{#2}}
\newcommand{\lt}[1][k]{\dfrac{\mathbb{S}^3}{\mathbb{U}_{#1}}}
\newcommand{\pl}[3][\omega]{\text{Im}\left(\left(#2\cdot #1 ,#3\cdot #1\right)\right)_{#1\in\mathbb{U}_k}}
\newcommand{\pla}[3][m]{\text{Im}\left(\left(#2\cdot \ec{\frac{2#1\pi}{k}} ,#3\cdot \ec{\frac{2#1\pi}{k}}\right)\right)_{#1\in[\![1,k]\!]}}
\newcommand{\gk}[1][k]{\underline{\gamma}_{#1}}

%Autres formules pour le m\'emoire :
\newcommand{\tc}[1]{\text{#1}}
\DeclareMathOperator{\sinc}{sinc}


\newcommand*{\etoile}
{
\begin{center}
\hspace{1pt}\par
*\hspace{5pt}*\hspace{5pt}*
\end{center}
}


\newcommand*{\ligneinter}
{
\begin{center}
\vspace{2pt}
\hfill\rule{0.5\linewidth}{0.1pt}\hfill\null
\end{center}
\vspace{7pt}
}

{
\theoremstyle{break}
\theoremprework{\vspace{0.2cm}\begin{minipage}{\textwidth}} %Pris en compte uniquement pour le premier newtheorem
\theorempostwork{\ligneinter\end{minipage}} %Même chose
\theoremheaderfont{\scshape}
\theorembodyfont{\itshape}
\theoremseparator{ :\newline\vspace{0.2cm}}
\newtheorem{defi}{D\'efinition}%[section]
%\newtheorem{prop}{Proposition}[section]
%\newtheorem{lemm}{Lemme}[section]
} 

{%
\theoremstyle{break}
\theoremprework{\vspace{0.2cm}\begin{minipage}{\textwidth}} %Pris en compte uniquement pour le premier newtheorem
\theorempostwork{\ligneinter\end{minipage}} %Même chose
\theoremheaderfont{\bfseries}
\theorembodyfont{\itshape}
\theoremseparator{ :\newline\vspace{0.2cm}}
%\newtheorem{lemm}{Lemme}[section]
\newtheorem{prop}{Proposition}[section]
}

{%
\theoremstyle{break}
\theoremprework{\vspace{0.2cm}\begin{minipage}{\textwidth}} %Pris en compte uniquement pour le premier newtheorem
\theorempostwork{\etoile\end{minipage}} %Même chose
\theoremheaderfont{\bfseries}
\theorembodyfont{\itshape}
\theoremseparator{ :\newline\vspace{0.2cm}}
%\newtheorem{lemm}{Lemme}[section]
\newtheorem{prefi}{Proposition-d\'efinition}[section]
}

{%
\theoremstyle{break}
%\theoremprework{\begin{tabular}{|p{\textwidth}|}\hline}
%\theorempostwork{\\ \hline\end{tabular}}
\theoremheaderfont{\scshape}
\theorembodyfont{\normalfont}
\theoremseparator{ :\newline\vspace{0.2cm}}
%\newtheorem{lemm}{Lemme}[section]
\newframedtheorem{theo}{Théor\`eme}[section]
}

{%
\theoremstyle{break}
\theoremprework{\vspace{0.2cm}\begin{minipage}{\textwidth}}
\theorempostwork{\ligneinter\end{minipage}}
\theoremheaderfont{\scshape}
\theorembodyfont{\itshape}
\theoremseparator{ :\newline\vspace{0.2cm}}
\newtheorem{lemm}{Lemme}[theo]
}
  
{%
\theoremstyle{break}
%\theoremprework{\begin{tabular}{|p{\textwidth}|}\hline}
%\theorempostwork{\\ \hline\end{tabular}}
\theoremheaderfont{\scshape}
\theorembodyfont{\itshape}
\theoremseparator{ :\newline\vspace{0.2cm}}
%\newtheorem{lemm}{Lemme}[section]
\newframedtheorem{coro}{Corollaire}[theo]
}



{
\theoremstyle{break}
\theoremprework{\vspace{0.5cm}}
\theorempostwork{\vspace{0.5cm}\ligneinter}
\theoremheaderfont{\scshape}
\theorembodyfont{\normalfont\small}
\theoremseparator{ :\newline\vspace{0.2cm}}
\newtheorem{exem}{Exemple}[section]
} 

{
\theoremstyle{break}
\theoremprework{\vspace{0.5cm}\begin{minipage}{\textwidth}}
\theorempostwork{\end{minipage}\ligneinter}
\theoremheaderfont{\scshape}
\theorembodyfont{\small}
\theoremseparator{ :\newline\vspace{0.2cm}}
\newtheorem{rema}{Remarque}[section]
} 

%\includeonly{gfibres_deb/gfibres_deb}
%\includeonly{spheres_eu/spheres_eu}
%\includeonly{ytopo/ytopo}

%opening
\title{M\'emoire sur les fibr\'es principaux, et leur utilisation en physique}
\author{Antoine Moreau}

\begin{document}

\maketitle

\begin{abstract}
L'objectif de ce travail de m\'emoire est de pr\'esenter les principes math\'ematiques qui sous-tendent les th\'eories de jauge, %
ainsi que quelques exemples issus de la physique.

\par
Les propri\'et\'es g\'eom\'etriques ou topologiques de cretaines structures math\'ematiques rencontr\'ees en chemin, %
comme les espaces lenticulaires de dimension $3$, seront \'etuidi\'ees pour elles-m\^emes.
\end{abstract}

\chapter{Fibr\'es principaux}
%\section{Introduction aux fibr\'es principaux}
\section{Fibr\'es localement triviaux, fibr\'es principaux}
%\chapter{Fibr\'es localement triviaux, fibr\'es principaux}

\subsection{D\'efinitions, premiers exemples}

\begin{defi}
Soient $X$ un espace topologique et $G$ un groupe topologique, éventuellement discret. On suppose que ces deux espaces sont séparés.\\
Un fibré $G-$principal sur $X$, de groupe de structure $G$, est défini par :
\begin{itemize}
\item un espace topologique $P$, appelé espace total, et une action à droite, continue, de $G$ sur $P$,
\item une surjection continue $\mathcal{P}$ de $P$ sur $X$, $G$-invariante autrement dit : $\forall (p,g) \in P \times G , \mathcal{P} (p \ast g) = \mathcal{P} (p)$,
\end{itemize}
tels que tout point $x$ de la base $X$ soit muni d'un voisinage ouvert $V$, associé à un homéomorphisme $\Phi$ de $\mathcal{P}^{-1}(V)$ sur $V \times G$ tel que:
\[
\Phi = (\mathcal{P} , \psi)\text{ et }\forall (p,g) \in \mathcal{P}^{-1} (V) , \Phi(p \ast g) = (\mathcal{P}(p), \psi(p).g)
\]
Un tel couple $(V, \Phi)$ est appelé trivialisation locale du fibré -bouquet de fibres. Nous étudierons explicitement le terme de phase $\psi$ dans les problèmes de physique que nous rencontrerons.\\
On sh\'ematise par $G \overset{\sigma}{\hookrightarrow} P \overset{\mathcal{P}}{\twoheadrightarrow} X$ le fibré ainsi défini, %
que nous noterons formellement $(P,X,\mathcal{P},G)$. Nous noterons souvent $e_G$ le neutre du groupe $G$ dans nos démonstrations.
\end{defi}

\begin{exem}
Pour un entier naurel $k$, la surjection continue $\mathbb{U}\rightarrow\mathbb{U}:z\mapsto z^k$ d\'efinit un fibr\'e principal, %
de groupe de structure $\mathbb{U}_k$, de base $\mathbb{U}$ hom\'eomorphe \g{a} l'espace des phases $\mathbb{U}$.
\par
Ce fibr\'e est d'ailleurs un rev\^etement, qui consiste \g{a} boucler $k$ fois $\mathbb{U}$ au-dessus de lui-m\^eme.
\end{exem}

On peut d\'efinir une version purement topologique de fibr\'e localement trivial :

\begin{defi}
Soient $X$ et $Y$ deux espaces topologiques s\'epar\'es.
\par
Un fibr\'e localement trivial sur $X$, de fibre $Y$, est d\'efini par :
\begin{itemize}
\item un espace topologique $P$, appel\'e espace total,% et une action \`a droite, continue, de $G$ sur $P$,
\item une surjection continue $\mathcal{P}$ de $P$ sur $X$, %$G$-invariante autrement dit : $\forall (p,g) \in P \times G , \mathcal{P} (p \ast g) = \mathcal{P} (p)$,
\end{itemize}
tels que pour tout point $x$ de la base $X$ %
il existe un voisinage ouvert de $x$ et un hom\'eomorphisme $\Phi$ de $V\times Y$ sur $\mathcal{P}^{-1}(V)$ v\'erifiant :
%soit muni d'un voisinage ouvert $V$, associé à un homéomorphisme $\Phi$ de $\mathcal{P}^{-1}(V)$ sur $V \times G$ tel que:
\[\forall (v,y)\in v\times Y , \mathcal{P}(\Phi (v,y))=v\]
Un tel couple $(V, \Phi)$ est appel\'e trivialisation locale du fibr\'e -botte de fibres. %Nous étudierons explicitement le terme de phase $\psi$ dans les problÚmes de physique que nous rencontrerons.\\
On sh\'ematise par $Y \hookrightarrow P \overset{\mathcal{P}}{\twoheadrightarrow} X$ le fibr\'e ainsi d\'efini, %
que nous noterons formellement \Fiy.
\par
On adoptera souvent la notation $\Phi$, $\Psi$ pour l'homorphisme de trivialisation et sa r\'eciproque.
\end{defi}

\begin{exem}[Ruban de M\"obius]
On r\'ealise le ruban de M\"obius $\mathcal{M}$ comme quotient de $\mathbb{R} \times [-1,1]$ par l'action de $\mathbb{Z}$ : %
\[\mu :n,(x,y) \mapsto (a+n,(-1)^ny)\]
La projection orthogonale $\mathcal{P}$ de $\mathbb{R} \times [-1,1]$ sur son axe $\mathbb{R} \times \{0\}$ %commute à cette action, elle %
commute \g{a} $\mu$, c'est-\g{a}-dire
\[n\underset{\mu_{|\mathbb{R}\times\{0\}}}{\cdot}\mathcal{P}(x,\lambda)=%
\mathcal{P}\left(n\underset{\mu}{\cdot}(x,\lambda\right)\text{ pour tous $n$, $x$ et $\lambda$}\]
Elle d\'efinit donc une projection $\mathcal{P}_{\mathcal{C}}$ de $\mathcal{M}$ sur son cercle m\'ediateur $\mathcal{C}$ %
-image de $\mathbb{R}\times\{0\}$ par $[-1,1]\times\mathbb{R}\xrightarrow{:\mu}\mathcal{M}$, %
isomorphe au quotient du groupe commutatif $(\mathbb{R},+)$ par son sous-groupe $\mathbb{Z}$.
%\par
%Par ailleurs, on montre par double inclusion que : $\mathcal{P}_{\mathcal{C}}^{-1}\left(\{c\}\right)=\text{Pr}\left((x+\mathbb{Z})\times\{0\}\right)$ %
%pour tout \'el\'ement $c$ de $\mathcal{C}$ dont on note $(x+\mathbb{Z}\times\{0\}$ l'image r\'eciproque par $\text{Pr}$.
%\par
%En cons\'equence, avec les m\^emes conventions :
%\[\mathcal{P}_{\mathcal{C}}^{-1}\left(\{c\}\right)=\text{Pr}\left((x+\mathbb{Z})\times[-1,1]\right)\]
%L'\'etude de $\text{Pr}$ sur un domaine fondamental de la forme $\{x\times [-1,1]$, compte tenu de la compacit\'e de $[-1,1]$, %
%permet de conclure que $\mathcal{P}_{\mathcal{C}}^{-1}\left(\{c\}\right)$ est hom\'eomorphe \g{a} $[-1,1]$.
%\par
%On a donc construit un fibr\'e topologique $[-1,1]\hookrightarrow\mathcal{M}\xrightarrow{\mathcal{P}_{\mathcal{C}}}\mathcal{C}$.
\par
\re pour un syst\g{e}me de trivialisations locales.
\end{exem}

Ce fibr\'e permet de construire des fibr\'es principaux de groupe de structure $\{-1,1\},\times$ \tr

Voici une propri\'et\'e basique sur les fibr\'es topologiques, de d\'emonstration \'evidente :

\begin{prop}
Soit \Fiy un fibr\'e localement trivial. Alors :
\begin{itemize}
\item $\mathcal{P}:P\rightarrow X$ est une surjection continue ouverte.
\item La topologie de $X$ est exactement la topologie quotient d\'efinie par la projection $\mathcal{P}$.
\end{itemize}
\end{prop}

Les propri\'et\'es toplogiques des fibr\'es localement triviaux diff\g{e}rent, nous le verrons, l\'eg\g{e}rement de celles des fibr\'es principaux.

%Exemples pour des fibr\'es principaux : $z\mapsto z^2$ etc ? On \'evoque une g\'en\'eralisation avec le projectif r\'eel.

\subsubsection{Sections locales de fibr\'es}

\begin{prop}\label{stV1}
Soit \Fig un fibr\'e principal. Soit de plus $U$ un ouvert de $X$ et soit $s_U$ une section continue de $\mathcal{P}$ d\'efinie sur $U$.
\par
Alors :
\[\Phi:V\times G\rightarrow\mathcal{P}^{-1}(V):x,g\mapsto s_V(x)\cdot g\]
est un hom\'eomorphisme.
\par$U$ est donc un ouvert de trivialisation de \Fig au-dessus duquel on peut d\'efinir une fonction de phase $\psi_U$, continue et $G-$\'equivariante \`a droite.
\end{prop}

\begin{proof}
$\Phi$ est clairement continue, surjective comme les fibres de $\mathcal{P}$ sont exactement les $G$-orbites de $P$. %
La propriété de section pour $s_V$ et la simplicité des $G$-orbites dans $P$ entraînent l'injectivité de $\Phi$.
\par
Montrons maintenant que la r\'eciproque $\Psi$ de $\Phi$ est continue.
\par
Comme $s_V$ est une section locale de $\mathcal{P}$, $\Psi$ s'écrit aussi $(\mathcal{P},\psi)$ %
o\`u $\psi$ est une application de $\mathcal{P}^{-1}(V)$ dans $G$ qui est, comme $\Phi$, $G$-équivariante à droite.
\par
On utilise maintenant toute la puissance de la structure de fibr\'e principal pour $(P,X,\mathcal{P},G)$ et de l'existence de $s_V$ :
Soit $\mathcal{U}$ un ouvert saturé de $\mathcal{P}^{-1}(V)$ qui se projette sur $V$ en un ouvert de trivialisation $U$ %
-il suffit en effet d'établir la continuité de $\psi$ sur de tels ouverts d'après la propriété de recouvrement.
\par
On note encore $(\mathcal{P},\psi_1)$ la trivialisation associée ici à $\mathcal{U}$. %
Comme $\mathcal{U}$ est $G$-saturé, le représentant canonique de chaque élément $p$ de cet ouvert défini à l'aide de $s_V$, %
autrement dit $s_V(\mathcal{P}(p)$, est dans $\mathcal{U}$. %
L'écriture $s_V(\mathcal{P}(p)) \psi(p)$ de $p$ qui définit $\Phi$ en ce point permet d'établir, %
en appliquant la fonction $G$-équivariante à droite $\psi_1$ définie sur $\mathcal{U}$ :
\[\psi_1(p) = \psi_1(s_V(\mathcal{P}(p)))\psi(p)\]
ce qui s'écrit encore :
\[\psi(p) = (\psi_1(s_V(\mathcal{P}(p))))^{-1} \psi_1(p)\]
La continuité de $\psi_1$ sur $\mathcal{U}$, et celle de la section $s_V$, permettent de conclure que $\psi$ est continue sur $\mathcal{U}$, ce qui ach\g{e}ve la démonstration.
\end{proof}

Ainsi, les trivialisations d'un fibr\'e principal $(P,X,\mathcal{P},G)$ de groupe structural $G$ sont en bijection avec les sections locales de $\mathcal{P}$.
%Peut-on d\'emontrer un tel r\'esultat pour un G-fibr\'e, que l'on ne suppose pas, au pr\'ealable, muni d'une famille de trivialisations locales ?
\par
Par exemple, une section $s$ de la projection $\mathcal{P}$ de $P$ sur $X$, autrement dit une section globale, %
nous donne alors une trivialisation globale pour $(P,X,\mathcal{P},G)$.

\etoile
Cette propri\'et\'e n'est pas vraie pour tous les fibr\'es topologiques localement triviaux, comme le montre l'exemple ci-dessous :

\begin{exem}[Retour sur le ruban de M\"obius]
L'inclusion $s_\mathcal{C}$ ne permet pas de d\'efinir une trivialisation globale de $\mathcal{M}$ : %
en effet, dans le cas contraire, $\mathcal{C}$ serait hom\'eomorphe au cylindre $\mathbb{S}^1\times [-1,1]$.
\end{exem} %Peut-on en d\'eduire que le segment $[-1,1]$ ne peut \^etre muni d'une structure de groupe compatible avec sa topologie ?

\begin{rema}[Question ouverte]
Peut-on g\'en\'eraliser la proposition\ref{stV1} au cas o\g{u} il n'existe pas \emph{a priori} de syst\g{e}me de trivialisations locales pour la projection %
$G\hookrightarrow P\overset{\mathcal{P}}{\twoheadrightarrow}X$ ?
\par
Quoi qu'il en soit, toutes les constructions qui vont suivre utiliseront une structure sous-jacente, pour l'espace de phases ou le groupe de structure, %
plus riches que celles de $G-$espace ou de groupe topologique.
\end{rema}

%Voici un exemple \g{u} une structure de groupe topologique, pour l'espace de phases, permet de construire syst\'ematiquement une trivialisation \g{a} partir d'une section locale :






%Exemple avec $\{-1,1\}$, constructions alg\'ebriques : papier Actions de groupes.

\subsection{Exemples importants}

\subsubsection{Espaces projectifs}

Ici, $\mathbb{K}$ d\'esigne l'un des trois corps r\'eels $\mathbb{R}$, $\mathbb{C}$ et $\mathbb{H}$.

\medskip
$\mathbb{K}$ est muni d'une valeur absolue, ou module, $|\ |$, multiplicative, qui prend ses valeurs dans $\mathbb{R}_+$ et s'annulle seulement en $0$. %
Ainsi, $|\ |$ induit un morphisme surjectif entre les groupes multiplicatifs $\mathbb{K}^{\ast}$ et $\mathbb{R}_+^{\ast}$, qui plus est une r\'etraction.

Soit $G$ le noyau de $|\ |$. Pour tout scalaire $\lambda$ non nul, le nombre $\frac{\lambda}{|\lambda|}$, %
bien d\'efini car $|\lambda |$ commute avec tout \'el\'ement de $\mathbb{K}$ pour la multiplication, %
est de module $1$, ce qui nous donne une d\'ecomposition polaire, commutative, pour $\lambda$.\\
Par ailleurs, la projection $\lambda\mapsto\frac{\lambda}{|\lambda|}$ de $\mathbb{K}^{\ast}$ sur $G$ se factorise canoniquement \`a droite %
par un isomorphisme de $\frac{\mathbb{K}^{\ast}}{\mathbb{R}_+^{\ast}}$ sur $G$. %
L'inverse de cet isomorphisme est une section de la projection canonique de $\mathbb{K}^{\ast}$ sur $\frac{\mathbb{K}^{\ast}}{\mathbb{R}_+^{\ast}}$, que nous noterons $\sigma$ dans le suite de ce paragraphe.

\etoile

Soit maintenant $n$ un entier naturel non nul. $\mathbb{K}^{n+1}$ est muni de sa structure de $\mathbb{K}$-espace vectoriel, \`a \textbf{droite} si $\mathbb{K}$ est le corps des quaternions.\\
On sait que $\mathbb{K}^{\ast}$ agit simplement sur l'ensemble $\mathbb{K}^{n+1}\setminus\{0\}$ par multiplication scalaire, %
cette action est d'ailleurs continue, de $\mathbb{K}^{\ast}\times\mathbb{K}^{n+1}\setminus\{0\}$ dans $\mathbb{K}^{n+1}\setminus\{0\}$. %
Le quotient de $\mathbb{K}^{n+1}$ par cette action est appel\'e espace projectif de dimension $n$ sur $\mathbb{K}$, %
et not\'e $\mathbb{P}^n(\mathbb{K})$. On notera $[\quad]$ la projection canonique de $\mathbb{K}^{n+1}\setminus\{0\}$ sur $\mathbb{P}^n(\mathbb{K})$.
%Voici deux mani\`eres diff\'erentes de factoriser $\mathbb{K}^{n+1}\setminus\{0\}\overset{v\mapsto [v]}{\longrightarrow} \mathbb{P}^n(\mathbb{K})$ :

Puisque deux vecteurs non nuls $v$ et $v'$ de $(\mathbb{K}^{n+1},\|\|_2)$ tels que $\frac{v}{\|v\|_2}=\frac{v'}{\|v'\|_2}$ sont colin\'eaires, %
on peut d\'ecomposer la projection canonique de $\mathbb{K}^{n+1}\setminus\{0\}$ comme ci-dessous :
\[
\xymatrix{% & \dfrac{\mathbb{K}^{n+1}\setminus\{0\}}{\mathbb{R}_+^{\ast}} \ar[rd] \\%
\mathbb{K}^{n+1}\setminus\{0\} \ar[rd]_{v\mapsto\frac{v}{\|v\|_2}} \ar[rr]^{v\mapsto [v]} & & \mathbb{P}^n(\mathbb{K})\\%
 & \Sk{K} \ar[ru]
}
\]
Soit $\tilde{\beta}_n$ l'action de $\mathbb{K}^{\ast}$ sur $\mathcal{S}(\mathbb{K}^{n+1},\|\|_2)$ transport\'ee par la multiplication scalaire :
\[\tilde{\beta}_n(\lambda ,v)=\frac{\lambda}{\abs{\lambda}}v\]
pour tout vecteur unitaire $v$ de $\mathbb{K}^{n+1}$ et tout scalaire non nul $\lambda$.\\
Ici encore, $\mathbb{P}^n(\mathbb{K})$ est obtenu en quotientant $\Sk{K}$ par l'action $\tilde{\beta}_n$.

\medskip
Soit maintenant $\beta_n$ l'action de $G$ sur $\Sk{K}$ induite par $\tilde{\beta}_n$. %
On remarque que cette action est aussi une restriction de la multiplication scalaire.\\
Contrairement \`a $\tilde{\beta}_n$, pour laquelle chaque point de $\Sk{K}$ a comme stabilisateur $\mathbb{R}_+^{\ast}$, %
cette action est \textbf{simple}. Les deux actions ont aussi les m\^emes orbites, d'apr\`es la formule explicite pour $\tilde{\beta}_n$ donn\'ee pr\'ec\'edemment.

\bigskip
\textit{\textbf{On peut donc voir $\mathbb{P}^n(\mathbb{K})$ comme un quotient de $\Sk{K}$ par $G$.}}

%\newpage
\par
Voici maintenant quelques pr\'ecisions concernant chacun des cas r\'eel, complexe et quaternionique :

\renewcommand{\arraystretch}{1.6}
\begin{center}
\begin{tabular}{|c||c c|c|}
\hline
$\mathbb{K}$ & $G$ & hom\'eomorphe \`a & $\Sk{K}$ \\
\hline
$\mathbb{R}$ & $\{-1,1\}$ & $\mathbb{S}^0$ & $\mathbb{S}^n$ \\
\hline
$\mathbb{C}$ & $\mathbb{U}$ & $\mathbb{S}^1$ & $\mathbb{S}^{2n+1}$ \\
\hline
$\mathbb{H}$ & $SU(2)$ & $\mathbb{S}^3$ & $\mathbb{S}^{4n+3}$ \\
\hline
\end{tabular}
\end{center}
\renewcommand{\arraystretch}{1}

Ainsi, $G$ est toujours un groupe de Lie. Dans le cas r\'eel, et dans ce cas seulement, il est discret; dans le cas quaternionique, et dans ce cas seulement, $G$ est non commutatif.
\par
Par ailleurs, on montre, avec des projections st\'er\'eographiques, que les droites projectives $\mathbb{P}^1(\mathbb{R})$, $\mathbb{P}^1(\mathbb{C})$ et $\mathbb{P}^1(\mathbb{H})$ %
sont hom\'eomorphes \`a $\mathbb{S}^1$, $\mathbb{S}^2$ et $\mathbb{S}^4$ respectivement.
\par
La projection de $\mathbb{S}^n$ sur $\mathbb{P}^n(\mathbb{R})$ induite par $[\quad]$ permet d'ailleurs de munir $\mathbb{P}^n(\mathbb{R})$ de sa structure diff\'erentielle.

\begin{rema}[Cas g\'en\'eral pour le projectif, vision purement alg\'ebriste]
Les fibres de la projection $v\mapsto\frac{v}{\|v\|_2}$ sont pr\'ecis\'ement %
les $\mathbb{R}_+^{\ast}$-orbites de $\mathbb{K}^{n+1}\setminus\{0\}$ par multiplication scalaire :
\[\xymatrix{&\Qr \ar[dd]^{b} \\
\mathbb{K}^{n+1}\setminus\{0\} \ar[ru]^{v\mapsto (v)} \ar[rd]_{v\mapsto\frac{v}{\|v|_2}} \\
& \Sk{K}
}\]
Il existe une bijection $b$, canoniquement associ\'ee \`a $v\mapsto\frac{v}{\|v\|_2}$, %
entre $\Qr$ et $\Sk{K}$, %
qui est $\mathbb{K}^{\ast}$-\'equivariante \`a droite si l'on consid\`ere les actions $\tilde{\alpha}_n$ et $\tilde{\beta}_n$, %
o\`u $\tilde{\alpha}_n$ est l'action de $\mathbb{K}^{\ast}$ sur $\Qr$ transport\'ee par la multiplication scalaire.

\medskip
Puisque $\mathbb{R}_+^{\ast}$ est un sous-groupe multiplicatif de $\mathbb{K}^{\ast}$, on peut factoriser la projection $[\quad]$ d'une autre mani\`ere :
\[
\xymatrix{ & \Qr \ar[rd]^{: \underline{\alpha}_n} \\%
\mathbb{K}^{n+1}\setminus\{0\} \ar[ru]^{: \mathbb{R}_+^{\ast}} & & \mathbb{P}^n(\mathbb{K})
}
\]
o\`u $\underline{\alpha}_n (\underline{\lambda},(v))=\tilde{\alpha}_n (\lambda ,(v))$ %
pour tout vecteur $v$, non nul, de $\mathbb{K}^{n+1}$, et tout scalaire non nul $\lambda$ de $\mathbb{K}$.
\par
Ainsi, l'espace projectif $\mathbb{P}^n(\mathbb{K})$ se r\'ealise comme un quotient de $\Qr$ %
par $\frac{\mathbb{K}^{\ast}}{\mathbb{R}_+^{\ast}}$.

\medskip
Enfin, on constate que le $\frac{\mathbb{K}^{\ast}}{\mathbb{R}_+^{\ast}},\underline{\alpha}_n$-ensemble $\Qr$ %
et le $G$-ensemble $(\Sk{K},\beta_n)$ %
sont isomorphes, via~la~section~$\sigma$ et~la~bijection~$b$.
%La sous-section qui suit porte sur une d\'ecomposition du m\^eme type avec une action-quotient.
\par
Une factorisation du m\^eme type, avec $\mathbb{U}_k$, $\mathbb{U}$ et $\gk$ en places de $\mathbb{R}_+^{\ast}$, $\mathbb{K}^{\ast}$ et $\underline{\alpha}_n$%
et $\mathbb{S}^3\xrightarrow{[\; ]_k}\lt$ au lieu de la projection canonique de $\mathbb{K}^{n+1}\setminus\{0\}$ sur $\mathbb{P}^n(\mathbb{K})$, sera \'etudi\'ee \g{a} la sous-section \ref{lt1}
\end{rema}

\begin{exem}[Projectif r\'eel, une famille de fibr\'es principaux]
Soit $n$ un entier naturel non nul. Ici, $[\quad]$ d\'esigne la projection de $\mathbb{S}^n$ sur $\mathbb{P}^n(\mathbb{R})$ que nous avons vue plus haut, %
qui revient \`a quotienter par l'action $\beta_n$. %
Dans ce cas, $\beta_n$ appelée \textit{action d'antipodie} du groupe multiplicatif sur la sph\`ere $\mathbb{S}^n$, nous la noterons $\mathcal{A}$.

\medskip
Afin de d\'efinir une famille de trivialisations locales pour %
$\{-1,1\} \overset{\mathcal{A}}{\hookrightarrow} \mathbb{S}^n \overset{[\quad ]}{\twoheadrightarrow} \mathbb{P}^n(\mathbb{R})$ %
on considère la famille $(V_i)_{i \in [\![[1,n]\!]}$ de parties de l'espace projectif réel de dimension $n$, définie pour tout indice $i$ par :
\[V_i = [U_i] \text{, o\g{u}} U_i = \{(x_j)_j \in \mathbb{S}^n | x_i \neq 0\}\]
On remarque que les termes de $(U_i)$ sont des ouverts satur\'es de $\mathbb{S}^n$. %
On en d\'eduit que ces ouverts sont exactement les images r\'eciproques par $[\quad]$ des termes de $(V_i)$, %
puis que $(V_i)$ est une famille d'ouverts de $\mathbb{P}^n(\mathbb{R})$.

\par
Par ailleurs, tout \'el\'ement de la sph\`ere $\mathbb{S}^n$ admet au moins une coordonn\'ee non nulle : %
il s'ensuit que $(V_i)$ est un recouvrement ouvert de $\mathbb{P}^n(\mathbb{R})$.

\medskip
Soit maintenant : $i \in [\![1,n]\!]$. On peut \'ecrire l'union disjointe :
\[U_i = U_i^+ \cup U_i^-\]
o\g{u} l'on note $U_i^+ = \{ (x_j)_j \in \mathbb{S}^n | x_i > 0 \}$ et sym\'etriquement : $U_i^- = \{ (x_j)_j \in \mathbb{S}^n | x_i < 0 \}$.

\par
Ainsi : $U_i^+(-1) = U_i^-$, et vice-versa, autrement dit : $\forall u^+ \in U_i^+ , -u^+ \in U_i^-$ et sym\'etriquement $\forall u_- \in U_i^- , -u^- \in U_i^+$.

\par
L'antipodie est une action libre donc les orbites de $\mathbb{S}^n$ sous cette action admettent toutes exactement deux \'el\'ements. %
On peut donc écrire : $\forall v \in V_i , \exists ! s_i^+(v) \in U_i^+ | [s_i^+(v)] = v$; $s_i^+$ est une section de la projection $[\quad]$. %
On remarque que $[\quad]$ induit une surjection continue de $U_i^+$ vers $V_i$, injective d'après les deux formules en $u_+$ , $u_-$ qui pr\'ec\g{e}dent.

\par
Par ailleurs, l'action d'antipodie est continue, donc la projection correspondante, $[\quad]$, est ouverte. %
Il s'ensuit que l'application induite par $[\quad]$ de $U_i$ vers $V_i$ est un homéomorphisme, sa réciproque est $s_i^+$.

\par
On construit maintenant la fonction de phase $\psi_i$ pour notre trivialisation locale, avec les formules :
\[\psi_i(U_i^+) = \{1\}\text{ qui donne la construction de $s_i^+$, et }\psi_i(U_i^-) = \{ -1 \}\]
\[\text{On note alors : }\Psi_i = U_i \rightarrow V_i \times \{ -1 , 1 \} : u \mapsto ([u],\psi_i(u))\]

Montrons la propriété d'équivariance pour $\psi_i$ :
\[\forall u^+ \in U_i^+ , u^+ (-1) \in U_i^-\text{ donc }\forall u^+ \in U_i^+ , \psi_i (u^+ (-1)) = \psi(u^+)(-1)\]
et symétriquement que cette formule est vraie sur $U_i^-$.

\par
On pourra en d\'eduire que $\Psi_i$ est surjectif, de plus la simplicité de l'action $\mathcal{A}$ nous donne l'injectivité pour ce morphisme.

\par
Ecrivons maintenant une formule explicite pour $\psi_i$  : soit $(x_j)_j$ un \'el\'ement de $U_i$. $(x_j)_j$ est élément de $U_i^+$ si et seulement si $x_i$ est strictement positif, %
autrement dit lorsque : $\mathbf{\frac{x_i}{\abs{x_i}} = 1}$

\par
De m\^eme : $(x_j)_j \in U_i^- \Leftrightarrow \frac{x_i}{\abs{x_i}} = -1$. %
On écrit alors : $\psi_i = (x_j)_j \mapsto \frac{x_i}{\abs{x_i}}$.

\par
Cette formule nous donne la continuit\'e pour $\psi_i$, le morphisme $\Psi_i$, dont les deux fonctions composantes sont continues, est donc continu.

\par
Enfin, l'application à variables s\'epar\'ees $V_i \times \{-1 , 1\} \rightarrow U_i : (v,\varepsilon) \mapsto s_i^+(v).\varepsilon$, %
bijection réciproque de $\Psi_i$, est continue.

\par
%\medskip
$\Psi_i$ est donc un hom\'eomorphisme, $(V_i,\Psi_i)$ est une trivialisation locale pour
\[\{-1,1\} \overset{\mathcal{A}}{\hookrightarrow} \mathbb{S}^n \overset{[\quad]}{\twoheadrightarrow} \mathbb{P}^n(\mathbb{R})\]
%
%\bigskip
\emph{En conclusion} $(V_i)_i$ est un recouvrement ouvert de $\mathbb{P}^n(\mathbb{R})$, nous avons muni cet espace d'une structure de $\{-1,1\}-$fibr\'e principal.
\end{exem}

Un cas remarquable de fibr\'e d\'efini ci-dessus est celui de la dimension $1$, \'evoqu\'e dans le tableau pr\'ec\'edent. %
Ce fibr\'e $\{-1,1\} \overset{\mathcal{A}}{\hookrightarrow} \mathbb{S}^1 \overset{[\quad]}{\twoheadrightarrow} \mathbb{P}^1(\mathbb{R})$ %
se construit alg\'ebriquement via la projection $z\mapsto z^2$ de $\mathbb{U}$ sur lui-m\^eme :

\[\xymatrix{%
\mathbb{U} \ar[r]^{z\mapsto z^2} \ar[d]_{:\mathbb{U}_2}&\mathbb{U}\\%
\dfrac{\mathbb{U}}{\mathbb{U}_2} \ar[ru]_{h_2}&%
}\]
La projection \g{a} gauche du diagramme, symbolis\'ee par une fl\g{e}che dirig\'ee vers le bas, revient \g{a} quotienter $\mathbb{U}$ par \emph{antipodie}. %
L'isomorphisme $h_2$ de groupes topologiques, que nous recontrerons au moment de construire de nouveaux fibr\'es principaux de base $\mathbb{S}^2$ \g{a} la sous-section\ref{lt1}, %
permet d'identifier la projection $z\mapsto z^2$ de $\mathbb{U}$ sur lui-m\^eme \g{a} l projection $[\quad ]$ construite ci-dessus dans le cas de la dimension $1$.

\par
$\{-1,1\} \overset{\mathcal{A}}{\hookrightarrow} \mathbb{S}^1 \overset{[\quad]}{\twoheadrightarrow} \mathbb{P}^1(\mathbb{R})$ est donc isomorphe, %
en un sens que nous verrons plus tard, \g{a} $\{-1,1\}\hookrightarrow\mathbb{U}\xrightarrow{z\mapsto z^2}~\mathbb{U}$ que nos avons \'evoqu\'e au d\'ebut de ce m\'emoire.

\par
\emph{En particulier, $\mathbb{P}^1(\mathbb{R})$ est hom\'eomorphe \g{a} $\mathbb{S}^1$.}

\subsubsection{Au-dessus de la sph\g{e}re $\mathbb{S}^2$ : fibrations de Hopf}

La projection canonique $[\;]$ de $\mathbb{C}^2\setminus\{(0,0)\}$ sur $\mathbb{P}^1(\mathbb{C})$ va nous permettre de construire %$\mathbb{C}^2\setminus\{(0,0)\}\overset{[ \; ]}{\longrightarrow}\mathbb{P}^1(\mathbb{C})$
\textbf{deux} fibr\'es principaux, de groupe de structure $\mathbb{U}$, au-dessus de $\mathbb{S}^2$ ; elles sont souvent appel\'ees \emph{fibrations de Hopf}.

\par
La premi\`ere partie, facile, du travail consiste \g{a} restreindre la projection $[\; ]$ \`a la sph\`ere de l'espace euclidien $\mathbb{C}^2$ : $\mathbb{S}^3$. %
On notera encore $[\; ]$ cette nouvelle projection ; par ailleurs la multiplication scalaire de $\mathbb{C}^2$ induit une action, simple, de $\mathbb{U}$ sur $\mathbb{S}^3$ : %
ainsi, $[\;]$ r\'ealise une projection de $\mathbb{S}^3$ sur $\mathbb{P}^1(\mathbb{C})$, %
dont les fibres sont exactement les orbites de $\mathbb{S}^3$ sous cette action de $\mathbb{U}$.

\par
Il nous reste \g{a} composer \g{a} gauche cette projection par des bijections bien choisies de $\mathbb{P}^1(\mathbb{C})$ sur $\mathbb{S}^2$.

\par
Remarquons tout d'abord que tout \'el\'ement de la sph\g{e}re-unit\'e de $\mathbb{C}^2$, muni simultan\'ement des structures d'espace hermitien de dimension $2$ et d'espace euclidien de dimension $4$, %
s'\'ecrit : $\left(r_1\ec{\xi_1},r_2\ec{\xi_2}\right)$, %
o\`u $r_1$ et $r_2$ sont des r\'eels positifs d\'efinis de mani\`ere unique et $\xi_1$ et $\xi_2$ deux r\'eels d\'efinis \`a un multiple de $2\pi$, v\'erifiant encore :
\[r_1^2+r_2^2=1\]
Avec cette notation, il existe un unique r\'eel $\phi$ compris entre $0$ et $\frac{\pi}{2}$ tel que : $(r_1,r_2)=(\cos\phi , \sin\phi)$. %
Afin de faciliter les calculs qui vont suivre, on \'ecrit plut\^ot :

\par
\emph{Tout \'el\'ement de la sph\g{e}re-unit\'e $\mathbb{S}^3$ de $\mathbb{C}^2$ s'\'ecrit de mani\g{e}re unique $\left(\cos\dfrac{\phi}{2}\ec{\xi_1},\sin\dfrac{\phi}{2}\ec{\xi_2}\right)$ ;} %
o\`u $\phi$ est un nombre r\'eel compris entre $0$ et $\pi$, $\xi_1$ et $\xi_2$ deux r\'eels d\'efinis \g{a} un multiple de $2\pi$ pr\g{e}s.

\par
Bien s\^ur, chaque triplet $(\phi,\xi_1,\xi_2)$ d\'efinit, de mani\`ere unique, un \'el\'ement de $\mathbb{S}^3$.

\ligneinter
Soit, en premier lieu, $\mathcal{U}_S$ l'ensemble des \'el\'ements de $\mathbb{S}^3$ dont la deuxi\g{e}me coordonn\'ee est non nulle, %
autrement dit $\mathbb{S}^3\setminus\mathbb{U}\times\{0\}$.

\par
Chaque \'el\'ement $\st{\phi}{\xi_1}{\xi_2}$ de $\mathcal{U}_S$, qui v\'erifie au passage : %
\boldmath$\phi\neq 0$\unboldmath, est envoy\'e sur $\cot\dfrac{\phi}{2}\ec{\xi_1-\xi_2}$ via la projection $[\;]$ %
et carte $\mathbb{P}^1(\mathbb{C})\setminus\{[1,0]\}\xrightarrow{[z_1,z_2] \mapsto \frac{z_1}{z_2}}\mathbb{C}\times\{1\}$ de la droite projective $\mathbb{P}^1(\mathbb{C})$.
%\overset{[z_1,z_2]\mapsto \left(\frac{z_1}{z_2},1\right)}{\longrightarrow}

\par
Nous pouvons maintenant r\'ealiser une projection $\mathcal{P}_S$ de l'espace $\mathcal{U}_S$ sur la sph\`ere-unit\'e $\mathbb{S}^2$ de l'espace euclidien $\mathbb{R}^3$, priv\'ee de son p\^ole nord $(0,0,1)$. %
On utilise pour cela l'inverse de la projection st\'er\'eographique $\varphi_S$, qui envoie $\mathbb{S}^2\setminus\{(0,0,1)\}$ sur le plan vectoriel $\mathbb{R}^2\times\{0\}$ %
-la calotte sur laquelle cette projection est d\'efinie sera not\'ee $U_S$.
%
%%%% --------------------- Attention : les %% sont inopérants dans un code asymptote --------------------- %%%%
%
%draw(Q--R);
%\ligneinter

\etoile
%$\st{\lambda^c}{\theta +\xi}{\xi}$
$\varphi_S^{-1}$ envoie le point $\left(\cot\dfrac{\phi}{2} \cos \left(\xi_1-\xi_2\right) , \cot\dfrac{\phi}{2} \sin\left(\xi_1-\xi_2\right) , 0\right)$ %
du plan $\mathbb{R}^2\times\{0\}$ auquel est identifi\'e $\mathbb{C}$ sur le point $(p_1^S,p_2^S,p_3^S)$ sch\'ematis\'e ci-dessus %
de la sph\`ere $\mathbb{S}^2$ priv\'ee de son p\^ole Nord.

\par
Si l'on rep\`ere les points de $\mathbb{S}^2$ par un couple de r\'eels : \og{}colatitude ; longitude\fg{} , on constate que la longitude de $P$ est $\xi_1-\xi_2$.%, nous la noterons $\theta$.
%
%Figure avec la section de $\mathbb{S}^2$ par le demi-plan rep\'er\'e par la coordonn\'ee cylindrique $\theta$, pour appliquer le th\'eor\`eme de Thal\`es.
%

\ligneinter
\begin{multicols}{2}
\psset{xunit=1cm , yunit=1cm}
\begin{pspicture*}(-0.2,-3.2)(7.5,4)
\def\xmin{0} \def\xmax{7} \def\ymin{-3} \def\ymax{3.5}
\psframe[linewidth=0.3pt,linecolor=gray](-0.2,-3.2)(7.5,3.5)
\def\pshlabel#1{\psframebox*[framesep=1pt]{\small #1}}
\def\psvlabel#1{\psframebox*[framesep=1pt]{\small #1}}
\psclip{%
\psframe[linestyle=none](\xmin,\ymin)(\xmax,\ymax)
}
\psset{linecolor=black, linewidth=.5pt, arrowsize=2pt 4}
\pscircle(0.0000,0.0000){3.0000}
\psdots[dotstyle=*, dotscale=1.0000](6.0000,0.0000)
\uput{0.3000}[90.0000](6.0000,0.0000){C:$\ \cot\left(\frac{\phi}{2}\right)$}
\psline(8.0000,-1.0000)(-2.0000,4.0000)
\psdots[dotstyle=*, dotscale=1.0000](2.4000,1.8000)
\psdots[dotstyle=o,dotscale=3.0000](0.0000,3.0000)
\uput{0.3000}[45.0000](2.4000,1.8000){P:$\ (p_1^S,p_2^S,p_3^S)$}
\uput{0.3000}[45.0000](0.0000,3.0000){N}
%maison
\psline[linestyle=dashed](0,1.8)(2.4,1.8)
\uput{0.3}[35](0,1.8){$p_3^S$}
\psline[linestyle=dotted](0,0)(2.4,1.8)
\psarcn{->}(0,0){1}{90}{35}
\uput{0.3}[20](0,1){$\lambda^c$}

\endpsclip
\psaxes[labels=none,labelsep=1pt,Dx=1,Dy=1,ticks=none]{->}(0,0)(\xmin,\ymin)(\xmax,\ymax)
\end{pspicture*}
\columnbreak

\vspace{2cm}

Voici $\mathbb{R}^3$ en coupe par le demi-plan rep\'er\'e par la coordonn\'ee cylindrique $\xi_1-\xi_2$.

\par
$\lambda^c$ est la colatitude du point $P$, obtenu par projection stéréographique inverse du point ambiant $C$ de $\mathbb{R}^2$.

\par
On note que : $p_3^S=\cos \lambda^c$ soit $\lambda^c=\arccos p_3^S$.
\end{multicols}
Par d\'efinition de la sph\`ere-unit\'e : $(p_1^S)^2+(p_2^S)^2+(p_3^S)^2=1$.

\par
On peut lire sur le sch\'ema pr\'ec\'edent, avec le th\'eor\`eme de Thal\`es, que : %
$(p_1^S)^2+(p_2^S)^2=(1-p_3^S)^2\times\cot^2\frac{\phi}{2}$ ce qui entra\^ine $1-(p_3^S)^2=(1-p_3^S)^2\times\cot^2 \frac{\phi}{2}$
puis $\dfrac{1+p_3^S}{1-p_3^S}=\cot^2\frac{\phi}{2}$ et enfin $p_3^S=\dfrac{\cot^2 \frac{\phi}{2}-1}{\cot^2 \frac{\phi}{2}+1}$ %
-on retrouve un proc\'ed\'e d'inversion pour les fonctions homographiques.

\par
Moyennant une multiplication, au num\'erateur et au d\'enominateur, par la quantit\'e $\sin^2\frac{\phi}{2}$ qui est non nulle par hypoth\`ese sur $\phi$, %
la derni\`ere \'egalit\'e qui pr\'ec\`ede devient : %
$p_3^S=\dfrac{\cos^2 \frac{\phi}{2}-\sin^2\frac{\phi}{2}}{\cos^2\frac{\phi}{2}+\sin^2 \frac{\phi}{2}}$ soit \fbox{$p_3^S=\cos\phi$}

\par
On en d\'eduit l'\'egalit\'e : \[\phi=\lambda^c\]
Autrement dit, l\'el\'ement $\st{\phi}{\xi_1}{\xi_2}$ de $\mathbb{S}^3\setminus\mathbb{U}\times\{0\}$ est envoy\'e sur %
le point de $\mathbb{S}^2$ rep\'er\'e par les coordonn\'ees sph\'eriques $(\phi,\xi_1-\xi_2)$.

\par
R\'eciproquement, tout couple de coordonn\'ees $(\lambda^c,\theta)$ de $]0,\pi]\times\mathbb{R}$ admet, entre autres, l'\'el\'ement $\st{\lambda^c}{\theta}{0}$ %
comme ant\'ec\'edent par $\mathcal{P}_S$. \emph{nous venons de d\'efinir une surjection continue de $\mathcal{U}_S$ sur $U_S$, %
dont les fibres sont les traces sur $\mathbb{S}^3$ des droites vectorielles de $\mathbb{C}^2$, donc les fibres de $\mathbb{S}^3$ sous l'action de $\mathbb{U}$.}

\par
On peut d\'efinir de la m\^eme mani\`ere la projection $\mathcal{P}_N$ de $\mathbb{S}^3\setminus\{0\}\times\mathbb{U}$ sur $\mathbb{S}^2\setminus\{(0,0,-1)\}$ %
-on note respectivement $\mathcal{U}_N$ et $U_N$ ces deux ouverts de $\mathbb{S}^3$ et $\mathbb{S}^2$ respectivement-, %
en composant l'application $\st{\phi}{\xi_1}{\xi_2}$ par une identification de $\{0\}\times\mathbb{C}$ \`a $\mathbb{C}$ puis \`a $\mathbb{R}^2\times\{(0,0)\}$, %
et enfin par l'inverse de la projection st\'er\'eorgaphique $\varphi_N$ de $\mathbb{S}^2\setminus\{(0,0,-1)\}$ sur $\mathbb{R}^2\times\{0\}$.

\par
Cette construction pr\'esente toutefois l'inconv\'enient que : $\mathcal{P}_N\left(\st{\phi}{\xi_1}{\xi_2}\right)=(\phi ,\xi_2-\xi_1)$, %
autrement dit $\mathcal{P}_S$ et $\mathcal{P}_N$ diff\`erent, sur $\mathcal{U}_S\cap\mathcal{U}_N$, de ce qu'elles attribuent, en un m\^eme point de $\mathbb{S}^3$, %
deux \'el\'ements de $\mathbb{S}^2$ de m\^eme colatitude mais de longitudes \emph{oppos\'ees}.

\par
On rem\'edie \`a cela en intercalant, entre l'identification de $\{(0,0)\}\times\mathbb{C}$ \`a $\mathbb{C}$ et celle de $\mathbb{C}$ \`a ($\mathbb{R}^2$ puis) $\mathbb{R}^2\times\{0\}$, %
la conjugaison complexe pour construire, au lieu de $\mathcal{P}_N$, une projection $\mathcal{P}_{\overline{N}}$ de $\mathbb{U}_N$ sur $U_N$. %
On remarque que les fibres de $\mathcal{P}_{\overline{N}}$ sont toujours des orbites de $\mathbb{S}^3$ sous l'action de $\mathbb{U}$.

\ligneinter
On note $\mathcal{P}^+$ la projection de $\mathbb{S}^3$ sur $\mathbb{S}^2$ construite par recollement de $\mathcal{P}_S$ et de $\mathcal{P}_{\overline{N}}$. %
On remarque que $\mathcal{P}^+$ est continue, surjective et ferm\'ee, par ailleurs ses fibres sont exactement les orbites de $\mathbb{S}^3$ sous l'action de $\mathbb{U}$.

\ligneinter
Par ailleurs, on r\'ealise une deuxi\`eme projection $\mathcal{P}^-$ de $\mathbb{S}^3$ sur $\mathbb{S}^2$ en recollant l'application $\mathcal{P}_N$ qui pr\'ec\`ede %
et la projection $\mathcal{P}_{\overline{S}}$, elle-m\^eme d\'efinie en intercalant la conjugaison complexe, %
de la m\^eme mani\`ere que pour $\mathcal{P}_{\overline{N}}$, dans la construction de la projection $\mathcal{P}_S$.

\etoile
Dans toute la suite du texte, on notera indiff\'eremment un point de $\mathbb{S}^2$ et ses coordonn\'es \og{}colatitude , longitude\fg{} $(\lambda^c,\theta)$, %
on notera par ailleurs $\phi$ la colatitude qui correspond invariablement au param\`etre du m\^eme nom pour les \'el\'ements de $\mathbb{S}^3$.

\par
\paragraph{Structures de fibr\'es principaux}\label{fis}~\\
%\subsubsection{Structures de fibr\'es principaux}
%\emph{\textbf{Structures de fibr\'es principaux :}}

\par
Montrons maintenant que la projection $\mathcal{P}^+$ de $\mathbb{S}^3$ sur $\mathbb{S}^2$, elle-m\^eme munie du recouvrement d'ouverts $(U_S,U_N)$, %
d\'efinit une structure de fibr\'e principal de groupe de structure $\mathbb{U}$.

\par
Il suffit pour cela d'exhiber deux sections, que nous noterons $\se{S}{+}$ et $\se{N}{+}$ respectivement, %
de la projection $\mathcal{P}^+$, d\'efinies sur les calottes $U_S$ et $U_N$ de $\mathbb{S}^2$. On pose pour cela :
\[\forall (\phi,\theta)\in]0,\pi]\times\mathbb{R}\ \se{S}{+}((\phi,\theta))=\left(\cos\dfrac{\phi}{2}\ec{\theta},\sin\dfrac{\phi}{2}\right)\text{ et }%
\forall (\phi,\theta)\in[0,\pi[\times\mathbb{R}\ \se{S}{+}((\phi,\theta))=\left(\cos\dfrac{\phi}{2},\sin\dfrac{\phi}{2}\ec{(-\theta)}\right)\]
On pose maintenant, d'une part :
\[\forall ((\phi,\theta),\omega)\in U_S\times\mathbb{U}\ \Phi_S^+((\phi,\theta),\omega)=\se{S}{+}(\phi,\theta)\cdot\omega\]%\left(\cos\dfrac{\phi}{2}\ec{\theta}\cdot\omega,\sin\dfrac{\phi}{2}\cdot\omega\right)\]
et d'autre part :
\[\forall ((\phi,\theta),\omega)\in U_N\times\mathbb{U}\ \Phi_N^+((\phi,\theta),\omega)=\se{N}{+}(\phi,\theta)\cdot\omega\]%\left(\cos\dfrac{\phi}{2}\cdot\omega,\sin\dfrac{\phi}{2}\ec{(-\theta)}\cdot\omega\right)\]
On peut encore \'ecrire :
\[\Phi_S^+((\phi,\theta),\ec{\xi})=\st{\phi}{(\theta+\xi)}{\xi}\text{ et }%
\Phi_N^+((\phi,\theta),\ec{\xi})=\st{\phi}{\xi}{(\xi-\theta)}\]
pour tous $\phi$ et $\theta$ pour lesquels ces expressions sont d\'efinies, et tout r\'eel $\xi$.

\par
Pour \'etablir la continuit\'e de $\Phi_S^+$, il suffit de d\'emontrer celle de la section $\se{S}{+}$, %
qui se v\'erifie sur $U_S\setminus\{S\}$ qui est localement diff\'eomorphe \`a $]0,\pi [\times\mathbb{R}$ via %
$(\phi,\theta )\mapsto\left(\cos\theta\sin\phi ,\sin\theta\sin\phi ,1-\cos\phi\right)$, puis \og{}manuellement\fg{} en $S$.

\par
On peut d\'efinir les inverses respectifs $\Psi_S^+$ et $\Psi_N^+$ de $\Phi_S^+$ et $\Phi_N^+$ par :
\[\Psi_S^+\st{\phi}{\xi_1}{\xi_2}=\left((\phi,\xi_1-\xi_2),\ec{\xi_2}\right)\text{ et }\Psi_N^+\st{\phi}{\xi_1}{\xi_2}=\left((\phi,\xi_1-\xi_2),\ec{\xi_1}\right)\]
qui vont respectivement de $\mathcal{U}_S$ vers $U_S\times\mathbb{U}$ et de $\mathcal{U}_N$ vers $U_N\times\mathbb{U}$.

\par
Par construction : $\Phi_S^+\circ\Psi_S^+=Id_{\mathcal{U}_S}$ et $\Phi_N^+\circ\Psi_N^+=Id_{\mathcal{U}_N}$, %
de m\^eme $\Psi_S^+\circ\Phi_S^+=Id_{U_S\times\mathbb{U}}$ et $\Psi_N^+\circ\Phi_N^+=Id_{U_N\times\mathbb{U}}$.

\par
$\Phi_S^+$ et $\Psi_S^+$, respectivement $\Phi_N^+$ et $\Psi_N^+$, sont des hom\'eomorphismes r\'eciproques %
entre $U_S\times\mathbb{U}$ et $\mathcal{U}_S$, respectivement $U_N\times\mathbb{U}$ et $\mathcal{U}_N$.

\par
Enfin, les applications $\psi_S^+:(z_1,z_2)\mapsto\dfrac{z_2}{\abs{z_2}}$ et $\psi_N^+:(z_1,z_2)\mapsto\dfrac{z_1}{\abs{z_1}}$ sont $\mathbb{U}-$\'equivariantes \`a droite %
-par d\'efinition $\Psi_S^+=(\mathcal{P}^+,\psi_S^+)$ et $\Psi_N^+=(\mathcal{P}^+,\psi_N^+)$.
%de la $\mathbb{U}-$\'equivariance qui r\'esulte de la d\'efinition de $\Phi_S^+$ et $\Phi_N^+$ prouve enfin

\ligneinter
Conclusion : nous venons de construire deux trivialisations, $(U_S,\Psi_S^+)$ et $(U_N,\Psi_N^+)$, du fibr\'e $(\mathbb{S}^3,\mathbb{S}^2,\mathcal{P}^+,\mathbb{U})$ %
qui est donc un fibr\'e principal de groupe de structure $\mathbb{U}$.

\etoile
Le cas de $\mathcal{P}^-$ se traite sym\'etriquement, on note en particulier :
\[\left\{\begin{array}{rlcl}%
\forall (\phi,\theta)\in ]0,\pi]\times\mathbb{R}&\se{S}{-}(\phi ,\theta )&=&\st{\phi}{(-\theta )}{0}\\%
\forall (\phi,\theta)\in [0,\pi [\times\mathbb{R}&\se{N}{-}(\phi ,\theta )&=&\st{\phi}{0}{\theta}\\%
%\forall (\phi,\theta)\in ]0,\pi [\times\mathbb{R}&g_{SN}^-(\phi,\theta)&=&\ec{(-\theta)}\\%
\end{array}\right.\]
%\etoile
\dots isomorphie du point de vue des fibr\'es localement triviaux ?

\subsubsection{Une autre construction de $\mathbb{U}\hookrightarrow\mathbb{S}^2\twoheadrightarrow\mathbb{S}^2$}

\paragraph{Fibr\'es et espaces homog\g{e}nes, d'apr\g{e}s \cite{ex_par}}~\\
%\subsubsection{Fibr\'es et espaces homog\g{e}nes}


Le th\'eor\g{e}me suivant donne une m\'ethode g\'en\'erale, \'etant donn\'e un groupe de Lie qui agit contin\^ument sur un espace topologique, %
pour construire un fibr\'e principal.

\begin{theo}%[suggestion de Serge Parmentier]
\label{esh}
Soient $G$ un groupe de Lie et $H$ un sous-groupe ferm\'e de $G$.

\par
Alors le fibr\'e $H\hookrightarrow G\xrightarrow{\mathcal{P}}G/H$ est principal.
\end{theo}

Nous n'utiliserons pas ce r\'esultat dans la construction du fibr\'e, %
mais sa d\'emonstration est \'eclairante pour la suite. %
Elle est inspir\'ee des notes \cite{ex_par}.

\begin{lemm}[Section locale et trivialisation]
Soit $U$ un ouvert de $G/H$ et une section $s_U$ du fibr\'e $\mathcal{P}:G\rightarrow G/H$.

\par
Alors $\Phi_U:U\times H\mapsto G:(u,h)\mapsto s_U(u)\cdot h$ est un hom\'eomorphisme.
\end{lemm}

\begin{proof}
Il suffit de remarquer que $\Phi_U$ admet comme r\'eciproque, continue car $G$ est un groupe topologique :
\[\Psi_U:g\mapsto \left(gH,\left(s_U(\mathcal{P}(g))\right)^{-1}g\right)\]
\end{proof}

\begin{lemm}\label{sfl}
Soient $G$ un groupe de Lie et $H$ un sous-groupe ferm\'e de $G$.

\par
Alors le fibr\'e $\mathcal{P}:G\rightarrow G/H$ admet en tout point une section locale.
\end{lemm}

\begin{rema}
Dans la pratique, $G$ est un sous-groupe ferm\'e d'un groupe de matrices \g{a} coefficients dans $\mathbb{R}$ ou $\mathbb{C}$, %
son alg\g{e}bre de Lie est par d\'efinition l'espace tangent en l'identit\'e $e_G$ de cette sous-vari\'et\'e diff\'erentielle d'espace de matrices.

\par
L'exponentielle est d\'efinie naturellement.
\end{rema}

\begin{proof}
On proc\g{e}de en deux temps, le premier fait intervenir l'exponentielle et le th\'eor\g{e}me d'inversion locle :

\par
\begin{description}
\item[Section locale en $e_G$ :] Soient $\mathfrak{g}$ et $\mathfrak{h}$ les alg\g{e}bres de Lie respectives de $G$ et $H$.

\par
Soit $\mathfrak{m}$ un suppl\'ementaire de $\mathfrak{h}$ dans $\mathfrak{g}$. %
Nous allons construire un ouvert $U$ de $\mathfrak{m}$ tel que $\mathcal{P}_{|\exp(U)}$ est injective, continue et ouverte.

\ligneinter
Soit $f:\mathfrak{m}\times\mathfrak{h}\rightarrow G:(m,h)\exp(m)\exp(h)$. Cette fonction est analytique en $(0,0)$ et :
\[df_{(0,0)}=(m,h)\mapsto m+h\]

Cette application lin\'eaire est un isomorphisme : d'apr\g{e}s le th\'eor\g{e}me d'inversion locale, %
il existe un voisinage $V$ de $(0,0)$ dans $\mathfrak{m}\times\mathfrak{h}$ tel que $f$ induit un $C^{\infty}-$ difff\'eomorphisme de $V$ sur $f(V)$.

\par
Quitte \g{a} restreindre, on peut supposer que $V$ est un rectangle ouvert $U_n\times U_h$, $f$ est ainsi un diff\'eomorphisme de $U_m\times U_h$ sur $\exp(U_m)\times\exp(U_h)$.

\par
Par ailleurs le th\'eor\g{e}me d'inversion locale s'applique aussi \g{a} $\exp :\mathfrak{h}\rightarrow H$ au voisinage de $0_{\mathfrak{h}}$, %
on suppose maintenant que $\exp :U_h\rightarrow \exp(U_h)$ est un diff\'eomorphisme.

\par
Puisque $\exp(U_h)$ est un voisinage de l'unit\'e dans $H$, on peut l'\'ecrire : $W\cap H$, o\g{u} $W$ est un voisinage ouvert de l'unit\'e dans $G$.

\par
La continuit\'e de l'exponentielle sur $U_m$ et celle de la loi inverse de $G$ prouvent qu'il existe un ouvert $U$ de $U_m$ tel que :

\[\exp(-U)\exp(U)\subset W\]
\begin{itemize}
\item Montrons que $\mathcal{P}$ est injective sur $\exp(U)$.

\par
Soient $m_1$ et $m_2$ deux \'el\'ements de $U$ tels que $\mathcal{P}(\exp(m_1))=\mathcal{P}(\exp(m_2))$. %
Montrons que \[\exp m_1=\exp m_2\]

\par
On peut \'ecrire : $\exists h\in H | \exp m_1=\exp m_2 \cdot h$, ce qui entra\^ine $\exp(-m_2)\exp(m_1)\in H$. %
Par d\'efinition de $U$, il s'ensuit que $\exp(-m_2)\exp(m_1)\in W\cap H$, autrement dit $\exp(-m_2)\exp(m_1)\in \exp(U_h)$.

\par
L'injectivit\'e de $f$ sur $U_m\times U_h$ permet d'en d\'eduire : $m_1=m_2$, donc $\exp m_1=\exp m_2$.
\item $\mathcal{P}$ est continue sur $\exp (U)$ par d\'efinition ;
\item ouvertitude \dots
\end{itemize}

\par
Il reste \g{a} montrer que $\mathcal{P}_{|\exp(U)}^{-1}$ est une section locale de $\mathcal{P}$, c'est-\g{a}-dire que $\mathcal{P}(U)$ est un voisinage de l'unit\'e dans $G/H$.

\par
Pour cela, on note que puisque $f$ induit un diff\'eomorphisme de $U\times U_h$ sur $\exp (U)\exp(U_h)$, ce dernier ensemble est un ouvert de $G$, %
de plus $\mathcal{P}$ est ouverte donc le projet\'e $\mathcal{P}(\exp(U))$ de $\exp (U)\exp(U_h)$ sur $G/H$ est aussi ouvert.

\item[Section locale en un \'el\'ement $g$ quelconque de $G$ :] Soit $g\in G$.
%
%\par
%On note respectivement $l_g$ de $\tau_g}$ les translations \g{a} gauche, dans $G$ et $G/H$, par $g$ et $gH$.

\par
Alors $\mathcal{P}(\exp(U))\rightarrow G:\bar{x}\mapsto g\mathcal{P}_{|\exp(U)}^{-1}(\overline{g^{-1}x})$ est une section de $\mathcal{P}$ au voisinage de $g$.
\end{description}
\end{proof}

\begin{proof}[D\'emonstration du th\'eor\g{e}me]
C'est une cons\'equence imm\'ediate des deux r\'esultats qui pr\'ec\g{e}dent.
\end{proof}

%\etoile

Dans le paragraphe suivant, on utilise la m\'ethode pr\'ec\'edente pour construire un $\mathbb{U}-$fibr\'e au-dessus de $\mathbb{S}^2$, ici vue comme un espace homog\g{e}ne.

\paragraph{Quaternions et groupes lin\'eaires}~\\
%\subsubsection{Quaternions}

\par
La construction -classique- des quaternions qui va suivre, propos\'ee dans \cite{dev_par}, %
permet de d\'efinir une action du groupe $SU(2)$ sur la sph\g{e}re unit\'e $\mathbb{S}^2$ d'un espace euclidien isomorphe \g{a} $\mathbb{R}^3$, %
et dont les stabilisateurs sont isomorphes \g{a} $\mathbb{U}$.

\par
On peut ensuite utiliser directement le th\'eor\g{e}me \ref{esh} pour d\'efinir un fibr\'e principal : $\mathbb{U}\hookrightarrow\mathbb{S}^3\twoheadrightarrow\mathbb{S}^2$, %
isomorphe \g{a} une fibration de Hopf.

\ligneinter
On d\'efinit l'espace vectoriel r\'eel $\mathbb{H}$ des \textbf{quaternions} comme ensemble des solutions, %
dans l'anneau matriciel $\mathcal{M}_2(\mathbb{C})$, de l'\'equation :
\[\text{com }M=\overline{M}\]

Autrement dit $\mathbb{H}$ est l'ensemble des matrices complexes carr\'ees d'ordre $2$ qui s'\'ecrivent :
\[\left(\begin{array}{cc}a&-b\\ \overline{b}&\overline{a}\end{array}\right)\text{ o\g{u} $a$ et $b$ sont deux nombres complexes}\]

On remarque ainsi que le $\mathbb{R}-$espace vectoriel $\mathbb{H}$ est engendr\'e par les quatre matrices :
\[I_2\text{ ; }\mcd{\mathbf{i}}{0}{0}{-\mathbf{i}}\text{ ; }\mcd{0}{-1}{1}{0}\text{ et }\mcd{0}{\mathbf{i}}{\mathbf{i}}{0}\]

Les trois derni\g{e}res matrices, que nous noterons respectivement $\mathbf{I}$, $\mathbf{J}$ et $\mathbf{K}$ v\'erifient de plus :
\[\mathbf{I}^2=-I_2\text{ ; }\mathbf{J}^2=-I_2\text{ ; }\mathbf{K}^2=-I_2\text{ et }\mathbf{IJK}=-I_2\]

Ces matrices sont antihermitiennes et engendrent un $\mathbb{R}-$espace vectoriel de dimension $3$, %
que nous noterons $\mathcal{V}$, dont les \'el\'ements sont appel\'es \emph{quaternions purs} ou \emph{vecteurs}. %
La somme directe : $\mathbb{R}I_2\oplus\mathcal{V}$ est la trace sur $\mathbb{H}$ de la d\'ecomposition de $\mathcal{M}_2(\mathbb{C})$ %
en espace des matrices hermitiennes et antihermitiennes.

\par
L'\'equation matricielle qui d\'efinit ici $\mathbb{H}$ prouve que :
\[
%\begin{align*}
\begin{array}{lccr}
\forall M\in\mathbb{H}& M^{\ast}M&=&(\det M)I_2\\
\forall M\in\mathbb{H}& MM^{\ast}&=&(\det M)I_2
\end{array}
%\end{align*}
\]
o\g{u} pour tout $M$, $M^{\ast}$ d\'esigne la matrice adjointe, transconjongu\'ee, de $M$.

\par
On remarque par ailleurs que, quels que soient les complexes $a$ et $b$ :
\[\det\mcd{a}{-b}{\overline{b}}{\overline{a}}=|a|^2+|b|^2\]
autrement dit $\det$ induit, sur $\mathbb{H}$, la forme quadratique dont d\'erive le produit scalaire canonique $\langle | \rangle$ du $\mathbb{R}-$espace vectoriel $\mathbb{H}$, %
muni de sa base $(I_2,\mathbf{I},\mathbf{J},\mathbf{K})$.

\par
Il s'ensuit notamment que les quaternions non nuls sont tous des matrices complexes inversibles.

\par
Puisque l'op\'erateur \og{}comatrice\fg{} est multiplicatif, $\mathbb{H}$ est, d'apr\g{e}s l'\'equation qui le d\'efinit, stable par multiplication matricielle.

\par
Enfin, comme l'op\'erateur \og{}comatrice\fg{} pr\'eserve la transposition et la conjugaison, %
la transconjugu\'ee d'une patrice \'el\'ement de $\mathbb{H}$ est encore dans $\mathbb{H}$. %
On en d\'eduit notamment, avec les deux \'egalit\'es ci-dessus qui relient une matrice de $\mathbb{H}$, son adjointe et l'identit\'e, %
qu'une matrice de $\mathbb{H}$ de d\'eterminant non nul est unversible \emph{dans $\mathbb{H}$}.

\par
Voici un premier r\'esultat important :

\begin{prop}
La sph\g{e}re-unit\'e $\mathbb{S}^3$ de l'espace euclidien $\mathbb{H}$ est exactement l'ensemble $SU_2(\mathbb{C})$ des matrices sp\'eciales unitaires \g{a} coefficients complexes.
\par
En particulier, $\mathbb{S}^3$ est munie, par la multiplication matricielle, d'une structure de groupe topologique -m\^eme de groupe de Lie.
\end{prop}

\begin{proof}
Soit $M$ une matrice de $\mathbb{H}$, de norme $1$ pour $\langle | \rangle$.

\par
Puisque : $\det M=\|M\|_2^2$, on en d\'eduit que $M$ est de d\'eterminant $1$.

\par
Il s'ensuit que $M$ et $M^{\ast}$ sont inverses, donc $M$ est une matrice unitaire, ce qui nous donne le r\'esultt esp\'er\'e.

\par
R\'eciproquement, soit : $U\in SU_2(\mathbb{C})$.

\par
On note que : $U^{\ast}U=\det U\cdot I_2$. Il s'ensuit que $^t\overline{U}=^t\text{com }U$ ce qui entra\^ine $\text{com }U=\overline{U}$ donc $U\in\mathbb{H}$.

\par
Puisque $U$ est de d\'eterminant $1$, il est de norme $1$ dans $\mathbb{H}$ donc : $U\in\mathbb{S}^3$.
\end{proof}

Comme $\mathbb{H}$ est stable par produit, on peut d\'efinir l'action $\chi$ de $\mathbb{H}^{\ast}$ sur $\mathbb{H}$ par conjugaison. %
Comme le d\'eterminant fait office de norme, om montre simplement que $\mathbb{H}^{\ast}$ agit sur $\mathbb{H}$ par isom\'etries, %
un calcul simple avec les premi\g{e}res formules \'etablies concernant $\mathbb{H}$ prouve aussi que $\chi$ pr\'eserve la d\'ecomposition $\mathbb{R}I_2\oplus\mathcal{V}$.

\par
On peut ainsi restreindre $\chi$ en une action, que nous noterons encore $\chi$, du groupe $\mathbb{S}^3$ sur la sph\g{e}re-unit\'e $\mathbb{S}^2$ de l'espace euclidien $\mathcal{V}$.

\begin{prop}
\begin{itemize}
\item $\chi$ est transitive.
\item Le stabilisateur de $\mcd{\mathbf{i}}{0}{0}{-\mathbf{i}}$ est isomorphe au groupe $\mathbb{U}$ des nombres complexes de module $1$.
\end{itemize}
\end{prop}

\begin{rema}
On peut aussi \'ecrire : $\forall (U,S)\in\mathbb{S}^2\times\mathbb{S}^2 , \chi(U,S)=U^{\ast}SU$.
\end{rema}
%Remarquons que cette proposition, coupl\'ee avec le th\'eor\g{e}me\ref{esh}, nous donne une construction de $\mathbb{U}-$ fibr\'e principal au-dessus de $\mathbb{S}^2$ ; %
%son espace de phases est $\mathbb{S}^3$, muni d'une structure de groupe de Lie.

\begin{proof}
Montrons que : $\mathbb{S}^2=\{S\in\mathbb{H}|s^2=-I_2\}$.
\par
Supposons en effet : $s\in\mathbb{S}^2$. On note que $S$ est de d\'eterminant $1$, donc que : $SS^{\ast}=I_2$. Or $S$ est antihermitienne donc $S^2=-I_2$.
\par
R\'eciproquement, soit $S$ un quaternion de carr\'e \'egal \g{a} $-1$. On peut \'ecrire : $S^{-1}=-S$, donc $\dfrac{S^{\ast}}{\|c\|^2}=-S$. %
Cette \'egalit\'e prouve du m\^eme coup : $s\in\mathcal{V}$ et $\det s=1$, donc $s\in\mathbb{S}^2$.
\par
Ainsi, les \'el\'ements de $\mathbb{S}^2$ sont tous diagonalisables, de m\^eme spectre $\{-\mathbf{I},\mathbf{I}\}$.
\par
On utilise maintenant, sans d\'emonstration, le r\'esultat suivant :

\begin{prop}
Soit $n$ un entier naturel non nul, et $M$ une matrice complexe $n\times n$, antihermitienne. Alors :
\begin{itemize}
\item Les valeurs propres de $M$ sont imaginaires pures ;
\item $M$ est diagonalisable, via une matrice complexe unitaire.
\end{itemize}
\end{prop}

D'apr\g{e}s ce qui pr\'ec\g{e}de :
\[\forall S\in\mathbb{S}^2,\exists U\in U_2(\mathbb{C})|S=U^{-1}\mathbf{I}U\]

Soit $\lambda$ un nombre complexe tel que : $\lambda^2=\det U$. On peut encore \'ecrire :
\[\left(\dfrac{U}{\lambda}\right)^{-1}\mathbf{I}\left(\dfrac{U}{\lambda}\right)=s\]

Montrons que : $\dfrac{U}{\lambda}\in SU_2(\mathbb{C})$.

\par
Tout d'abord : $\left(\dfrac{U}{\lambda}\right)^{\ast}\cdot\dfrac{U}{\lambda}=\dfrac{I_2}{|\det U|}$. %
Puisque le d\'eterminant d'une matrice unitaire est de module $1$, on en d\'eduit aue $\frac{U}{\lambda}$ est une matrice unitaire.

\par
De plus, par d\'efinition de $\lambda$ : $\det \frac{U}{\lambda}=1$, ce qui nous donne le r\'esultat esp\'er\'e. Ainsi :
\[S=\chi\left(\frac{U}{\lambda},\mathbf{I}\right)\]
ce qui nous donne la transitivit\'e de $\chi$.

\par
Soit maintenant : $G\in SU_2(\mathbb{C})_{\mathbf{I}}$. On remarque que $G$ commute avec la matrice $\mcd{\mathbf{i}}{0}{0}{-\mathbf{i}}$, %
donc l'endomorphisme de $\mathbb{C}^2$ canoniquement associ\'e \g{a} $G$ stabilise les droites vectorielles $\mathbb{C}(1,0)$ et $\mathbb{C}(0,1)$, %
on peut donc \'ecrire :
\[G=\mcd{\lambda}{0}{0}{\lambda^{-1}}\text{ puisque $G$ est de d\'eterminant $1$.}\]
Enfin, puisque $^t\overline{G}=G^{-1}$, $\lambda$ est de module $1$.

\par
R\'eciproquement, si $\omega\in\mathbb{U}$ : $\mcd{\omega}{0}{0}{\omega^{-1}}\in SU_2(\mathbb{C})_{\mathbf{I}}$.

\par
$\omega\mapsto\mcd{\omega}{0}{0}{\omega^{-1}}$ est donc un isomorphisme entre les groupes topologiques $\mathbb{U}$ et $SU_2(\mathbb{C})_{\mathbf{I}}$.
\end{proof}

\etoile
\emph{Nous avons ainsi construit, d'apr\g{e}s le th\'eor\g{e}me\ref{esh}, un $\mathbb{U}-$ fibr\'e au-dessus de $\mathbb{S}^2$, %
avec un espace de phases hom\'eomorphe \g{a} $\mathbb{S}^3$.}

\begin{rema}
$\chi$ induit une action par conjugaison matricielle de $SU_2(\mathbb{C})$ sur $\mathcal{V}$.
\par
Cette action n'est pas fid\g{e}le et son noyau est $\{-I_2,I_2\}$. %
De plus $\dfrac{SU_2(\mathbb{C})}{\{-I_2,I_2\}}$ agit simplement, via le quotient de $\chi$ par son noyau, %
par isom\'etries directes sur l'espace euclidien de dimension trois $\mathcal{V}$.

\par
Par ailleurs, le morphisme $u\mapsto \left(p\mapsto u^{-1}pu\right)$ de $SU_2(\mathbb{C})$ vers $SO(\mathcal{V})$ est \emph{surjectif}, %
ceci entra\^ine notamment que les groupes $\dfrac{SU_2(\mathbb{C})}{\{-I_2,I_2\}}$ et $SO(3,\mathbb{R})$ sont isomorphes.
\end{rema}

\begin{comment}
\etoile
D\'ecliner la d\'emonstration du th\'eor\g{e}me\ref{esh} au cas particulier de $\mathbb{S}^3\twoheadrightarrow\dfrac{\mathbb{S}^3}{\mathbb{S}^3_{\mathbf{I}}}$ %
pour construire explicitement un syst\g{e}me de trivialisations.

\par
Formule explicite pour une section ? Interpr\'etation g\'eom\'etrique ?
\end{comment}

\paragraph{Structure de fibr\'e principal}

\par
Afin d'expliciter cette structure construite \g{a} partir de l'action $\chi$, %
on suit la d\'emonstration du th\'eor\g{e}me \ref{esh} vue pr\'ec\'edemment, sans utiliser ce dernier r\'esultat.

\subsubsection{Fonctions de transition}

Les fibr\'es principaux sont munis de fonctions de transition, analogues aux changements de carte pour les variétés. La proposition suivante, évidente, permet de définir ces objets.

\begin{prefi}\label{ftr}
Soit $G \overset{\ast}{\hookrightarrow} P \overset{\mathcal{P}}{\twoheadrightarrow} X$ un fibr\'e principal, %
et $(V_i,\Phi_i)_{i \in I}$ une famille de trivialisations de ce fibré, telle que $(V_i)_i$ recouvre $X$.
\par
Alors, quels que soient deux indices $i$ et $j$, non nécessairement distincts, pour la trivialisation $(V_k,\Psi_k)_k$ tels que : $V_i \cap V_j \neq \varnothing$, %
et pour tout élément $x$ de cette intersection :
\[\mathcal{P}^{-1}(x) \rightarrow G : p \mapsto \psi_j(p) (\psi_i (p))^{-1}\text{ est constante.}\]
On a ainsi d\'efini, pour un tel couple $(j,i)$, une fonction $g_{ji}$ de $V_i \cap V_j$ dans $G$ que nous appellerons application de transition pour le fibré. %
La famille $(g_{ji})_{(j,i)}$ ainsi définie vérifie alors les propriétés suivantes :
\begin{itemize}
\item $\forall i \in I , g_{ii} = \tilde{e_G}$ o\g{u} $e_G$ est le neutre de $G$ ;
\item $\forall (i,j) \in I^2 , V_i \cap V_j \neq \varnothing \Rightarrow g_{ij} = g_{ji}^{-1}$ ;
\item[Propri\'et\'e de cocycle :] $\forall (i,j,k) \in I^3 , V_i \cap V_j \cap V_k \neq \varnothing \Rightarrow g_{kj}g_{ji} = g_{ki}$
\end{itemize}
%La dernière formule est appelée propriété de cocycle, elle résume d'ailleurs les deux autres. On note, dans le cas où $G$ et $P$ sont lisses, que ces fonctions de transition sont toujours lisses.
\end{prefi}

\begin{proof}
En effet on note, avec la locale trivialité pour $\mathcal{P}$ sur $V_i \cap V_j$, que la fibre d'un tel élément $x$ est en bijection $G$-équivariante avec $%
\{x\} \times G$ via $\Phi_i$ et via $\Phi_j$, ce qui donne le résultat.
\end{proof}

Reprise de tous les exemples qui pr\'ec\g{e}dent.

\begin{exem}[Fibrations de Hopf]
Nous pouvons maintenant calculer les deux fonctions de transitions -mises \`a part les fonctions \'evidentes pour la transition d'un ouvert de $\mathbb{S}^2$ sur lui-m\^eme- %
$g_{SN}^+$ et $g_{NS}^+$, du fibr\'e principal ainsi d\'efini, selon la notation du paragraphe \dots .
\par
Rappelons que : \[\se{S}{+}(\phi ,\theta )=\left(\cos\frac{\phi}{2}\ec{\theta},\sin\frac{\phi}{2}\right)\text{ et }%
\se{N}{+}(\phi ,\theta )=\left(\cos\frac{\phi}{2},\sin\frac{\phi}{2}\ec{(-\theta)}\right)\]
quels que soient $\phi$ et $\theta$ tels que $\phi$ soit diff\'erent de $0$ ou $\pi$.
\par
Ainsi, avec la m\^eme notation : $\psi_S^+\left(\cos\frac{\phi}{2}\ec{\theta},\sin\frac{\phi}{2}\right)=1$ et $\psi_N^+\left(\cos\frac{\phi}{2}\ec{\theta},\sin\frac{\phi}{2}\right)=\ec{\theta}$. %
La formule : $\psi_N^+=g_{NS}^+\psi_S^+$, qui ne d\'epend que de $\phi$ et $\theta$ par $\mathbb{U}$-\'equivariance des fonctions de phase $\psi_S^+$ et $\psi_N^+$, %
permet de conclure :\[\mathbf{g_{SN}^+(\phi ,\theta )=\ec{\theta}}\]
\par
On en d\'eduit naturellement la seule autre fonction de transition non triviale du fibr\'e $(\mathbb{S}^3,\mathbb{S}^2,\mathcal{P}^+,\mathbb{U})$ que nous venons de d\'efinir : %
$g_{SN}^+(\phi ,\theta )=\ec{(-\theta )}$.
\etoile
Sym\'etriquement :
\[\forall (\phi,\theta)\in ]0,\pi [\times\mathbb{R}g_{SN}^-(\phi,\theta)=\ec{(-\theta)}\]
\end{exem}

%\newpage

\subsection{Equivalences de fibr\'es}

\subsubsection{D\'efinition : fibr\'es \'equivalents}

\begin{prefi}
Soient $(P_1,X_1,\mathcal{P}_1,G)$ et $(P_2,X_2,\mathcal{P}_2,G)$ deux fibrés principaux de même groupe de structure $G$.

\par
On considère une application $\tilde{f}$, continue et $G$-équivariante à droite, de $P_1$ dans $P_2$. Alors :
\begin{itemize}
\item $\tilde{f}$ envoie chaque $G$-orbite de $P_1$ sur une $G$-orbite de $P_2$,
\item $\tilde{f}$ induit donc, via les projections $\mathcal{P}_1$ et $\mathcal{P}_2$ qui sont des applications ouvertes, %
une application $f$ de $X_1$ dans $X_2$, cette application est continue.
\end{itemize}
\[\xymatrix{%
P_1\ar[d]_{\mathcal{P}_1}\ar[r]^{\tilde{f}}&P_2\ar[d]^{\mathcal{P}_2}\\%
X_1\ar@{-->}[r]_{f}&X_2%
}\]
Nous avons ainsi d\'efini un morphisme entre les deux fibr\'es $(P_1,X_1,\mathcal{P}_1,G)$ et $(P_2,X_2,\mathcal{P}_2,G)$.
\end{prefi}

\begin{proof}
Facile.
\end{proof}

\begin{rema}%[Importance ?]
Les $G$-orbites de $P_1$ et $P_2$, simples, sont bien sûr toutes hom\'eomorphes entre elles. %
De plus, $\tilde{f}$ induit un hom\'eomorphisme, entre chaque $G$-orbite de $P_2$ à l'arrivée et son image réciproque.

\par
Soit en effet $p_1$ un élément de $P_1$. Tout ouvert de $p_1G$ est de la forme : $p_1O$ o\g{u} $O$ est un ouvert de $G$. % d'après \dots%la la remarque\ref{gr1}.\\
L'image d'un tel ouvert de $p_1G$ est $\tilde{f}(p_1)O$ puisque $\tilde{f}$ est $G-$\'equivariante \g{a} droite.

\par
Soit $(V_2,(\mathcal{P}_2,\psi_2))$ une trivialisation au-dessus de $X_2$ telle que : $\mathcal{P}_2(\tilde{F}(p_1)) \in V_2$, %
on note $\mathcal{V}_2$ l'ouvert correspondant dans $P_2$.

\begin{comment}
Comme $\tilde{f}$ est continue l'image réciproque de $\mathcal{V}_2$ par cette application est un ouvert de $P_1$, satur\'e comme $\tilde{f}$ est $G$-équivariante. %
Comme l'ensemble des ouverts de trivialisation au-dessus de $X_1$ est une base il existe une trivialisation $(V_1,(\mathcal{P}_1,\psi_1))$ au-dessus de $X_1$, %
telle que $p_1$ soit contenu dans l'image réciproque $\mathcal{V}_1$, de $V_1$ par $\mathcal{P}$.

\par
On remarque que $\mathcal{V}_1$ est envoyé de manière $G$-\'equivariante à droite, non nécessairement ouverte, dans $\mathcal{V}_2$. %
Comme $\tilde{f}$ est $G$-équivariante à droite, la section :
\[\{p \in \mathcal{V}_1 | \psi_1(p) \in \psi_1(p_1)O\}\]
de l'ouvert de trivialisation $\mathcal{V}_1$ est envoyée, de manière non nécessairement ouverte ni surjective, vers :
\[\{p \in \mathcal{V}_2 | \psi_2(p) \in \psi_2(\tilde{f}(p_1))O\}\]
dans $\mathcal{V}_2$, nous notons $S_2$ ce dernier ensemble, ouvert dans $P_2$.
\end{comment}

\par
Soit maintenant $S_2$ l'ouvert, satur\'e, de $P_2$ d\'efini par :
\[S_2=\{p \in \mathcal{V}_2 | \psi_2(p) \in \psi_2(\tilde{f}(p_1))O\}\]
On remarque que : $\tilde{f}(p_1) O = \tilde{f}(p_1) G \cap S_2$, on en d\'eduit que cet ensemble, \'egal \g{a} $\tilde{f}(p_1O)$, est ouvert dans la $G$-orbite de $\tilde{f}(p_1)$.

\par
Ainsi, $\tilde{f}$ induit une application ouverte, bijective par $G-$\'equivariance \g{a} droite, de $p_1G$ vers $\tilde{f}(p_1)G$. La continuit\'e de cette application est \'evidente.
\end{rema}

\begin{prop}[Vraie ? Importance en pratique ?]
On reprend la notation de la d\'efinition pr\'ec\'edente.

\par
Si $f$ est un hom\'eomorphisme de $X_1$ sur $X_2$, alors son rel\g{e}vement $\tilde{f}$, est un hom\'eomorphisme de $P_1$ sur $P_2$.
\end{prop}

\begin{proof}
Il suffit de montrer que si $f$ est ouverte et bijective, alors il en est de m\^eme pour $\tilde{f}$.

\par
Supposons tout d'abord que $f$ soit bijective \dots facile.

\par
Supposons de plus que $f$ soit ouverte. \tr%Soit $\mathcal{U}_1$ un ouvert, non n\'ecessairement satur\'e, de $P_1$. %
%On note $U_1$ le projet\'e de cet ensemble sur $X_1$ de $\mathcal{U}_1$. On note que $U_1$ est ouvert, de m\^eme que son image $f(U_1)$ par $f$. %
%Afin de montrer que $\tilde{f}(\mathcal{U}_1)$ est ouvert, il suffit de montrer 
\end{proof}

Nous sommes maintenant arm\'es pour le d\'efinition \'eponyme :

\begin{defi}[Equivalence de fibr\'es]
Soient $(P_1,X,\mathcal{P}_1,G)$ et $(P_2,X,\mathcal{P}_2,G)$ deux fibr\'es principaux de groupe de structure commun $G$, sur une m\^eme base $X$.

\par
On suppose qu'il existe une application continue $\tilde{f}$, $G$-\'equivariante à droite, entre $P_1$ et $P_2$ qui induit, selon la construction de la d\'efinition pr\'ec\'edente, %
l'application identit\'e sur $X$. %
$\tilde{f}$ est appelée \emph{\'equivalence entre les fibr\'es} $G\hookrightarrow P_1 \xrightarrow{\mathcal{P}_1} X$ et $G\hookrightarrow P_2 \xrightarrow{\mathcal{P}_2} X$.

\par
La proposition pr\'ec\'edente entra\^ine que $\tilde{f}$ est alors un hom\'eomorphisme, sa réciproque $\tilde{f}^{-1}$ d\'efinit aussi une \'equivalence.
\end{defi}

\begin{rema}
\begin{itemize}
\item D'apr\g{e}s la proposition pr\'ec\'edente deux fibr\'es \'equivalents ont des espaces totaux isomorphes.
\item Si les deux fibr\'es sont \'egaux alors $\tilde{f}$ est un automorphisme de fibr\'e.
\end{itemize}
\end{rema}

\begin{exem}
Soit \Fig un fibr\'e muni d'une trivialisation globale $(X,\Phi)$ :
\[\xymatrix{X \times G & P \ar[l]^{\Phi} \ar[d]^{\mathcal{P}}\\ & X}\]
$\Phi$ est une \'equivalence de $(P,X,\mathcal{P},G)$ vers le $G$-fibr\'e trivial au-dessus de $X$.
\end{exem}

\subsubsection{Th\'eor\`emes sur la classification de fibr\'es principaux}

\begin{theo}[Fibr\'es \'equivalents]\label{fbl}
Soient $G \hookrightarrow P_1 \xrightarrow{\mathcal{P}_1} X$ et $G \hookrightarrow P_2 \xrightarrow{\mathcal{P}_2} X$ deux fibrés principaux.

\par
On peut d'après la remarque \dots consid\'erer une base de trivialisation commune $(V_i)_i$ commune aux deux fibr\'es, %
on définit également les familles $(g^1_{ji})$ et $(g^2_{ji})$ de fonctions de transition pour ces deux fibrés munis de cette base de trivialisation.

\par
$G \hookrightarrow P_1 \xrightarrow{\mathcal{P}_1} X$ et $G \hookrightarrow P_2 \xrightarrow{\mathcal{P}_2} X$ sont alors \'equivalents si et seulement si %
il existe une famille $(\lambda_i)_i$ de fonctions continues, d\'efinies sur les ouverts de la base $(V_i)_i$ et continues \g{a} valeurs dans $G$, v\'erifiant de plus :
\[\forall x \in V_j \cap V_i , g^2_{ji}(x) = (\lambda_j(x))^{-1}g^1_{ji}(x)\lambda_i(x)\]
lorsque $V_j$ et $V_i$ se recoupent.
\end{theo}

\begin{proof}
Supposons que ces deux fibrés soient équivalents, via un homéomorphisme $G$-équivariant $\tilde{f}$ entre les espaces totaux.

\par
Soient $(V_i,(\mathcal{P}_1,\psi_1^i))$ et $(V_i,(\mathcal{P}_2,\psi_2^i))$ des trivialisations respectives pour les deux fibrés, pour un même indice $i$. %
On peut définir, au-dessus de chaque élément $x$ de $V_i$, le déphasage $\lambda_i(x)$ impliqué par l'action de $\tilde{f}$ avec la formule :

\[\lambda_i(x) = \psi_2^i(\tilde{f}(s_{V_i}^1(x))\]

que l'on peut étendre $G$-équivariance à droite :
\[\forall p \in \mathcal{P}_1^{-1}(x) , \psi_2^i(\tilde{f}(p)) = \lambda_i(x) \psi_1^i(p)\]
La fonction $\lambda_i$ est ainsi définie sans ambigüité sur $V_i$, à l'aide de la fonction continue $\psi_2^i (\psi_1^i)^{-1}$.

\par
Montrons maintenant la formule énoncée, sur les familles $(g_{ji})$ et $(\lambda_i)$.

\par
Soient $i$ et $j$ deux indices de la famille de trivialisations tels que :
\[V_j \cap V_i \neq \varnothing\]

On \'etudie la correspondance $\tilde{f}$ au-dessus de cette intersection adopte les notations %
$s^1_{V_i}$, $s^2_{V_i}$, $s^1_{V_j}$, $s^2_{v_j}$, $\lambda_i$, $\lambda_j$, $g^1_{ji}$, $g^2_{ji}$ suggérées dans les paragraphes qui précèdent.

\par
Soit $x$ un élément de $V_j \cap V_i$. Par définition :
\[s^2_{V_j}(x) g_{ji}^2(x) = s^2_{V_i}(x)\]

Par ailleurs :
\[s^2{V_j}(x) \lambda_j(x) = \tilde{f}(s^1_{V_j}(x))\text{ et }\tilde{f}(s^1_{V_j}(x)) g^1_{ji}(x) = \tilde{f} (s^1_{V_i}(x))\]

comme $\tilde{f}$ est $G$-équivariante à droite, et encore par définition :
\[\tilde{f}(s^1_{V_i}(x)) (\lambda_i(x))^{-1} = s^2_{V_i}(x)\]

nous pouvons conclure : $g_{ji}^2(x) = \lambda_j(x) g^1_{ji}(x) (\lambda_i(x))^{-1}$.

Supposons réciproquement pour deux fibrés $G \hookrightarrow P_1 \xrightarrow{\mathcal{P}_1} X$ %
et $G \hookrightarrow P_2 \xrightarrow{\mathcal{P}_2} X$, qu'une famille $(\lambda_i)$ de fonctions de $X$ dans $G$ existe, %
et relie les fonctions de transitions respectives des deux fibrés, selon la formule qui précède.

\par
On définit donc, au-dessus de tout ouvert $V_i$ de la trivialisation, $\tilde{f}$ de sorte que :
\[\forall x \in V_i , \tilde{f}(s^1_{V_i}(x)) = s^2_{V_i}(x) \lambda_i(x)\]

Cette formule équivaut, par $G$-équivariance à droite, à :
\[\forall p \in \mathcal{P}_1^{-1}(V_i) , \psi_2^i(\tilde{f}(p)) = \lambda_i (x) \psi_1^i(p)\text{, où $p$ se projette en }x\]

Montrons que cette définition est cohérente, %
autrement dit que les formules établies pour tout indice $i$ sont compatibles, %
pour toute intersection non vide d'ouverts de trivialisation sur $X$.

\par
Soient en effet deux indices $i$ et $j$ de la famille d'ouverts de trivialisation sur $X$, tels que : $V_j \cap V_i \neq \varnothing$. %
On suppose $\tilde{f}$ définie, sur $V_i$ et en particulier l'intersection précédente, avec la formule :
\[\forall p \in \mathcal{P}_1^{-1}(V_i) , \psi_2^i(\tilde{f}(p)) = \lambda_i (x) \psi_1^i(p)\]

On rappelle la relation :
\[g^2_{ji}(x) = \lambda_j(x) g^1_{ji}(x) (\lambda_i(x))^{-1}\]
sur $V_j \cap V_i$.

\par
Si l'on multiplie terme à terme, la deuxième à gauche de la première, les formules qui précèdent on peut écrire %:
$\forall p \in V_j \cap V_i , g^2_{ji}(x) \psi_2^i(p) = \lambda_j(x) g^1_{ji}(x) \psi_1^i(p)$, par télescopage avec $\lambda_i$; on en déduit :
\[\forall p \in V_j \cap V_i , \psi_2^j(p) = \lambda_j(x) \psi_1^j(p)\]
ce qui concide avec la d\'efinition de $\tilde{f}$ sur $V_j$.
\end{proof}

%Les invariants complets $g_{ij}$ qui apparaissent dans la d\'emonstration pr\'ec\'edente d\'etiennent, de fait, toute l'information de la structure de fibré principal pour le groupe $G$ :

\begin{theo}[Th\'eor\`eme de reconstruction]
Soit $X$ un espace topologique, $G$ un groupe topologique et $(V_i)_{i \in I}$ un recouvrement ouvert de $X$.

\par
On d\'efinit, pour tout couple $(i,j)$ d'indices de cette famille d'ouverts, tel que $V_j$ et  $V_i$ ne soient pas disjoints, %
une fonction continue $g_{ji}$ de $V_j \cap V_i$ dans $G$, qui satisfasse à la propri\'et\'e de cocycle énoncée dans la proposition-définition\ref{ftr}.

\par
Alors il existe un $G$-fibré principal au-dessus de $X$ pour lequel les ouverts de la famille $(V_i)_i$ sont les ouverts de trivialisation.
\end{theo}

\begin{proof}
Méthode brutale : on recolle tous les produits $V_i \times G$ à l'aide des formules de transition, la condition de cocycle assure la cohérence de l'édifice. %
Il s'agit de construire une somme amalgamée de fibrés triviaux, on considère l'espace topologique produit :
\[X \times G \times I\]

où $I$ est muni de sa topologie discrète.

\par
Nous allons travailler dans le sous-ensemble $\bigcup_{i \in I} V_i \times G \times \{i\}$, somme des trivialisations, que nous noterons encore $T$. %
Définissons maintenant, sur $T$, la relation $\sim$ par :
\[(x_j,g_j,j) \sim (x_i,g_i,i) \Leftrightarrow (x_j = x_i) \wedge (g_j = g_{ji}(x_j) g_i\]

Les composantes dans $G$ des triplets de $T$, qui sont les phases pour les trivialisations que nous construisons, %
vérifient ainsi les relations de transition par la famille $(g_{ji})$. %
La condition de cocycle pour la famille $(g_{ji})$ entraîne que $\sim$ est une relation d'équivalence.

\par
On note $\mathcal{Q}$ la projection de $T$ sur son quotient, $P$, par cette relation d'équivalence.

\par
Soit maintenant $\mathcal{P}$ la projection de $P$ sur $X$ définie par :
\[\forall (x,g,i) \in T , \mathcal{P}(\mathcal{Q}((x,g,i))) = x\]

Cette définition est univoque d'après les axiomes que vérifie $\sim$. %
De plus $\sim$ est compatible, dans $T$, avec la multiplication à droite de la deuxième coordonnée par un élément de $G$ : %
$\mathcal{Q}$ permet de définir sur $P$, une action de $G$ à droite de façon cohérente.

\par
Montrons que la projection $\mathcal{Q}$ définit naturellement une trivialisation de $\mathcal{P}$ au-dessus de $X$.

\par
On remarque que, pour tout indice $i$, $\mathcal{Q}$ projette $V_i \times G \times \{i\}$ injectivement dans $P$, %
autrement dit $V_i \times G \times \{i\}$ est un domaine fondamental pour le passage au quotient %
$T \xrightarrow{\mathcal{Q}} P$, on note $\mathcal{V}_i$ son image dans $P$.

\[V_i \times G \times \{i\} \hookrightarrow \mathcal{P}^{-1}(X)\]

Soit de plus $x_i$ un élément de $V_i$, $\mathcal{Q}(x_i,g,k)$ un relevé de $x_i$ dans $P$; $k$ est un indice de $I$ tel que $V_k$ contient $x_i$.

\par
On peut écrire : $(x_i,g,k) \sim (x_i,g_{ik}g,i)$, %
donc tout relevé dans $P$ d'un élément de $V_i$ est dans $\mathcal{V}_i$.

\par
Les deux diagrammes ci-dessous sont équivalents, d'après le raisonnement qui précède celui de gauche résume donc toute la correspondance via $\mathcal{P}$ entre $V_i$ et $P$ :

\[\xymatrix{V_i \times G \times \{i\} \ar[r]^{\mathcal{Q}} & \mathcal{V}_i \ar[d]^{\mathcal{P}} \\%
& V_i} \quad \xymatrix{V_i \times G \times \{i\} & \ar[l]^{\Phi_i} P \ar[d]^{\mathcal{P}} \\  & V_i}\]

celui de droite met en jeu la bijection $\Phi_i$, réciproque de la flèche en haut du premier diagramme, $G$-équivariante à droite, %
dont on montre qu'elle détermine une trivialisation de $P \xrightarrow{\mathcal{P}} X$ au-dessus de $V_i$.

\par
En effet $\Phi_i$ est ouverte car $\mathcal{V}_i$, relevé de $V_i$ par $\mathcal{P}$, est un ouvert de $P$, et car $\mathcal{Q}$ est continue.

\par
De plus, $V_i \times G \times \{i\}$ est un ouvert de $T$, c'est un domaine fondamental pour $\sim$ donc pour tout ouvert $O$ de ce sous-espace de $T$ :

\[\mathcal{Q}(O) = \mathcal{Q}(\overline{O})\]

où $\overline{O}$ est le saturé de $O$ dans $T$, pour la relation d'équivalence $\sim$.

\par
La projection $\mathcal{Q}$ envoie l'ouvert saturé $\overline{O}$ sur un ouvert $\mathcal{O}$ de $P$ inclus dans $\mathcal{V}_i$.

\par
Enfin, $\mathcal{O}$ est ouvert dans $\mathcal{V}_i$ et s'écrit : $\Phi_i^{-1}(O)$. Cette formule en $O$ établit la continuité de $\Phi_i$.

\par
Conclusion : $\Phi_i$ est un homéomorphisme $G$-équivariant à droite, qui réalise une trivialisation de $P \xrightarrow{\mathcal{P}} X$ au-dessus de $V_i$.

\par
Montrons enfin que la famille de trivialisations $(V_i,\Phi_i)_i$ est reliée par les termes de la famille $(g_{ji})$ de fonctions de transition.

\par
Soient $i$ et $j$ deux indices de trivialisation, tels que $V_j$ et $V_i$ soient d'intersection non vide.

\[\xymatrix{(x,\psi_j(p),j) & \ar[l]^{\Phi_j} p \ar[r]^{\Phi_i} & (x,\psi_i(p),i) \\%
T \ar[r]_{\mathcal{Q}} & T/\sim \ar[d]^{\mathcal{P}} & \ar[l]_{\mathcal{Q}} T \\ & V_j \cap V_i & }\]

Selon ce shéma, l'équivalence entre les relevés de $p$ pour la projection $\mathcal{Q}$, via $\Phi_j$ et $\Phi_i$ respectivement, équivaut à :
\[\psi_j(p) = g_{ji}(x) \psi_i(p)\text{, où $x$ est le projeté de $p$ sur $X$ via $\mathcal{P}$}\]

Cette relation établit précisément que $g_{ji}$ est la fonction de transition pour $G \hookrightarrow P \xrightarrow{\mathcal{P}} X$, %
de sa trivialisation $(V_i,\Phi_i)$, vers $(V_j,\Phi_i)$.
\end{proof}

%\section{Fibr\'es et sphères euclidiennes}
\chapter{Cas des sph\`eres %euclidiennes 
$\mathbb{S}^n$}

\emph{%
Dans ce paragraphe, on classifie compl\g{e}tement, pour tout entier positif $n$, les $G-$fibr\'es au-dessus de $\mathbb{S}^n$ %
\g{a} l'aide de la topologie du groupe de structure $G$, connexe par arcs, choisi. %
Le cas o\g{u} $G=\mathbb{U}$ et $n=3$, particuli\g{e}rement \'eclairant, sera ensuite \'etudi\'e de fa\c con d\'etaill\'ee.%
}

%\setcounter{subsection}{-1}

\section{Pr\'eliminaires topologiques : homotopie}

\subsection{Rel\`evement homotopique}

La d\'efinition suivante est issue de \cite{NaberF}.

\begin{defi}[Rel\g{e}vement homotopique sur un fibr\'e localement trivial]
Soit \Fiy un fibr\'e localement trivial. Soit de plus $Z$ un espace topologique.

\par
On dit que \Fiy poss\g{e}de la propri\'et\'e de rel\g{e}vement homotopique relativement \g{a} $Z$ si et seulement si, %
quelles que soient les application continues $f$ et $F$, de $Z$ dans $X$ et de $Z\times [0,1]$ dans $X$ respectivement, v\'erifiant : 

{\em%
\begin{description}
\item[Homotopie :] il existe une homotopie $F$ dans $Z$ \`a valeurs dans $X$, telle que :
\[\forall y \in Z , F(y,0) = f(y)\]
%
\item[Rel\g{e}vement :] il existe une application $\tilde{f}$ de $Z$ dans $P$ telle que le diagramme suivant soit commutatif
\[\xymatrix{ &P \ar[d]^{\mathcal{P}} \\ Z \ar[r]^{f} \ar[ru]^{\tilde{f}} & X}\]
\end{description}
}

$F$ admet un rel\`evement $\tilde{F}$ sur $P$, autrement dit une homotopie $\tilde{F}$ d\'efinie sur $[0,1]^n \times [0,1]$ et \`a valeurs dans $P$, %
telle que $F = \mathcal{P} \circ \tilde{F}$, qui rend commutatif le diagramme :
\[\xymatrix{ &P \ar[d]^{\mathcal{P}} \\ Z \times [0,1] \ar[r]^{F} \ar[ru]^{\tilde{F}} & X}\]
\end{defi}

\begin{exem}
\begin{enumerate}
\item Dans l'annexe \ref{an_y}, on montre -th\'eor\g{e}me \ref{rchefiy}- qu'un fibr\'e \Fiy %
poss\g{e}de la propri\'et\'e de rel\g{e}vement homotopique relativement \g{a} un singleton.
%
\item Plus g\'en\'eralement, on montre dans \cite{NaberF} qu'un fibr\'e localement trivial \Fiy %
poss\g{e}de la propri\'et\'e de rel\g{e}vement homotopique relativement \g{a} tout cube $[0,1]^n$.
\end{enumerate}
\end{exem}

\begin{minipage}{\textwidth}
La d\'emonstration de la deuxi\g{e}me propri\'et\'e, valable pour tout fibr\'e localement trivial \Fiy , %
utilise les propri\'et\'es de l'espace $[0,1]^n$ muni de sa topologie usuelle :

\begin{enumerate}[label=(A\arabic *)]
\item $[0,1]^n$ est compact ;
\item Quel que soit l'ensemble ferm\'e $A$ de $[0,1]^n$, et quelque soit le voisinage ouvert $V$ de $A$, %
il existe un voisinage $W$ et $A$ tel que $\overline{W}\subseteq V$ ;
\item Deux ferm\'es $F_1$ et $F_2$ de $[0,1]^n$ peuvent \^etre s\'epar\'es par une fonction num\'erique autremant dit il existe, %
quels que soient les ferm\'es $F_1$ et $F_2$ une application continue $\mu$ de $[0,1]^n$ dans $[0,1]$, %
\'egale \g{a} $0$ en tout point de $F_1$ et \g{a} $1$ en tout point de $F_2$.
\end{enumerate}

\etoile
\end{minipage}

D'apr\g{e}s \cite{BouTG}, chapitre IX, les deux derniers axiomes sont \'equivalents \g{a} : \og{} $[0,1]^n$ est un espace normal \fg{}. %
D'apr\g{e}s la m\^eme r\'ef\'erence, tout espace topologique compact ou m\'etrisable est normal.

\par
On prouve donc, dans \cite{NaberF} :

\begin{theo}[Condition suffisante pour la propri\'et\'e de rel\g{e}vement homotopique]
Soit \Fiy un fibr\'e localement trivial.

\par
Alors, quel que soit l'espace topologique compact $Z$, \Fiy poss\g{e}de la propri\'et\'e de rel\g{e}vement topologique relativement \g{a} $Z$.
\end{theo}

%\subsubsection{Construction de sections globales}

\subsection{Sph\g{e}res et boules}

\begin{prop}\label{hsd}
Soient $n$ un entier naturel non nul, $Y$ un espace topologique, et $f$ une application continue de $\mathbb{S}^{n-1}$ dans $Y$.

\par
Alors $f$ est homotope à une application constante de $\mathbb{S}^{n-1}$ dans $Y$ si et seulement si %
elle se prolonge continûment en une application continue $\tilde{F}$ du disque unité $D^n$ de l'espace euclidien de dimension $n$, vers $Y$.
\end{prop}

\section{Classification des $G-$fibr\'es au-dessus de $\mathbb{S}^n$}

Dans la sous-section qui suit, $n$ est un entier positif et $G$ un groupe topologique connexe par arcs.

\subsection{Fonction caract\'eristique}
%Soit $n$ un entier naturel non nul.
%
%Venons-en maintenant à la classification des fibrés principaux au-dessus des sphères euclidiennes.\\

On note, dans la suite de ce paragraphe, $n$ un entier naturel non nul, et on se fixe un groupe topologique $G$ connexe par arcs. %
On s'intéresse aux $G$-fibrés principaux au-dessus de $\mathbb{S}^n$, nous allons construire un invariant complet pour ces objets.

\par
Soit $P \xrightarrow{\mathcal{P}} \mathbb{S}^n$ un tel fibré. On définit, pour un réel strictement positif $\varepsilon$ :
\[
V_1^{\varepsilon} := \{(x_i) \in \mathbb{S}^n | x_{n+1} \in ]-\varepsilon , 1]\}%
\]
%
\[
V_2^{\varepsilon} := \{(x_i) \in \mathbb{S}^n | x_{n+1} \in [-1,\varepsilon [ \}%
\]
avec le repérage canonique de $\mathbb{S}^n$ dans $\mathbb{R}^{n+1}$.

\par
Ces ouverts sont deux calottes de la sphère, nord et sud respectivement, %
chacune incluant l'équateur $\mathbb{S}^{n-1}$ et empiétant légèrement sur l'hémisphère opposé. %
En particulier ces deux ouverts s'intersectent sur la bande $\mathcal{B}$ de $\mathbb{S}^n$, %
dont les points sont repérés dans $\mathbb{R}^{n+1}$ avec une dernière coordonnée comprise, strictement, entre $-\varepsilon$ et $\varepsilon$.

\par
On remarque que la projection stéréographique $\varphi_S$ de $\mathbb{S}^n$, depuis son pôle nord, envoie $\mathcal{B}$ dans la couronne :
\[
\left\{ X \in \mathbb{R}^n | \|X\| \in \left] \left(\frac{1-\varepsilon}{1+\varepsilon}\right)^{\frac{1}{2}} , %
\left(\frac{1+\varepsilon}{1-\varepsilon}\right)^{\frac{1}{2}} \right[ \right\}
\]

Par convexité cette couronne se rétracte, de manière homotopique, sur la sphère-unité de $\mathbb{R}^n$, %
il existe donc une rétraction continue $r$ de $\mathcal{B}$ sur l'équateur $\mathbb{S}^{n-1}$ de $\mathbb{S}^n$.

\par
Les adhérences respectives de $V_1^{\varepsilon}$ et $V_2^{\varepsilon}$ sont, via les projections stéréographiques $\varphi_S$ et $\varphi_N$ respectivement, %
homéomorphes à une boule fermée de l'espace euclidien $\mathbb{R}^n$, donc également au cube $[0,1]^n$.

\par
On en déduit, pour les fibrés induits par $G \hookrightarrow P \xrightarrow{\mathcal{P}_{\Box}} \mathbb{S}^n$ sur ces deux fermés, %
l'existence de deux sections globales $\tilde{s_1}$ et $\tilde{s_2}$.

\par
On note encore $\tilde{s_1}$ et $\tilde{s_2}$ les restrictions aux ouverts $V_1^{\varepsilon}$ et $V_2^{\varepsilon}$ de ces sections de fibré. On note :
\[(V_1^{\varepsilon},(\mathcal{P},\tilde{\psi_1}))\text{ et }(V_2^{\varepsilon},(\mathcal{P},\tilde{\psi_2}))\]

les deux trivialisations ainsi définies pour notre fibré, elles recouvrent la base $\mathbb{S}^n$ comme $\varepsilon$ est strictement positif.

\par
Intéressons-nous aux fonctions de transition pour ce fibré, associées à cette famille de trivialisations. On pose par définition :
\[
\forall x \in V_1^{\varepsilon} \cap V_2^{\varepsilon} , \tilde{\psi_2}(\tilde{s_2}(x)) = \tilde{g_{21}}(x) \tilde{\psi_1}(\tilde{s_1}(x))
\]
autrement dit par $G$-équivariance à droite :

\[\forall p \in \mathcal{P}^{-1}(V_1^{\varepsilon} \cap V_2^{\varepsilon}) , \tilde{\psi_2}(p) = \tilde{g_{21}}(x) \tilde{\psi_1}(p)\]
avec la notation habituelle pour $x$ et $p$.

\par
La propriété de cocycle permet de réduire notre connaissance de $(\tilde{g_{11}},\tilde{g_{12}},\tilde{g_{21}},\tilde{g_{22}}$ à la seule connaissance de $\tilde{g_{21}}$.

\par
On normalise maintenant la famille de fonctions de transitions associée aux deux trivialisations qui précèdent, %
au-dessus d'un point $x_0$ choisi arbitrairement sur l'équateur. On pose :
\[\tilde{g_{21}}(x_0) = a\text{ Déphasage : }\psi_1 = a \tilde{\psi_1}\]

Ainsi : $\tilde{\psi_2}(x_0) = \psi_1(x_0)$.

La famille $(g_{ji})$ des fonctions de transition définie par $(\psi_1,\tilde{\psi_2})$, %
d'après l'égalité qui précède et par propriété de cocycle vérifie : $g_{ji}(x_0) = e_G$, pour chacun de ses termes.

\par
On définit maintenant l'application :
\[T : (\mathbb{S}^{n-1},x_0) \rightarrow (G,e_G) : x \mapsto g_{21}(r(x))\]
que nous appellerons fonctions caractéristique pour le fibré.

L'existence de la rétraction $r$ permet de remonter, depuis une fonction continue définie à l'équateur, %
à une application continue de la bande $\mathcal{B}$ qui prend globalement les mêmes valeurs.

\par
D'après le théorème de reconstruction des fibrés principaux, %
on en déduit que toute application $T$ définie comme précédemment est caractéristique, pour un $G$-fibré au-dessus de $\mathbb{S}^n$.

\par
Cette construction, non univoque a priori, va nous permettre de définir un invariant d'équivalence, pour des $G$-fibrés au-dessus de $\mathbb{S}^n$.

\subsection{Un invariant complet pour la classification}

\begin{theo}\label{tinv}
Soient $G \hookrightarrow P_\tc{\Rom{1}} \xrightarrow{\mathcal{P}_{\Rom{1}}} \mathbb{S}^n$ et $G \hookrightarrow P_{\Rom{2}} \xrightarrow{\mathcal{P}_{\Rom{2}}} \mathbb{S}^n$ %
deux fibrés principaux sur un groupe $G$ connexe par arcs. %
On construit comme précédemment deux fonctions caractéristiques $T_{\Rom{1}}$ et $T_{\Rom{2}}$.
\par
Si les deux fibrés sont équivalents alors $T_{\Rom{1}}$ et $T_{\Rom{2}}$ sont homotopes relativement à $\{x_0\}$.
\end{theo}

%\begin{center}
%\emph{Nous avons ainsi construit un invariant d'\'equivalence pour les fibr\'es au-dessus des sph\`eres.}
%\end{center}

\begin{proof}
On adopte les notations usuelles pour les trivialisations des deux fibrés en écrivant leurs indices, $I$ et $II$, en exposant, %
on définit $\lambda_1$, $\lambda_2$ à la manière du paragraphe sur l'équivalence de fibrés, par la correspondance entre les deux familles de trivialisations.

\par
Par définition :
\[\forall x \in \mathcal{B} , g_{21}^{\Rom{2}}(x) = (\lambda_2(x))^{-1} g_{21}^{\Rom{1}}(x) \lambda_1(x)\]
ce qui entraîne :
\[\forall x \in \mathbb{S}^{n-1} , T_{\Rom{2}}(x) = (\lambda_2(x))^{-1} T_{\Rom{1}}(x) \lambda_1(x)\]

\etoile
On veut maintenant se débarasser des facteurs en $\lambda_i$, nous allons pour cela utiliser des homotopies sur $\mathbb{S}^{n-1}$ relativement à $x_0$.

\par
On remarque que $\lambda_1$ et $\lambda_2$ coïncident en $x_0$, nous noterons $a_0$ la valeur commune de ces applications en ce point.

\par
Afin de réaliser une homotopie de $T_{\Rom{2}}$ vers $T_{\Rom{1}}$ relativement à $x_0$ on plonge l'espace d'arrivée $\mathbb{S}^{n-1}$, %
respectivement dans les demi-sphères fermées, supérieure $D_1$ et inférieure $D_2$, de $\mathbb{S}^n$, %
dans lesquelles vivent resectivement $\lambda_1$ et $\lambda_2$.

\par
Chacune de ces demi-sphères fermées, que l'on peut envover sur une partie étoilée de $\mathbb{R}^n$ par projection stéréographique, %
est contractile donc d'après l'annexe \ref{an_y} : l'inclusion de $\mathbb{S}^{n-1}$ dans $D_1$, %
et l'application qui envoie chaque point de l'équateur sur $x_0$, toujours dans cet hémisphère, %
sont homotopes relativement à $x_0$, via une homotopie que nous noterons $H_1$.

\par
$\lambda_1 \circ H_1$ réalise une homotopie, sur $D_1$, entre $\lambda_1$ et l'application constante $\tilde{a_0}$ relativement à $x_0$, %
pointée en $a_0$ dans $G$. Nous notons $K_1$ cette homotopie.

\par
On définit, de même, une homotopie $K_2$ entre $\lambda_2$ et $\tilde{a_0}$ relativement à $x_0$.

\[\mathbb{S}^{n-1} \times [0,1] : (x,t) \mapsto (K_2(x,t))^{-1} T_{\Rom{1}}(x) K_1(x,t)\]
réalise une homotopie de $T_{\Rom{2}}$ vers $a_0^{-1} T_{\Rom{1}} a_0$, relativement à $x_0$.

\par
Enfin par hypothèse, il existe un chemin $\alpha$ de $a_0$ vers $e_G$,
\[(x,t) \mapsto \alpha(t)^{-1} T_{\Rom{1}}(x) \alpha(t)\]
réalise une homotopie de $a_0^{-1} T_{\Rom{1}} a_0$ vers $T_{\Rom{1}}$, relativement à $x_0$.
\end{proof}

\begin{theo}\label{tinvc}
L'invariant d\'efini au th\'eor\g{e}me \ref{tinv} est complet.
\end{theo}

\begin{proof}
On suppose construites, pour deux fibrés $G \hookrightarrow P_{\Rom{1}} \xrightarrow{\mathcal{P}_{\Rom{1}}} \mathbb{S}^n$ %
et $G \hookrightarrow P_{\Rom{2}} \xrightarrow{\mathcal{P}_{\Rom{2}}} \mathbb{S}^n$, %
les deux fonctions caractéristiques $T_{\Rom{1}}$ et $T_{\Rom{2}}$ selon la notation précédente, %
et on émet l'hypothèse que ces deux fonctions sont homotopes, sur $\mathbb{S}^{n-1}$, relativement à $x_0$.

\par
Montrons que nos deux fibrés sont équivalents, en utilisant le critère du théorème \ref{fbl}.

\par
On remarque que l'application $T_{\Rom{1}} T_{\Rom{2}}^{-1}$ de l'équateur $\mathbb{S}^{n-1}$ dans $G$ est homotope à l'application constante sur $\mathbb{S}^{n-1}$, %
de valeur le neutre $e_G$ de $G$ -on ne s'intéresse pas aux points-base $x_0$ et $e_G$.

\par
On montre avec la proposition \ref{hsd} et en composant par la projection stéréographique $\varphi_N$ %
que cette application se prolonge sur $D_1$ en une fonction $\nu$ continue, à valeurs dans $G$.

\par
On construit par recollement, sur $V_1^{\varepsilon}$, une application continue $\lambda_1$ et à valeurs dans $G$, telle que :
\[\forall x \in D_i , \lambda_1(x) = \nu(x)\text{ et }\forall x \in V_1^{\varepsilon} \cap U_2 , \lambda_1 (x) = g^{\Rom{1}}_{21}(x) (g^{\Rom{2}}_{21}(x))^{-1}\]
où $U_2$ est la demi-sphère ouverte sud de $\mathbb{S}^n$.

On munit alors les fibrés des trivialisations :

\[(V_1^{\varepsilon},(\mathcal{P}_{\Rom{1}},\psi_1^{\varepsilon ,\Rom{1}}))\text{ et }(U_2,(\mathcal{P}_1,\psi_2^{\Rom{1}}))\]
pour le premier,

\[(V_1^{\varepsilon},(\mathcal{P}_{\Rom{2}},\psi_1^{\varepsilon ,\Rom{2}}))\text{ et }(V_1^{\varepsilon},(\mathcal{P}_2,\psi_2^{\Rom{2}}))\]
pour le deuxième, où $\psi_2^{\Rom{1}}$ et $\psi_2^{\Rom{2}}$ sont les phases induites sur $U_2$, %
par les trivialisations des deux fibrés au-dessus de $V_2^{\varepsilon}$ que nous supposons construites préalablement à cette démonstration.

\par
On définit, de manière équivalente à ces familles de trivialisations, %
les deux fonctions de transition $g_{21}^{\Rom{1}}$ et $g_{21}^{\Rom{2}}$, sur l'intersection $V_1^{\varepsilon} \cap U_2$, par :

\[\forall p \in \mathcal{P}_{\Rom{1}}^{-1}(V_1^{\varepsilon}) , \psi_2^{\Rom{1}}(p) (\psi_1^{\varepsilon ,\Rom{1}}(p))^{-1} = g_{21}^{\Rom{1}}(\mathcal{P}_{\Rom{1}}(p))\]
et la formule symétrique pour le fibré $\Rom{2}$.

\par
Construisons enfin une équivalence $\tilde{f}$ du fibré $\Rom{1}$ vers le fibré $\Rom{2}$, à l'aide des deux couples de trivialisations locales qui précèdent :
\[\forall p \in \mathcal{P}_{\Rom{1}}^{-1}(V_1^{\varepsilon}) , \psi_1^{\varepsilon ,\Rom{2}}(\tilde{f}(p) = \lambda_1(\mathcal{P}_{\Rom{1}}(p)) \psi_1^{\varepsilon , \Rom{1}}(p)%
\text{ et }%
\forall p \in \mathcal{P}_{\Rom{1}}^{-1}(U_2) , \psi_2^{\Rom{2}}(\tilde{f}(p) = \lambda_1(\mathcal{P}_I(p)) \psi_2^{\Rom{1}}(p)\]

Montrons que cette définition est cohérente : il s'agit de vérifier que les deux formules qui précèdent sont équivalentes, au-dessus de $V_1^{\varepsilon} \cap U_2$.

\par
Par définition de $\tilde{f}$:
\[\forall p \in V_1^{\varepsilon} , \psi_2^{\Rom{2}}(\tilde{f}(p))(\psi_1^{\varepsilon , \Rom{2}}(\tilde{f}(p)))^{-1} = \psi_2^{\Rom{1}}(p) (\psi_1^{\varepsilon ,\Rom{1}}(p))^{-1} (\lambda_1(x))^{-1}\]

On en d\'eduit : $\forall p \in V_1^{\varepsilon} , \psi_2^{\Rom{2}}(\tilde{f}(p))(\psi_1^{\varepsilon ,\Rom{2}}(\tilde{f}(p)))^{-1} = g_{21}^{\Rom{1}}(\mathcal{P}_{\Rom{1}}(p))(\lambda_1(x))^{-1}$, puis :

\[\forall p \in V_1^{\varepsilon} , \psi_2^{\Rom{2}}(\tilde{f}(p))(\psi_1^{\varepsilon ,\Rom{2}}(\tilde{f}(p)))^{-1} = g_{21}^{\Rom{2}}(\mathcal{P}_{\Rom{2}}(\tilde{f}(p)))\]

comme les fibres au-dessus de $\mathcal{P_{\Rom{1}}}$ et $\mathcal{P}_{\Rom{2}}$ se confondent.
\end{proof}

\section{$U(1)\hookrightarrow\mathbb{S}^3\twoheadrightarrow\mathbb{S}^2$ et $SU(2)\hookrightarrow\mathbb{S}^7\twoheadrightarrow\mathbb{S}^4$}

\subsection{Une famille de $\mathbb{U}-$fibr\'es au-dessus de $\mathbb{S}^2$}

\subsubsection{Pr\'eambule : quotients par le groupe $\mathbb{U}_k$ des racines $k-$i\`emes de l'unit\'e}\label{ltk}

\textit{%
Dans cette section, on construit, \`a l'aide de chacune des projections $\mathcal{P}^+$ et $\mathcal{P}^-$ et en quotientant par le groupe $\mathbb{U}_k$, %
une famille de $\mathbb{U}-$ fibr\'es non isomorphes entre eux au-dessus de $\mathbb{S}^2$, index\'ee par l'ensemble des entiers naturels non nuls. %
Ici encore, nous nous cantonnerons \`a la projection $\mathcal{P}^+$ comme point de d\'epart, le cas \og{}n\'egatif\fg{} se traite de la m\^eme mani\`ere.%
}

\emph{Dans cette section et la suivante, $k$ d\'esigne un entier naturel non nul et $\mathbb{U}_k$ le groupe des racines $k-$i\`emes de l'unit\'e dans $\mathbb{C}$.}

\paragraph{Des endomorphismes du groupe topologique $\mathbb{U}$ :}~\\

\par
Voici un petit r\'esultat facile, mais int\'eressant pour la suite.
\par
L'endomorphisme $\omega\mapsto\omega^k$ du groupe topologique $\mathbb{U}$ se factorise canoniquement :
\[\xymatrix{%
\mathbb{U} \ar[r]^{\omega\mapsto\omega^k} \ar[d]_{:\mathbb{U}_k}&\mathbb{U}\\%
\dfrac{\mathbb{U}}{\mathbb{U}_k} \ar[ru]_{h_k}&%
}\]%
Puisque $\mathbb{U}$ est compact, c'est \'egalement le cas du quotient $\frac{\mathbb{U}}{\mathbb{U}_k}$ sur lequel ce groupe topologique se projette. %
$h_k$ est donc une bijection continue et ferm\'ee qui relie $\frac{\mathbb{U}}{\mathbb{U}_k}$ et $\mathbb{U}$, %
image de lui-m\^eme par le morphisme $\omega\mapsto\omega^k$ - pour v\'erifier sa surjectivit\'e et donc celle de $h_k$, il suffit de repr\'esenter un \'el\'ement de l'ensemble image par une exponentielle complexe, %
puis de diviser son argument par $k$.
\par
On peut en conclure que $h_k$ est un isomorphisme de groupes topologiques. En particulier, %
\emph{%
les groupes topologiques $\mathbb{U}$ et $\dfrac{\mathbb{U}}{\mathbb{U}_k}$ sont isomorphes.%
}
\ligneinter
\dots enrouler le groupe $\mathbb{U}$, en $k$ tours, sur lui-m\^eme.

\begin{rema}
On peut utiliser le th\'eor\`eme de rel\`evement et la monog\'en\'eit\'e des sous-groupes ferm\'es de $\mathbb{R}$ pour montrer que les \emph{seuls} %
endomorphismes non triviaux de $\mathbb{U}$ sont de la forme $\omega\mapsto\omega^k$ o\`u $k$ est un entier relatif.
\end{rema}

\paragraph{Cas de la sph\`ere $\mathbb{S}^3$}~\\

\par
L'action, par multiplication scalaire, de $\mathbb{U}$ sur $\mathbb{S}^3$ induit \'evidemment une action du groupe discret $\mathbb{U}_k$ sur cette sph\`ere. %
On peut donc quotienter $\mathbb{S}^3$ par $\mathbb{U}_k$, on note $[\; ]_k$ la projection canonique de la shp\`ere de dimension $3$ sur l'espace $\lt$ ainsi d\'efini, %
par analogie avec $\mathbb{S}^3\xrightarrow{[\; ]}\mathbb{P}^1(\mathbb{C})$.
\par
L'espace topologique, d\'efini \`a hom\'eomorphisme pr\`es par le quotient $\lt$ est un cas d'\emph{espace lenticulaire}, not\'e parfois $\mathcal{L}(k,1)$. %
Par exemple, $SO(3)$ ou $\lt[2]$, est hom\'eomorphe \`a $\mathcal{L}(2,1)$. Sch\'ematiquement, les complexes lenticulaires sont des sortes de \og{}sph\`eres tordues\fg{}.

\subsubsection{Construction d'une famille de $\mathbb{U}-$ fibr\'es au-dessus de $\mathbb{S}^2$}\label{lt1}

\paragraph{Actions de groupes sur le quotient $\lt$ :}~\\

\par
Nous allons construire une \emph{action-quotient}, simple, de $\dfrac{\mathbb{U}}{\mathbb{U}_k}$ sur $\lt$. Soit tout d'abord $\gamma$ l'action de $\mathbb{U}$, %
que nous noterons \`a droite par coh\'erence avec \dots , image de la multiplication scalaire par la projection $[\; ]_k$. %
Cette action poss\`ede un noyau non trivial : $\mathbb{U}_k$.

\par
En effet, soit : $\alpha\in\mathbb{U}_k$. Soit de plus $(z_1,z_2)$ un \'el\'ement de $\mathbb{S}^3$, qui d\'efinit un \'el\'ement $[z_1,z_2]_k$ de $\lt$. %
On note que $[z_1,z_2]_k=\text{Im}((z_1\cdot\omega ,z_2\cdot\omega ))_{\omega\in\mathbb{U}_k}$. %
De m\^eme, $\gamma ([z_1,z_2]_k,\alpha)=\text{Im}((z_1\alpha\cdot\omega ,z_2\alpha\cdot\omega ))_{\omega\in\mathbb{U}_k}$. %
Puisque $\mathbb{U}_k$ est globalement invariant par la translation \`a droite par son \'el\'ement $\alpha$, les deux images \'ecrites pr\'ec\'edemment sont identiques.

\par
R\'eciproquement, soit $\alpha$ un \'el\'ement de $\ker\gamma$. On remarque que : $\gamma([1,1]_k,\alpha)=[1,1]_k$ soit
\[\pl{\alpha}{\alpha}=\pl{1}{1}\]
En particulier : \[\exists\omega\in\mathbb{U}_k|\alpha\cdot\omega=1\]
Il s'ensuit que $\alpha=\omega^{-1}$, donc $\alpha\in\mathbb{U}_k$.
\par
On note $\gk$ l'action-quotient de $\gamma$ par son noyau $\mathbb{U}_k$, elle est simple et v\'erifie :
\[\gk([z_1,z_2]_k,\omega\cdot\mathbb{U}_k)=\gamma([z_1,z_2]_k,\omega)\]
autrement dit
\[\gk([z_1,z_2]_k,\omega\cdot\mathbb{U}_k)=\pl[\omega ']{z_1\omega}{z_2\omega}\]
L'action $\gk$ fait donc commuter le diagramme suivant :
\[%
\xymatrixrowsep{1in}              %%% Attention avec Quickpreview
\xymatrixcolsep{1in}
\xymatrix%
{%
\mathbb{U}\times\mathbb{S}^3\ar[d]_{\left(\cdot\mathbb{U}_k,\cdot\mathbb{U}_k\right)}\ar[r]_{\text{scalaire}}^{\text{multiplication}}&\mathbb{S}^3\ar[d]^{[\; ]}\\%
\dfrac{\mathbb{U}}{\mathbb{U}_k}\times\lt\ar@{-->}[r]_{\gk}&\lt%
}%
\]
%%% Modèle pris sur l'Internet
%\[
%\xymatrixrowsep{1in}
%\xymatrixcolsep{2in}
%\xymatrix
%{
%A\ar[r]\ar[d] & B\ar[d]\\
%C\ar[r] & D
%}
%\]
On en d\'eduit que $\gk\circ \left(\cdot\mathbb{U}_k,\cdot\mathbb{U}_k\right)$ est continue. %
Par ailleurs, $\left(\cdot\mathbb{U}_k,\cdot\mathbb{U}_k\right)$ est une application ouverte de $\mathbb{U}\times\mathbb{S}^3$ vers $\dfrac{\mathbb{U}}{\mathbb{U}_k}$ %
comme les projections canoniques de $\mathbb{U}$ et $\mathbb{S}^3$ sur leurs quotients respectifs par $\mathbb{U}_k$.
\par
Soit maitenant $O$ un ouvert de $\lt$. On note que
\[\left(\gk\circ\left(\cdot\mathbb{U}_k,\cdot\mathbb{U}_k\right)\right)^{-1}(O)=%
\left(\cdot\mathbb{U}_k,\cdot\mathbb{U}_k\right){-1}\left(\gk^{-1}(O)\right)\]
cet ouvert de $\mathbb{U}\times\mathbb{S}^3$ sera not\'e $O'$ .%
La surjectivit\'e de $\left(\cdot\mathbb{U}_k,\cdot\mathbb{U}_k\right)$ permet d'en d\'eduire que
\[\left(\cdot\mathbb{U}_k,\cdot\mathbb{U}_k\right)(O')=\gk^{-1}(O)\]
il s'ensuit que $\gk^{-1}(O)$ est un ouvert de $\dfrac{\mathbb{U}}{\mathbb{U}_k}\times\lt$ car $\left(\cdot\mathbb{U}_k,\cdot\mathbb{U}_k\right)$ est une application ouverte.

\par
Conclusion : $\gk$ est une action simple, \textbf{continue} de $\dfrac{\mathbb{U}}{\mathbb{U}_k}$ sur $\lt$.

\par
On peut enfin construire, \`a l'aide de $\gk$ et du paragraphe pr\'ec\'edent, une action simple $\delta_k$ de $\mathbb{U}$ sur $\lt$ en posant :
\[\forall ([z_1,z_2]_k,\omega)\in\lt\times\mathbb{U},\delta_k([z_1,z_2]_k,\omega)=\gk([z_1,z_2]_k,h_k^{-1}(\omega))\]
On remarque que :
\[\delta_k\left([z_1,z_2]_k,\omega^k\right)=\gamma([z_1,z_2]_k,\omega)\text{ soit }\delta_k\left([z_1,z_2]_k,\omega^k\right)=[z_1\cdot\omega,z_2\cdot\omega]_k\]
quels que soient $(z_1,z_2)$, $\omega$, dans $\mathbb{S}^3$ ou $\mathbb{U}$.

\par
\emph{Nous avons ainsi construit \textbf{deux} actions continues $\gk$ et $\delta_k$, \textbf{simples}, respectivement de $\dfrac{\mathbb{U}}{\mathbb{U}_k}$ et $\mathbb{U}$, %
\textbf{sur} $\mathbf{\lt}$.}

\par
\emph{L'action $\mathbf{\gk}$ agit \`a droite par \textbf{classes \`a gauche} modulo $\mathbb{U}_k$. %
Cela est possible car $\mathbb{U}_k$ est un sous-groupe \textbf{distingu\'e} de $\mathbb{U}$.}

\paragraph{D\'efinition, \`a l'aide de $\mathcal{P}^+$, d'une projection de $\lt$ sur $\mathbb{S}^2$ :}~\\

%\par
On remarque que $\mathbb{U}_k$ est un sous-groupe de $\mathbb{U}$ : les projections $\mathcal{P}^+$ et $\mathcal{P}^-$ se factorisent \`a droite par $[\; ]$ selon les diagrammes suivants :
\[%
\xymatrix{%
\mathbb{S}^3\ar[d]_{[\; ]_k}^{:\mathbb{U}_k}\ar[r]^{\mathcal{P}^+}_{:\mathbb{U}}&\mathbb{S}^2\\%
\lt\ar[ru]_{\mathcal{P}^{+k}}&%
}%
\qquad\text{et}\qquad%
\xymatrix{%
\mathbb{S}^3\ar[d]_{[\; ]_k}^{:\mathbb{U}_k}\ar[r]^{\mathcal{P}^-}_{:\mathbb{U}}&\mathbb{S}^2\\%
\lt\ar[ru]_{\mathcal{P}^{-k}}&%
}%
\]
Nous venons de d\'efinir deux projections, $\mathcal{P}^+$ et $\mathcal{P}^-$, de $\lt$ sur $\mathbb{S}^2$.

\par
Par ailleurs, la d\'efinition de $\gk$ d'une part, et la relation :
\[[z_1,z_2]_k\in\left(\mathcal{P}^{+k}\right)^{-1}(p)\Rightarrow (z_1,z_2)\in\left(\mathcal{P}^+\right)^{-1}(p)\]\label{inc}
valable pour tout point $p$ de $\mathbb{S}^2$ d'autre part, entra\^inent que les fibres de la projection $\mathcal{P}^{+k}$ %
sont exactement les orbites, dans $\lt$, de l'action $\gk$.

\par
La m\^eme chose est vraie, bien s\^ur, pour la projection $\mathcal{P}^{-k}$ ; d'ailleurs les fibres de deux projections sont aussi les orbites de $\lt$ sous l'action $\delta$.

\paragraph{Structures de fibr\'es principaux :}~\\

%\par
Nous allons encore construire une structure de fibr\'e principal, de groupe de structure isomorphe \`a $\mathbb{U}$, au-dessus de $\mathbb{S}^2$. %
On utilise pour cela la construction pr\'ec\'edente, d\'etaill\'ee ici seulement avec $\mathcal{P}^{+k}$ ; %
le cas de $\mathcal{P}^{-k}$ se traite de la m\^eme man\`ere et donne un autre fibr\'e de m\^eme espace de phases et m\^eme groupe de structure.

\par
Il suffit, pour notre construction, d'exhiber deux sections $\se{S}{+k}$ et $\se{N}{+k}$, %
les ouverts de trivialisation $U_S$ et $U_N$ que pour la structude de $\mathbb{U}-$fibr\'e d\'efinie \`a partir de $\mathcal{P}^+$.

\par
On note, tout d'abord, $\mathcal{U}_S^k$ et $\mathcal{U}_N^k$ les images r\'eciproques respectives, par $\mathcal{P}^{+k}$, des ouverts $U_S$ et $U_N$ de $\mathbb {S}^2$. %
%D'apr\`es la remarque pr\'ec\'edente concernant les fibres de $\mathcal{P}^{+k}$ et celles de $\mathcal{P}^+$, on note que $\mathcal{U}_s^k=[\mathcal{U}_S]_k$ et $\mathcal{U}_n^k=[\mathcal{U}_N^k]_k$.
Soit maintenant $\underline{q}\in\mathcal{U}_S$. On note d'apr\`es \ref{inc} que pour tout ant\'ec\'edent $q$ de $\underline{q}$ par la projection $[\; ]_k$, %
$q\in\mathcal{U}_S$. Nous venons de montrer que $\mathcal{U}_S^k\subseteq [\mathcal{U}_S]_k$. R\'eciproquement, soit $q\in\mathcal{U}_S$. %
Par d\'efinition de $\mathcal{P}^{+k}$, $\mathcal{P}^{+k}([q]_k)=\mathcal{P}^+(q)$ donc $\mathcal{P}^{+k}([q]_k)\in U_S$. Ainsi $[\mathcal{U}_S]_k\subseteq\mathcal{U}_S^k$.

\par
On \'etablit de la m\^eme mani\`ere que $\mathcal{U}_N^k=[\mathcal{U}_N]_k$.

\par
On pose maintenant, pour tout poit $p$ de $U_S$ : 
\[%
\left\{\begin{array}{rccl}%
\forall (\phi,\theta )\in U_S&\se{S}{+k}(p)&=&\pl{\cos\frac{\phi}{2}\ec{\theta}}{\sin\frac{\phi}{2}}\\%
\forall (\phi,\theta )\in U_N&\se{N}{+k}(p)&=&\pl{\cos\frac{\phi}{2}}{\sin\frac{\phi}{2}\ec{(-\theta )}}%
\end{array}\right.%
\]
Ainsi, les sections $\se{S}{+k}$ et $\se{N}{+k}$ sont d\'efinies de sorte que les diagrammes :
\[%
\xymatrix{%
\mathcal{U}_S\ar[d]_{[\; ]_k}&U_S\ar[l]_{\se{S}{+}}\ar[ld]^{\se{S}{+k}}\\%
\mathcal{U}_S^k&%
}%
%\qquad%
\text{\hspace{1.5cm}et\hspace{1.5cm}}%\qquad%
\xymatrix{%
\mathcal{U}_N\ar[d]_{[\; ]_k}&U_N\ar[l]_{\se{N}{+}}\ar[ld]^{\se{N}{+k}}\\%
\mathcal{U}_N^k&%
}%
\]
sont commutatifs.

\par
On d\'efinit alors naturellement les fonctions r\'eciproques de trivialisation $\Phi_S^{+k}$ et $\Phi_N^{+k}$. %
Tout d'abord, d'apr\`es l'observation qui pr\'ec\`ede concernant les fibres de $\mathcal{P}^{+k}$, on munit $\lt$ de l'action $\gk$. %
On pose ensuite, pour tout r\'eel $\xi$ :
\[\Phi_S^{+k}\left((\phi,\theta);\ec{\xi}\mathbb{U}_k\right)=\pla{\cos\frac{\phi}{2}\ec{\theta+\xi}}{\sin\frac{\phi}{2}\ec{\xi}}\]
ce qui s'\'ecrit encore :
\[\Phi_S^{+k}\left((\phi,\theta);\ec{\xi}\mathbb{U}_k\right)=\left[\st{\phi}{\theta+\xi}{\xi}\right]_k\]
soit :
\[\Phi_S^{+k}\left((\phi,\theta),\ec{\xi}\mathbb{U}_k\right)=\left[\Phi_S^+\left((\phi,\theta),\ec{\xi}\right)\right]_k\]
On remarque que $\se{S}{+k}=[\; ]_k\circ\se{S}{+}$ donc la section $\se{S}{+k}$ est continue. %
La continuit\'e de $\gk$ permet d'en d\'eduire que $\Phi_S^{+k}$ est continue ; le cas de $\Phi_N^{+k}$ se traite de la m\^eme mani\`ere.

\par
Il reste \`a prouver la continuit\'e des fonctions de phase $\psi_S^{+k}$ et $\psi_N^{+k}$, %
qui vont de $\mathcal{U}_S^k$ et $\mathcal{U}_N^k$ respectivement, vers $\dfrac{\mathbb{U}}{\mathbb{U}_k}$.

\par
On remarque que : \[\forall (z_1,z_2)\in\mathbb{S}^3 , %
\left\{\begin{array}{lcr}%
\psi_S^{+k}\left(\left[z_1,z_2\right]_k\right)&=&\psi_S^+(z_1,z_2)\cdot\mathbb{U}_k\\%
\psi_N^{+k}\left(\left[z_1,z_2\right]_k\right)&=&\psi_N^+(z_1,z_2)\cdot\mathbb{U}_k%
\end{array}\right\}\]
Autrement dit les deux diagrammes :
\[%
\xymatrix{\mathcal{U}_S\ar[d]_{[\; ]_k}\ar[r]^{\psi_S^+}&\mathbb{U}\ar[d]^{\cdot\mathbb{U}_k}\\%
\mathcal{U}_S^k\ar[r]_{\psi_S^{+k}}&\dfrac{\mathbb{U}}{\mathbb{U}_k}%
}%
\text{ et }%
\xymatrix{\mathcal{U}_N\ar[d]_{[\; ]_k}\ar[r]^{\psi_N^+}&\mathbb{U}\ar[d]^{\cdot\mathbb{U}_k}\\%
\mathcal{U}_N^k\ar[r]_{\psi_N^{+k}}&\dfrac{\mathbb{U}}{\mathbb{U}_k}%
}%
\]
sont commutatifs.

\par
Or les applications $\psi_S^+$ et $\psi_N^+$ sont continues d'apr\`es la sous-section \ref{fis}, %
de m\^eme que la projection canonique de $\mathbb{U}$ sur $\dfrac{\mathbb{U}}{\mathbb{U}_k}$ ; %
$[\; ]_k$ est une application ouverte.
\par
Le raisonnement utilis\'e pour d\'emontrer la continuit\'e de l'action $\gk$ permet de conculre que les fonctions $\psi_S^{+k}$ et $\psi_N^{+k}$ sont continues.
\etoile
\emph{Nous avons construit un fibr\'e principal de base $\mathbb{S}^2$ et de groupe de structure $\dfrac{\mathbb{U}}{\mathbb{U}_k}$, %
\`a l'aide des deux ouverts de trivialisation $U_S$ et $U_N$, l'espace de phases est $\lt$, muni de l'action $\gk$}.

\par
Pour construire un $\mathbb{U}-$fibr\'e \`a l'aide de la structure qui pr\'ed\`ede en utilisant l'identification entre $\dfrac{\mathbb{U}}{\mathbb{U}_k}$ et $\mathbb{U}$ \'etabli au d\'ebut de la section, %
on consid\`ere l'action $\delta_k$ de $\mathbb{U}$ sur $\lt$ et on compose les fonctions de phase $\psi_S^{+k}$ et $\psi_N^{+k}$ \`a gauche par $h_k$ ; %
on note $\left(\lt ,\mathbb{S}^2,\mathcal{P}^{+k},\mathbb{U}\right)$ le fibr\'e ainsi d\'efini.

\ligneinter
Bien s\^ur, la construction qui pr\'ec\`ede est possible \`a partir de la projection $\mathcal{P}^-$. %
On note que $\se{S}{-k}(\phi,\theta)=\pla{\cos\frac{\phi}{2}\ec{(-\theta)}}{\sin\frac{\phi}{2}}$ %
et $\se{N}{-k}(\phi,\theta)=\pla{\cos\frac{\phi}{2}}{\sin\frac{\phi}{2}\ec{\theta}}$.

\par
\emph{Il existe ainsi une famille $\left(\lt[\abs{k}],\mathbb{S}^2,\mathcal{P}^k,\mathbb{U}\right)_k$, index\'ee par $\mathbb{Z}\setminus\{0\}$, de fibr\'es principaux de base $\mathbb{S}^2$ et de groupe de structure $\mathbb{U}$. %
De plus, nous allons voir que ces fibr\'es sont \emph{non isomorphes} deux \`a deux.} -on identifie $\mathbb{S}^3$ \`a $\lt[1]$.

\paragraph{Fonctions de transition pour les fibr\'es $\left(\lt[\abs{k}],\mathbb{S}^2,\mathcal{P}^k,\mathbb{U}\right)_k$ : actions $\gk$ et $\delta_k$.}~\\

%\par
Les formules qui pr\'ec\g{e}dent concernant $\se{S}{+k}$, $\se{N}{+k}$, $\se{S}{-k}$ et $\se{N}{-k}$ nous donnent, avec la notation idoine pour $g_{SN}^{+k}$ et $g_{SN}^{-k}$ :
\[g_{SN}^{+k}(\phi ,\theta)=\ec{\theta}\cdot\mathbb{U}_k\text{ et }g_{SN}^{-k}(\phi ,\theta)=\ec{(-\theta)}\cdot\mathbb{U}_k\]
quels que soient les r\'eels $\phi$ et $\theta$ tels que : $\phi\in ]0,\pi[$.

\ligneinter
Les fonctions de transition $g_{SN}^{k,\delta_{\abs{k}}}$ des fibr\'es de la famille $\left(\lt[\abs{k}],\mathbb{S}^2,\mathcal{P}^k,\mathbb{U}\right)_{k\in\mathbb{Z}\setminus\{0\}}$ %
s'en d\'eduisent via le morphisme de groupes topologiques $h_k$ :
\[\boxed{\forall (\phi,\theta)\in ]0,\pi [,g_{SN}^{k,\delta_{\abs{k}}}(\phi,\theta)=\ec{k\theta}}\]

\begin{rema}
Les espaces topologiques de la famille $(\mathcal{L}(k,1))_{k\in\mathbb{N}^{\ast}}$, d\'efinis \g{a} hom\'eomorphisme pr\g{e}s, %
sont \'etudi\'es dans \cite{Lens}.
\end{rema}

\subsubsection{Rev\^etements et espaces lenticulaires :}

\subsection{$SU(2)-$fibr\'es au-dessus de $\mathbb{S}^4$}

%Document \'eponyme.


%\chapter{Connexions}

%\chapter{physique}

\appendix
%\section{Fibr\'es localement triviaux et homotopie}
%\newpage
\chapter{Pr\'erequis en math\'ematiques et en physique}

\section{Un peu de topologie}\label{an_y}

\subsection{Fibr\'es localement triviaux, rev\^etements}

\begin{defi}
Soient $X$ et $Y$ deux espaces topologiques s\'epar\'es.

\par
Un fibr\'e localement trivial sur $X$, de fibre $Y$, est d\'efini par :
\begin{itemize}
\item un espace topologique $P$, appel\'e espace total,% et une action \`a droite, continue, de $G$ sur $P$,
\item une surjection continue $\mathcal{P}$ de $P$ sur $X$, %$G$-invariante autrement dit : $\forall (p,g) \in P \times G , \mathcal{P} (p \ast g) = \mathcal{P} (p)$,
\end{itemize}
tels que pour tout point $x$ de la base $X$ %
il existe un voisinage ouvert de $x$ et un hom\'eomorphisme $\Phi$ de $V\times Y$ sur $\mathcal{P}^{-1}(V)$ v\'erifiant :
%soit muni d'un voisinage ouvert $V$, associé à un homéomorphisme $\Phi$ de $\mathcal{P}^{-1}(V)$ sur $V \times G$ tel que:
\[\forall (v,y)\in v\times Y , \mathcal{P}(\Phi (v,y))=v\]

Un tel couple $(V, \Phi)$ est appel\'e trivialisation locale du fibr\'e -botte de fibres. %Nous étudierons explicitement le terme de phase $\psi$ dans les problÚmes de physique que nous rencontrerons.\\
On sh\'ematise par $Y \hookrightarrow P \overset{\mathcal{P}}{\twoheadrightarrow} X$ le fibr\'e ainsi d\'efini, %
que nous noterons formellement \Fiy.

\par
On adoptera souvent la notation $\Phi$, $\Psi$ pour l'homorphisme de trivialisation et sa r\'eciproque.
\end{defi}

\begin{exem}[Rev\^etements]
Un rev\^etement est d\'efini par une projection $\mathcal{P}$, continue, d'un espace connexe $\tilde{X}$ sur un espace topologique $X$, %
telle que tout point $x$ de l'espace $X$ admet un voisinage $U_x$ v\'erifiant :
\[\mathcal{P}^{-1}(\{U_x\})=\underset{i}{\bigcup}U_x^i\]
o\`u $(U_x^i)_i$ est une famille, disjointe, d'ouverts de $\tilde{X}$ telle que pour tout $i$, $\mathcal{P}$ induit un hom\'eomorphisme de $U_x^i$ sur $U_x$.

On peut montrer qu'un rev\^etement d\'efinit un fibr\'e localement trivial, dont la fibre est discr\`ete.
\end{exem}

\subsection{Rel\`evement d'applications}

\begin{comment}
Il est ais\'e de montrer que, lorsque \Fiy est un fibr\'e localement trivial, tel que $X$ et $Y$ sont connexes, alors l'espace total $P$ est connnexe. %
La d\'emonstration de la m\^eme propri\'et\'e concernant la connexit\'e par arcs est la premi\`ere occasion de relever une application continue d'un espace connexe, %
\`a valeurs dans $X$, %
par une fonction \`a valeurs dans $P$.
\end{comment}

\begin{theo}[Rel\g{e}vement de chemins]\label{rchefiy}
Soient \Fiy un fibr\'e localement trivial de fibre $Y$ %connexe par arcs,
et $\alpha$ un chemin de $X$, %
c'est- \`a-dire une application continue de $[0,1]$ dans $X$. %
Soit de plus $\tilde{x_0}$ un \'el\'ement de la fibre de $\alpha (0)$.

Alors il existe un chemin $\tilde{\alpha}$ de $P$ tel que :
\begin{itemize}
\item $\tilde{\alpha}(0)=\tilde{x_0}$
\item $\forall t \in [0,1] , \mathcal{P}(\tilde{\alpha}(t))=\alpha (t)$
\end{itemize}
\end{theo}

\begin{proof}
Voir \cite{NaberF}
\end{proof}

\begin{coro}
Si \Fiy est un fibr\'e localement trivial et si $X$ et $Y$ sont connexes par arcs, alors $P$ est connexe par arcs.
\end{coro}

Si \Fiy est un rev\^etement, on d\'emontre que ce rel\`evement est unique. On peut m\^eme facilement d\'emontrer un peu plus :

\begin{prop}\label{urc}
Soient $\tilde{X} \overset{\mathcal{P}}{\longrightarrow} X$ un rev\^etement, $Z$ un espace topologique connexe, %
$z_0$, $x_0$ et $\tilde{x_0}$ des \'el\'ements de $Z$, $X$ et $\tilde{X}$ respectivement.

\par
On suppose de plus que : $\tilde{x_0}\in\mathcal{P}^{-1}(\{x_0\})$.

\par
Si $f$ est une application continue de $Z$ dans $X$, qui envoie $z_0$ sur $x_0$, et si $f$ admet un rel\`evement vers $P$ qui vaut $\tilde{x_0}$ en $z_0$, %
alors ce rel\`evement est unique.
\end{prop}

\begin{rema}
Une application continue $f$ entre deux espaces topologiques $X$ et $Y$, telle que $f(x_0)=y_0$ %
pour deux \'el\'ements de $X$ et $Y$ respectivement, est appel\'ee application continue entre les espaces point\'es $(X,x_0)$ et $(Y,y_0)$.

\par
On peut reformuler la proposition ci-dessus : %
soient $\tilde{X}\overset{\mathcal{P}}{\longrightarrow} X$ un rev\^etement, $Z$ un espace topologique connexe, %
$z_0$, $x_0$ et $\tilde{x_0}$ des \'el\'ements de $Z$, $X$ et $\tilde{X}$ respectivement.

\par
Si $f$ est une application continue entre les espaces point\'es $(Z,z_0)$ et $(X,x_0)$ qui admet un rel\`evement $\tilde{f}$, %
d\'efini entre les espaces point\'es $(\tilde{X},\tilde{x_0})$, alors ce rel\`evement est unique.
\end{rema}

\begin{proof}
Supposons l'existence de deux rel\g{e}vements $\tilde{f}_1$ et $\tilde{f}_2$ de $f$ qui satisfont les propri\'et\'es \'enonc\'ees. Soit :

\[F:=\left\{z\in Z|\tilde{f}_1(z)=\tilde{f}_2(z)\right\}\]
Comme $\tilde{X}$ est s\'epar\'e cet ensemble est un ferm\'e de $Z$. Montrons qu'il est aussi ouvert.

\par
Soit : $z\in F$. Il existe un ouvert $U$ de $X$ contenant $z$ et une famille disjointe $(\mathcal{U}_i)_i$ d'ouverts de $\tilde{X}$ tels que :

\begin{itemize}
\item $\mathcal{P}$ induit un hom\'eomorphisme de $\mathcal{U}_i$ sur pour tout $i$ ;
\item $\mathcal{P}^{-1}(U)=\bigsqcup\limits_i \mathcal{U}_i$.
\end{itemize}

Par hypoth\g{e}se sur $z$, $\tilde{f}_1(z)=\tilde{f}_2(z)$, on note $i_0$ l'unique indice pour $(\mathcal{U}_i)_i$ tel que $\tilde{f}_1(z)\in\mathcal{U}_{i_0}$.

\par
On note que les images r\'eciproques de $\mathcal{U}_{i_0}$ par $\tilde{f}_1$ et $\tilde{f}_2$ sont ouvertes dans $Z$.

\par
Soit :

\[O_z=\tilde{f}_1^{-1}(\mathcal{U}_{i_0})\cap \tilde{f}_2^{-1}(\mathcal{U}_{i_0})\]
Cet ensemble est aussi un ouvert de $Z$.

Soit enfin : $a\in O_z$. On peut \'ecrire :

\[%
\begin{array}{ccc}
f(a)&\in&U\\
(\tilde{f}_1(a),\tilde{f}_2(a))&\in&\mathcal{U}_{i_0}\times\mathcal{U}_{i_0}
\end{array}
\]

Il s'ensuit que : $\tilde{f}_1(a)=\tilde{f}_2(a)$. On en d\'eduit : $O_z\subset F$, cela prouve que $F$ est un ouvert.

\par
$F$ est donc un ouvert et ferm\'e de l'espace connexe $Z$, non vide puisqu'il contient $z_0$. Cet ensemble est donc \'egal \g{a} $Z$, ce qui nous donne l'unicit\'e recherch\'ee.
\end{proof}

\begin{comment}
L'existence s'\'etablit avec le lemme de Zorn : FAUX, il faut une hypothèse additionnelle biscornue,pour avoir desintersections connexes d'ouverts de relèvement.
C-exemple : on ne peut pas relever sur \mathbb{R} l'application identti\'e de \mathbb{U} dans lui-m\^eme, via l'exponentielle complexe, après un choix d'image pour un point de  cette sphère.

\begin{prop}
Avec les hypoth\`eses de la proposition qui pr\'ec\`ede concernant $f$, $z_0$ et $x_0$, il existe (au moins) une application continue $\tilde{f}$ de $Z$ vers $P$, %
qui vaut $\tilde{x_0}$ en $z_0$, qui rel\`eve $f$ au sens o\`u :
\[\forall z\in Z, \mathcal{P}(\tilde{f}(z))=f(z)\]
\end{prop}

\begin{proof}[\es]
Rel\g{e}vements locaux, un rel\g{e}vement d\'efini sur un ouvert strictement inclus dans $Z$ n'est pas maximal.
\end{proof}

%\begin{rema}
%Dans la d\'emonstration qui concerne un chemin de la base d'un fibr\'e, %
%de m\^eme que dans une autre d\'emonstration que nous verrons pour le rel\`evement d'une homotopie dans un rev\^etement, %
%le recours au lemme de Zorn est \'evit\'e gr\^ace \`a un argument de compacit\'e.
%\end{rema}
\end{comment}

La proposition suivante sera utile pour la sous-sous section \ref{gs1}.% Un argument de compacit\'e permet d'\'eviter, dans sa d\'emonstration, le recours au lemme de Zorn.

\begin{prop}\label{red2}
Soient $\tilde{X}\overset{\mathcal{P}}{\longrightarrow} X$ un rev\^etement, tel que $\mathcal{P}(\tilde{x_0})=x_0$ pour deux \'el\'ements de $\tilde{X}$ et $X$. %
Soit de plus $F$ une application continue entre les espaces point\'es $([0,1]^2,(0,0))$ et $(X,x_0)$.

\par
Alors il existe un rel\g{e}vement $\tilde{F}$ de $F$ \g{a} valeurs dans $\tilde{X}$, tel que : $\tilde{F}((0,0))=\tilde{x_0}$.
\end{prop}

\begin{proof}[\re]
\end{proof}

\begin{rema}
\begin{enumerate}
\item L'argument qui pr\'ec\g{e}de, o\g{u} le recours au lemme de Zorn est \'evit\'e \g{a} l'aide de la construction de $(V_n)_{n\in [\![1,N]\!]}$, %
est en fait valable si l'on prend n'importe quel espace topologique connexe et compact $Z$ au lieu de $[0,1]^2$.
\item D'apr\g{e}s la proposition \ref{urc}, le rel\g{e}vement ainsi construit est unique.
\end{enumerate}
\end{rema}

\bigskip
Mentionnons mainenant une cons\'equence connue %de la proposition 
du th\'eor\g{e}me \ref{rchefiy} :

\begin{prop}[Th\'eor\`eme de rel\`evement, exponentielle complexe]
Soit $f$ une application continue, d\'efinie sur un un intervalle r\'eel $I$, qui prend ses valeurs dans $\mathbb{U}$. %
Soient \'egalement $t_0$ un \'el\'ement de $I$, et $\theta_{t_0}$ un argument de $f(t_0)$.

\par
Il existe au moins une application continue $\theta$ de $I$ dans $\mathbb{R}$, telle que :
\[\forall t \in I, f(t)=\exp (i\theta (t))\]
\end{prop}

Pour m\'emoire, l'exponentielle complexe $\theta \mapsto \exp (i\theta )$ d\'efinit un rev\^etement de $\mathbb{U}$ par $\mathbb{R}$, %
la fibre est ici hom'eoorphe \`a $\mathbb{Z}$. C'est m\^eme un fibr\'e principal, de groupe de structure $2\pi\mathbb{Z}$.

\begin{proof}
Nous connaissons d\'ej\`a le cas o\`u $I$ est un segment.

\par
Supposons : $I=[a,b[$, o\`u $a$ est un nombre r\'eel, $b$ un nombre sup\'erieur \`a $a$, ou $+\infty$. Soit $\theta_a$ un argument de $f(a)$

\par
Soit de plus $(b_n)_{n\in\mathbb{N}^{\ast}}$ la suite r\'eelle, strictement croissante, d\'efinie pour tout $n$ strictement positif par :
\begin{itemize}
\item $b_n=b-\frac{b-a}{n+1}$ si $b$ est fini,
\item $b_n=a+n$ si $b$ est infini.
\end{itemize}
Dans les deux cas : $[a,b[=\underset{n\in\mathbb{N}^{\ast}}{\bigcup}[a,b_n[$.

\par
Il existe un rel\`evement $\tilde{f}_1$ de $f$ sur $[a,b_1]$,  tel que : $\tilde{f}_1(a)=\theta_a$.

\par
Supposons construit, pour un entier naturel $n$ non nul, un rel\`evement $\tilde{f}_n$ de $f$, d\'efini sur $[a,b_n]$, tel que : $\tilde{f}_n(a)=\theta_a$. %
Toujours d'apr\`es la proposition \ref{rchefiy}, %
il existe une application $\tilde{f}_{b{n+1}}$ de $[b_n,b_{n+1}]$ dans $\mathbb{R}$, continue, telle que :
\[\forall t \in [b_n,b_{n+1}], \exp (i\tilde{f}_{b_{n+1}}(t))=f(t) \text{ et }\tilde{f}_{b_{n+1}}(b_n)=\tilde{f}_n(b_n)\]
On peut ainsi d\'efinir, par recollement sur $[a,b_{n+1}]$, un rel\`evement continu $\tilde{f}_{n+1}$ de $f$, qui vaut $\theta_a$ en $a$.

\par
Nous avons ainsi construit, par r\'ecurrence, une suite $(\tilde{f}_n)$ de rel\`evements de $f$ qui valent $\theta_a$ en $a$, et qui v\'erifie de plus :
\[\forall n \in\mathbb{N}^{\ast}, \forall t \in [a,b-n], \tilde{f}_n(t)=\tilde{f}_{n+1}(t)\]

Cette propri\'et\'e de coh\'erence se g\'en\'eralise ais\'ement entre deux fonctions $\tilde{f}_m$ et $\tilde{f}_n$, %
d\`es que $m$ est inf\'erieur \`a $n$, sur le domaine de d\'efinition de $\tilde{f}_m$.

\par
On peut donc d\'efinir, sur l'intervalle $[a,b[$, le prolongement commun $\tilde{f}$ de tous les termes de $(\tilde{f}_n)_n$, %
ce qui nous donne le rel\`evement souhait\'e pour $f$.

\par
Un raisonnement similaire permet de r\'esoudre le cas d'un intervale ferm\'e \`a gauche, et ouvert \`a droite; %
ceci permet enfin de r\'egler le cas d'un intervalle ouvert de $\mathbb{R}$.
\end{proof}

\subsection{Homotopie}

Voici quelques g\'en\'eralit\'es qui concernent l'homotopie entre des fonctions continues, la d\'efinition qui suit est utile dans toutet la suite du paragraphe.

\begin{defi}
Soient $X$ et $Y$ deux espaces topologiques, soit de plus $A$ une partie de $X$, \'eventuellement vide.

Une \textbf{homotopie}, relativement \`a $A$, entre deux applications coninues $f$ et $g$ de $X$ vers $Y$, %
qui co\"incident sur $A$, %
est une application $H$ de $X\times [0,1]$ dans $Y$, continue, telle que :
\begin{itemize}
\item $\forall x \in X, H(x,0)=f(x) \wedge H(x,1)=g(x)$
\item $\forall t \in [0,1] , \forall a \in A , H(a,t)=f(a)$
\end{itemize}
Si une telle homotopie existe, on dit que $f$ et $g$ sont homotopes relativement \`a $A$, %
et on note : $f\simeq g \text{ rel } A$.
\end{defi}

\begin{prop}
Avec la notation qui pr\'ec\`ede, l'homotopie relativement \`a $A$ d\'efinit une relation d'\'equivalence entre les applications continues de $X$ vers $Y$.
\end{prop}

Les deux exemples qui suivent sont importants :

\begin{exem}[Homotopie libre]
Lorsque $A$ est vide, l'homotopie est dite \textbf{libre}, on dit simplement que $f$ et $g$ sont homotopes et on note : $f\simeq g$.

L'ensemble des clases d'\'equivalence d'applications continues entre $X$ et $Y$ est not\'e $[X,Y]$.
\end{exem}

\begin{exem}[Homotopie de chemins]
Cette fois-ci $X$ est le segment $[0,1]$, $\alpha$ et $\alpha '$ sont deux chemins de $Y$.
%$A$ est choisi comme \'etant l'ensemble $\{0,1\}$.\\

Sauf mention explicite du contraire, une homotopie entre $\alpha$ et $\alpha '$ d\'esigne toujours une homotopie, relativement \`a $\{0,1\}$, %
entre les applications continues $\alpha$ et $\alpha '$ de $[0,1]$ vers $Y$. On note encore $\alpha\simeq\alpha ' \text{ rel }\{0,1\}$.

Pour qu'une telle homotopie existe, il faut en particulier que $\alpha(0)=\alpha '(0)$ et $\alpha(1)=\alpha '(1)$.
\end{exem}

\begin{defi}
Une application continue entre deux espaces topologiques $X$ et $Y$, qui est homotope \`a une application constante de $X$ vers $Y$, %
est dite homotope \`a z\'ero.
\end{defi}

Les espaces qui suivent sont courants en topologie alg\'ebrique, nous les retrouvons dans la classification des fibr\'e principaux.

\begin{prefi}[Espace contractile]
Un espace topologique $Y$ est dit contractile lorqu'il v\'erifie l'une des deux caract\'erisations, \'equivalentes, qui suivent :
\begin{enumerate}[label=(C\arabic *)]
\item[C1] $Id_Y$ est homotope \`a z\'ero ;
\item[C2] deux applications continues d'un espace topologique $X$ dans $Y$ sont toujours homotopes entre elles.
\end{enumerate}
\end{prefi}

La deuxi\`eme caract\'erisation entra\^ine notamment qu'un espace contractile est connexe par arcs.

\par
L'homotopie est \'egalement une relation entre les espaces topologiques :

\begin{defi}
Une application continue $h$ entre deux espaces topologiques $X$ et $Y$ est appel\'ee \'equivalence homotopique si et seulement si %
il existe une application continue $h'$ de $Y$ vers $X$ telle que :
\begin{itemize}
\item $h'\circ h\simeq Id_X$ ;
\item $h\circ h'\simeq Id_Y$.
\end{itemize}
La relation ainsi d\'efinie sur tout ensemble d'espaces topologiques, appel\'ee \'equivalence homotopique, est une relation d'\'equivalence.
\end{defi}

\begin{exem}
Un hom\'eomorphisme entre deux espaces topologiques $X$ et $Y$ est une \'equivalence homotopique entre $X$ et $Y$.
\end{exem}

Bien s\^ur, deux espaces topologiques homotopiquement \'equivalents ne sont pas n\'ecessairement hom\'eomorphes, c'est l'int\'er\^et de la d\'efinition qui pr\'ec\`ede. %
Voici un exemple important qui illustre ce fait :

\begin{prop}
Un espace contractile est homotopiquement \'equivalent \`a un singleton, muni de sa -seule- topologie discr\`ete.
\end{prop}

\subsection{Le groupe fondamental}

Dans ce paragraphe, on \'etablit des propri\'et\'es alg\'ebriques sur les lacets dans des espaces topologiques. La structure de groupe fondamental sur un espace point\'e est particuli\g{e}remant int\'eressante.

\begin{defi}
Soit $\alpha$ un chemin qui prend ses valeurs dans un espace topologique $X$. $\alpha '(0)$ et $\alpha (1)$ sont respectivement appel\'es origine et extr\'emit\'e.
\end{defi}

\begin{exem}
Soient $\alpha$ et $\beta$ deux chemins homotopes relativement \g{a} $\{0,1\}$, via une homotopie de chemins $H$ \g{a} valeurs dans un espace topologique $X$.

\par
Alors les chemins interm\'ediaires $(s\mapsto H(s,t)_t$ ont tous les m\^emes origines et extr\'emit\'es.
\end{exem}

Lorsque $\alpha$ et $\beta$ sont deux chemins tels que $\alpha (1)=\beta (0)$, on peut les composer, par concat\'enation : le chemin $\alpha \beta$ est d\'efini par les formules
\[\forall s\in \left[0,\frac{1}{2}\right] , \alpha \beta (s) = \alpha (2s)\]% \text{ et } 
et
\[\forall s \in \left[\frac{1}{2} , 1\right] , \alpha \beta (s) = \beta (2s-1)\]

On peut aussi d\'efinir un inverse $\alpha^{\leftarrow}$ pour un chemin $\alpha$ :
\[\forall s \in [0,1] , \alpha^{\leftarrow} (s) = \alpha (1-s)\]

Les cons\'equences en termes d'homotopie de ces deux d\'efinitions sont illustr\'ees dans la proposition suivante :

\begin{prop}
Soient $\alpha$ et $\beta$ deux chemins sunr un espace topologique $X$. On suppose $\alpha (1)$ \'egal \g{a} $\beta (0)$.
\begin{itemize}
\item Si $\alpha '$ et $\beta '$ sont deux chemins %
respectivement homotopes $\alpha$ et $\beta$, alors $\alpha \beta$ est homotope \g{a} $\alpha '\beta '$.
\item Si $\alpha '$ est un chemin homotope \g{a} $\alpha$, alors $\alpha '^{\leftarrow}$ est homotope \g{a} $\alpha^{\leftarrow}$.
\end{itemize}
\end{prop}

Pour tout chemin $\alpha$ d'un espace topologique $X$, %
on note $[\alpha ]$ sa classe d'homotopie relativement \g{a} $\{0,1\}$.

\par
La proposition qui pr\'ec\g{e}de permet de d\'efinir une loi de composition entre les classes d'homotpie de chemins. %
Le produit $[\alpha ][\alpha^{\leftarrow}]$ est ainsi la classe d'homotopie du chemin constant $\tilde{\alpha (0)}$.

\par
Les lacets sont des chemins qui b\'en\'eficient de propri\'et\'es alg\'ebriques encore plus riches.

\begin{defi}[Lacets]
Soit $(X,x_0)$ un espace topologique point\'e.

\par
Un chemin de $X$, dont l'origine et l'extr\'emit\'e se situent en $x_0$ est appel\'e lacet, ou boucle, en $x_0$.
\end{defi}

\begin{rema}
Avec la notation qui pr\'ec\g{e}de, les chemins interm\'ediaires r\'ealis\'es par une homotopie entre $\alpha$ et un autre lacet $\alpha '$ %
sont encore des lacets en $x_0$. C'est en particulier le cas de $\alpha '$.
\end{rema}

\begin{prefi}[Groupe fondamental]
Soit $(X,x_0)$ un espace topologique point\'e.

\par
Alors l'ensemble $[([0,1],\{0,1\});(X,x_0)]$ des classes d'homotopie de lacets de $X$ en $x_0$, %
muni de la loi de composition des lacets -concat\'enation-, est un groupe, appel\'e %
\textbf{groupe fondamental de l'espace $X$ en $x_0$.} Son \'el\'ement neutre est le lacet constant dont la valeur est $x_0$.

\par
On le note $\pi_1(X,x_0)$
\end{prefi}

\es Il suffit maintenant d'\'etablir l'associativit\'e de la loi de concat\'enation, \'etendue aux classes d'homotopie.

\begin{rema}
La classe $[s\mapsto x_0]$ est l'ensemble des lacets de $X$ en $x_0$, homotopes \g{a} z\'ero.
\end{rema}

\begin{exem}
Soit un espace topologique contenant un seul point $x_0$.

\par
Alors le groupe fondamental $\pi_1(\{x_0\},x_0)$ est trivial, %
il admet comme seul \'el\'ement la classe d'hmotopie $[s\mapsto x_0]$, elle-m\^eme constitu\'ee du seul lacet constant de $\{x_0\}$.
\end{exem}

Nous allons voir d'autres exemples de groupes fondamentaux; d'une fa\c con g\'en\'erale, un groupe fondamental est difficle \g{a} calculer, surtout si ce groupe est non trivial.

Avec la notation qui pr\'ec\g{e}de, si $x_0$ est un autre \'el\'ement de $X$ reli\'e, dans $X$, \g{a} $x_0$ par un chemin $\sigma$, %
alors l'application $[\alpha]\mapsto [\sigma^{-1}][\alpha][\sigma]$ r\'ealise un isomorphisme entre les groupes $\pi_1(X,x_0)$ et $\pi_1(X,x_1)$.

\par
Ainsi, lorsque $X$ est connexe par arcs, tous les groupes fondamentaux d\'efinis sur $X$ sont isomorphes entre eux. %
On parle alors du groupe fondamental de $X$, d\'efini \g{a} isomorphisme pr\g{e}s, ce groupe se note $\pi_1(X)$.

\par
Toutefois, l'isomorphisme qui pr\'ec\g{e}de n'est pas canonique : il d\'epend de la classe d'homotopie de $\sigma$.

\etoile
On peut montrer que toute application continue $f$ entre deux espaces point\'es $(X,x_0)$ et $(Y,y_0)$ d\'efinit, %
avec la formule $[\alpha]\mapsto [f\circ\alpha]$, un morphisme $f_{\sharp}$ entre les groupes $\pi_1 (X,x_0)$ et $\pi_1 (Y,y_0)$.

\par
La correspondance entre $f$ et $f_{\sharp}$ respecte la composition des applications et envoie l'application identit\'e d'un espace $(X,x_0)$ sur l'identit\'e de son groupe fondamental. %
On \'etablit notamment la proposition suivante :

\begin{prop}
Soient $(X,x_0)$ et $(Y,y_0)$ deux espaces point\'es tel qu'il existe un hom\'eomorphisme $h$ entre $X$ et $Y$ qui pr\'eserve les points de base.

\par
Alors : $\pi_1(X,x_0)$ et $\pi_1(Y,y_0)$ sont isomorphes.

\par
Si de plus $X$ et $Y$ sont connexes par arcs, alors les groupes $\pi_1 (X)$ et $\pi_1 (Y)$, d\'efinis \g{a} isomorphisme pr\`es, sont isomorphes.
\end{prop}

\etoile
Voici maintenant un premier exemple g\'en\'eral pour le calcul du groupe fondamental.

\begin{theo}
Soit $(X,x_0)$ un espace topologique point\'e. On suppose que $X$ est contractile : cet espace est donc, en particulier, connexe ar arcs.

\par
Alors : $\pi_1(X,x_0)$ est trivial. De plus, si $x_1$ est un autre \'el\'ement de $X$, alors $\pi_1(X,x_0)$ et $\pi_1(X,x_1)$ sont \textit{canoniquement} isomorphes.

\par
Le groupe $\pi_1(X)$, d\'efini \g{a} isomorphisme pr\g{e}s, est trivial.
\end{theo}

La d\'emonstration de ce th\'eor\g{e}me demande un peu de travail en termes de construction d'homotopies. Voici un r\'esultat interm\'ediaire, qui nous resservira ensuite :

\begin{lemm}\label{abcd}
$\alpha\beta\gamma\delta H$
\end{lemm}

\begin{proof}{\re}
\end{proof}

\begin{proof}[Th\'eor\g{e}me : \tr]
\end{proof}

\begin{defi}[Simple connexit\'e]
Un espace topologique $X$, connexe par arcs, et dont le groupe fondamental $\pi_1(X)$, d\'efini \g{a} isomorphisme pr\g{e}s, est trivial, est dit simplement connexe.
\end{defi}

\begin{exem}
Un espace contractile est simplement connexe.
\end{exem}

\subsubsection{Classes d'homotopie de lacets}

La caract\'erisation qui va suivre pour les classes d'homotopie de lacets est particuli\g{e}remant int\'eressante du point de vue topologique, %
m\^eme si elle oublie la structure alg\'ebrique qui \'emane de la concat\'enation des boucles.

Soit $\mathcal{Q}^{[0,1]}$ l'application : $t\mapsto \exp (2i\pi t)$. %
$\mathcal{P}^{[0,1]}$ est une surjection continue de $[0,1]$ vers $\mathbb{U}$, ferm\'ee car $[0,1]$ est compact. %
Il s'ensuit notamment que $\mathcal{Q}^{[0,1]}$ encoie un ouvert satur\'e de $[0,1]$ sur un ouvert de $\mathbb{U}$.
De plus, cette application identifie $0$ et $1$, et seulement ces deux points.

\begin{prop}\label{la_s1}
Soit $(X,x_0)$ un espace topologique point\'e, et $p_0$ un point de la sph\g{e}re $\mathbb{S}^1$ de l'espace euclidien $\mathbb{R}^2$.

\par
Il existe une bijection entre $\pi_1(X,x_0)$ et $[(\mathbb{S}^1,p_0);(X,x_0)]$.
\end{prop}

\begin{proof}
Prenons ici pour $(\mathbb{S}^1,p_0)$ l'espace point\'e $(\mathbb{U},1)$ afin de simplifier les formules de cette d\'emonstration, %
la preuve est exactement la m\^eme pour la sph\g{e}re de $\mathbb{R}^2$ point\'ee en un $p_0$ quelconque.

\par
Soit maintenant $\alpha$ une boucle de $(X,x_0)$. %
Puisque $0$ et $1$ sont envoy\'es en $1$ par l'application $\mathcal{Q}^{[0,1]}$, %
et puisque cette application envoie injectivement $]0,1[$ sur $\mathbb{U}\setminus\{1\}$, %
on peut d\'efinir de mani\g{e}re unique l'application $\tilde{\alpha}$ de $(\mathbb{U},1)$ vers $(X,x_0)$ par la formule :
\[\forall s \in [0,1],\tilde{\alpha}\left(\mathcal{Q}^{[0,1]}(s)\right)=\alpha (s)\]
De plus, cette application est continue car $\mathcal{Q}^{[0,1]}$ envoie un ouvert satur\'e de $[0,1]$ sur un ouvert de $\mathbb{U}$.

\par
R\'eciproquement, on associe, pour toute application continue $\tilde{\alpha}$ entre les espaces point\'es $(\mathbb{U},1)$ et $(X,x_0)$ %
une boucle $\alpha$ d\'efinie par :
\[\forall s\in [0,1],\alpha(s)=\tilde{\alpha}\left(\mathcal{Q}^{[0,1]}(s)\right)\]

Ces deux applications, $\alpha\mapsto\tilde{\alpha}$ et $\tilde{\alpha}\mapsto\alpha$ sont inverses l'une de l'autre, elles sont donc bijectives.

\par
Montrons que, si $\alpha$ et $\alpha '$ sont deux lacets homotopes de $(X,x_0)$,%
alors $\tilde{\alpha}$ et $\tilde{\alpha '}$ sont \'egalement homotopiquements \'equivalentes du point de vue des applications continues de $(\mathbb{U},1)$ vers $(X,x_0)$.

\par
Soit en effet $H$ une homotopie entre deux boucles $\alpha$ et $\alpha '$ de $(X,x_0)$. %
De m\^eme que dans le cas des lacets, les propri\'etes ensemblistes de $\mathcal{Q}^{[0,1]}$ %
permettent de d\'efinir, sur $\mathbb{U}\times [0,1]$, une application $\tilde{H}$ par :
\[\forall (s,t)\in[0,1]\times [0,1],\tilde{H}\left(\mathcal{Q}^{[0,1]}(s),t\right)=H(s,t)\]

La satur\'ee-ouvertitude de $\mathcal{Q}^{[0,1]}$ entra\^ine celle de $\left(\mathcal{Q}^{[0,1]},Id_{[0,1]}\right)$ -on identifie les $(0,t)$ et de $(1,t)$-, %
il s'ensuit, comme en  ce qui concernait la d\'efinition de $\tilde{\alpha}$, que l'application $\tilde{H}$ est continue sur $\mathbb{U}\times[0,1]$.

\par
V\'erifions que $\tilde{H}$ est une homotopie, relativement \g{a} $1$, entre les applications $\tilde{\alpha}$ et $\tilde{\alpha '}$.

\par
Remarquons d'abord que, pout tout $t$ r\'eel compris entre $0$ et $1$ :
\[\tilde{H}(1,t)=H(0,t)\text{ et de m\^eme }\tilde{H}(1,t)=H(1,t)\]

Puisque $H$ est une homotopie de lacets, on en d\'eduit que la fonction de $t$ ci-dessus prend une valeur unique, %
autrement dit $\tilde{H}$ est une homotopie relativement \g{a} $\{1\}$.

\par
Soit par ailleurs : $z\in\mathbb{U}$. Il existe au moins un \'el\'ement $s$ de $[0,1]$ tel que $\mathcal{Q}^{[0,1]}(s)$ vaut $z$. %
Ainsi : $\tilde{H}(z,1)=H(s,0)$ donc $\tilde{H}(z,0)=\alpha (s)$ par d\'efinition de l'homotopie $H$; %
$\alpha (s)=\tilde{\alpha}\left(\mathcal{Q}^{[0,1]}\right)$ autrement dit $\alpha (s)=\tilde{\alpha}(z)$, on en conclut :
\[\tilde{H}(z,0)=\tilde{\alpha}(z)\]
On d\'emontre de la m\^eme mani\g{e}re que : $\forall z \in\mathbb{U},\tilde{H}(z,1)=\tilde{\alpha '}(z)$.

\par
Enfin, si $\tilde{\alpha}$ et $\tilde{\alpha '}$ sont deux applications continues entre les espaces point\'es $(\mathbb{U},1)$ et $(X,x_0)$, %
alors les lacets $\alpha$ et $\alpha '$, d\'efinis avec la convention de langage qui pr\'ec\g{e}de, sont homotopes. %
Pour s'en convaincre, il suffit de construire une homotopie entre ces deux lacets en composant des applications continues.
\end{proof}

\begin{rema}
Cette proposition semble naturelle du point de vue topologique : %
en effet, $\mathbb{S}^1$ et $[0,1]$ peuvent tous les deux \^etre vus comme des compactifi\'es de l'espace $]0,1[$, %
le premier, par un point \g{a} l'infini, le deuxi\g{e}me par les points $0$ et $1$, %
qui sont justement identifi\'es par la projection qui relie $[0,1]$ \g{a} $\mathbb{S}^1$, %3
ou plut\^ot la bijection qui lui est canoniquement associ\'ee.
\end{rema}

\subsubsection{Un exemple important : $\pi_1(\mathbb{S}^1)$}\label{gs1}

Puisque deux espaces topologiques connexes par arcs, hom\'eomorphes entre eux, ont m\^eme groupe fondamental, %
d\'efini \g{a} isomorphisme pr\`es, on peut \'enoncer le th\'eor\g{e}me suivant :

\begin{theo}
Le groupe $\pi_1(\mathbb{S}^1)$, d\'efini \g{a} hom\'eomorphisme pr\`es, v\'erifie :
\[\pi_1(\mathbb{S}^1)\cong\mathbb{Z}\]
\end{theo}

\begin{proof}
Voici un r\'esultat interm\'ediaire pour l'\'etude de $[([0,1],\{0,1\}),(\mathbb{U},1)]$ :

\begin{lemm}[Rel\g{e}vement pour une homotopie de chemins]
Soient $\alpha$ et $\beta$ deux lacets de $\mathbb{U}$ bas\'es en $1$. On suppose qu'il existe une homotopie $F$ entre ces deux chemins.

\par
Alors : il existe une application continue $\tilde{F}$ de $[0,1]^2$ dans $\mathbb{R}$ telle que :
\[\forall (s,t)\in[0,1]^2 , \exp(2i\pi \tilde{F}(s,t))=F(s,t)\]

Si l'on demande aussi : $\tilde{F}(0,0)=0$, alors $\tilde{F}$ existe et est unique, %
c'est une homotopie entre les chemins $\tilde{\alpha}$ et $\tilde{\beta}$, %
respectivement uniques rel\g{e}vements de $\alpha$ et $\beta$ qui valent $0$ en $0$.
\end{lemm}

\begin{proof}
L'existence de l'application $\tilde{F}$ v\'erifiant la formule ci-dessus nous est donn\'ee par la proposition \ref{red2}, l'unicit\'e vient de \ref{urc}.

\par
On remarque que $s\mapsto \tilde{F}(s,0)$ est un rel\`evement de $\alpha$ qui vaut $0$ en $0$ : par unicit\'e dans la d\'efinition de $\tilde{\alpha}$, on en d\'eduit que
\[\forall s\in [0,1],\tilde{F}(s,0)=\tilde{\alpha}(s)\]

Par ailleurs :
\[\forall t\in [0,1],\mathcal{P}(\tilde{F}(0,t))=F(0,t)\]
o\g{u} $\mathcal{P}=a\mapsto\exp (2i\pi a)$.

\par
Puisque $F$ est une homotopie de chemins : $\forall t\in[0,1], F(0,1)=F(0,0)$, %
donc $[0,1]\rightarrow\mathbb{R}:t\mapsto\tilde{F}(0,t)$ parcourt la fibre de $F(0,0)$ pour la projection $\mathcal{P}$.

\par
On cette fibre est discr\g{e}te : la continuit\'e de $t\mapsto \tilde{F}(0,t)$ nous donne :
\[\forall t\in[0,1],\tilde{F}(0,t)=\tilde{F}(0,0)\text{ soit }\forall t\in[0,1],\tilde{F}(0,t)=0\]

Puisque $\tilde{\beta}(0)=0$, il s'ensuit que : $\tilde{F}(0,1)=\tilde{\beta}(0)$.

\par
Par ailleurs, l'\'egalit\'e : $\forall s\in[0,1],F(s,1)=\beta (s)$ entra\^ine %
$\forall s\in[0,1], \mathcal{P}(\tilde{F}(s,1))=\beta (s)$ donc d'apr\g{e}s \ref{urc} et l'\'egalit\'e d\'emontr\'ee ci-dessus :
\[\forall s\in[0,1], \tilde{F}(s,1)=\tilde{\beta}(s)\]

Enfin : $t\mapsto\tilde{F}(1,t)$ est une application continue sur $[0,1]$ qui parcourt $\mathcal{P}^{-1}(\alpha (1))$ puisque $F(t,1)$ vaut $\alpha (1)$ pour tout $t$ de $[0,1]$. %

\par
Il s'ensuit que : $\forall t\in[0,1] , \tilde{F}(1,t)=\tilde{F}(1,0)$ puisque cette fibre est discr\`ete, donc :
\[\forall t\in[0,1],\tilde{F}(1,t)=\tilde{\alpha}(1)\]

En particulier : $\tilde{\alpha}(1)=\tilde{\beta}(1)$, ce qui nous permet de conclure : %
$\tilde{F}$ est une homotopie entre les chemins $\tilde{\alpha}$ et $\tilde{\beta}$, qui prennent leurs valeurs dans $\mathbb{R}$.
\end{proof}

On peut donc associer sans ambig\"uit\'e, \g{a} toute classe d'homotopie $[\alpha]$ des lacets de $\mathbb{U}$ bas\'es en $1$, %
la classe d'homotopie de chemins $[\tilde{\alpha}]$, o\`u $\tilde{\alpha}$ est l'unique rel\`evement de $\alpha$ qui vaut $0$ en $0$.

\par
Toujours avec le lemme pr\'ec\'edent, on d\'efinit, de mani\g{e}re unique, pour toute classe $[\tilde{\alpha}]$ de chemins homotopes de $\mathbb{R}$, le r\'eel : $\tilde{\alpha}(1)$.

\par
Enfin, pour tout lacet $\alpha$ de $(\mathbb{U},1)$ : %
$\mathcal{P}^{-1}(\alpha (1))=\mathbb{Z}$ o\g{u} $\mathcal{P}$ d\'esigne toujours l'exponentielle complexe $a\mapsto\exp (2i\pi a)$.

\par
Il existe donc une unique application
\[\deg :\pi_1(\mathbb{U},1)\rightarrow \mathbb{Z}:[\alpha]\mapsto \tilde{\alpha}(1)\]
$\deg ([\alpha]$ sera appel\'e le degr\'e de la classe d'homotpie $[\alpha]$, pour tout lacet $\alpha$ de $(\mathbb{U},1)$.

\begin{lemm}\label{isodeg}
L'application $\deg :\pi_1(\mathbb{U},1)\rightarrow\mathbb{Z}:[\alpha]\mapsto\tilde{\alpha}(1)$ r\'ealise un isomorphisme de groupes.
\end{lemm}

\begin{proof}
On proc\g{e}de en trois \'etapes :
\begin{itemize}
\item[\textit{Propri\'et\'e de morphisme :}]
Soient $\alpha$ et $\beta$ deux lacets de $\mathbb{U}$ bas\'es en $1$. Nous allons calculer : $\tilde{\alpha\cdot\beta} (1)$.

Rappelons que le lacet $\alpha\cdot\beta$ est d\'efini par :
\[
%\left\{
\begin{array}{rcccl}
\forall t\in&\left[0,\frac{1}{2}\right]&\alpha\cdot\beta (t)&=&\alpha (2t)\\[1ex]
\forall t\in&\left[\frac{1}{2},1\right]&\alpha\cdot\beta (t)&=&\beta (2t-1)
\end{array}
%\right.
\]
Soient respectivement, $\tilde{alpha}$ et $\tilde{\beta}$ les rel\g{e}vements de $\alpha$ et $\beta$ qui valent $0$ en $0$. Soit de plus $\tilde{\gamma}$ le chemin d\'efini par :
\[
%\left\{
\begin{array}{rcccl}
\forall t\in&\left[0,\frac{1}{2}\right]&\tilde{\gamma} (t)&=&\tilde{\alpha} (2t)\\[1ex]
\forall t\in&\left[\frac{1}{2},1\right]&\tilde{\gamma} (t)&=&\tilde{\alpha}(1)+\tilde{\beta} (2t-1)
\end{array}
%\right.
\]
Pa recollement, $\tilde{\gamma}$ est continu. On remarque de plus que : $\tilde{\gamma}(1)=\tilde{\alpha}(1)+\tilde{\beta}(1)$.

\par
Montrons maintenant que $\tilde{\gamma}$ est un rel\`evement dans $\mathbb{R}$ de $\alpha\cdot\beta$, qui prend la valeur $0$ en $0$.

\par
La deuxi\g{e}me proposition est \'evidente.

\par
Par d\'efinition de $\tilde{\gamma}$ et $\tilde{\alpha}$ : $\forall t\in\left[0,\frac{1}{2}\right],\exp (2i\pi\tilde{\gamma}(t))=\alpha (2t)$ donc %
\fbox{$\forall t\in\left[0,\frac{1}{2}\right],\exp (2i\pi \tilde{\gamma}(t))=\alpha\cdot\beta (t)$}.

\par
Par ailleurs, puisque $\exp (2i\pi\tilde{\alpha}(1))=1$, $\tilde{\alpha}(1)\in\mathbb{Z}$ donc :

\[\forall t\in\left[\frac{1}{2},1\right]\exp (2i\pi(\tilde{\beta}(2t-1)+\tilde{\alpha}(1)))=\exp(2i\pi\tilde{\beta}(2t-1))\]
donc : $\forall t\in\left[\frac{1}{2},1\right],\exp (2i\pi \tilde{\gamma}(t))=\beta (2t-1)$ %
autrement dit : \fbox{$\forall t\in\left[\frac{1}{2},1\right],\exp (2i\pi\tilde{\gamma}(t))=\alpha\cdot\beta (t)$}

\par
Nous avons montr\'e la propri\'et\'e de rel\`evement pour $\tilde{\gamma}$.

\par
Ainsi : $\tilde{\gamma}=\tilde{\alpha\cdot\beta}$, ce qui entra\^ine : $\tilde{\alpha\cdot\beta} (1)=\tilde{\alpha}(1)+\tilde{\beta}(1)$. Il s'ensuit que :
\[\deg [\alpha\cdot\beta ]=\deg [\alpha ]+\deg [\beta ]\]
%
\item[\textit{Surjectitiv\'e :}]
Pour tout entier relatif $n$ : $\alpha :[0,1]\rightarrow \mathbb{U}:s\mapsto \exp{2in\pi s}$ admet comme rel\g{e}vement, nul en $0$, $s\mapsto ns$. %
$[\alpha]$ est donc de degr\'e $n$.
%
\item[\textit{Injectivit\'e :}]
Soit : $[\alpha]\in\ker\deg$.

\par
On peut \'ecrire $\tilde{\alpha}(1)=0$. Ainsi, $\tilde{\alpha}$ est un lacet de $\mathbb{R}$ bas\'e en $0$. Soit $\tilde{H}$ l'homotopie de chemins dans $\mathbb{R}$, d\'efinie par :
\[\forall (s,t)\in [0,1]^2,\tilde{H}(s,t)=(1-t)\tilde{\alpha (s)}+t\cdot 0\]
On note : $\forall (s,t)\in[0,1]^2,H(s,t)=\exp(2i\pi \tilde{H}(s,t))$

\par
Comme $\tilde{\alpha}$ est un rel\g{e}vement de $\alpha$ : $\forall s\in[0,1],H(s,0)=\alpha (s)$. 

\par
De plus :
\[\forall t\in[0,1],\left(\tilde{H}(0,t)=0\right)\wedge\left(\tilde{H}(1,t)=0\right)\]
donc
\[\forall t\in[0,1],\left(H(0,t)=1\right)\wedge\left(H(1,t)=1\right)\]
car $H$ est continue et car car la fibre de l'exponentielle complexe au-dessus de $1$ est discr\`ete.

\par
Enfin : $\forall s\in[0,1],\tilde{H}(s,1)=0$ donc $\forall s\in[0,1],H(s,1)=1$.

\par
Conclusion : $H$ est une homotopie entre $\alpha$ et le chemin constant, de valeur $1$. %
Il s'ensuit que : $[\alpha]$ est l'\'el\'ement neutre de $\pi_1(\mathbb{U},1)$, ce qui ach\g{e}ve la prueve pour l'injectivit\'e.
\end{itemize}
\end{proof}

On peut maintenant terminer la preuve du th\'eor\g{e}me. $\mathbb{U}$ est connexe par arcs : on d\'eduit du lemme \ref{isodeg} que le groupe $\pi_1(\mathbb{U}$, d\'efini \g{a} isomorphisme pr\g{e}s, est $\mathbb{Z}$.

Enfin : $\mathbb{S}^1\simeq\mathbb{U}$ \g{a} hom\'eomorphisme pr\g{e}s, il s'ensuit que :\[\pi_1(\mathbb{S}^1)\cong\mathbb{Z}\]
\end{proof}

\subsubsection{Pour finir ...}

On peut aller plus loin et constater que le groupe fondamental est un \textbf{invariant homotopique}, %
c'est-\g{a}-dire qu'il peut \^etre transport\'e par une \'equivalence homotopique entre deux espaces point\'es.

\begin{prop}
Soient $(X,x_0)$ un espace topologique point\'e, $Y$ un autre espace topologique, on suppose qu'il existe une \'equivalence homotopique $h$ de $X$ vers $Y$.

Alors : $\pi_1(X,x_0)$ est isomorphe \g{a} $\pi_1(Y,h(x_0))$.
\end{prop}

Ici encore, une d\'emonstration est possible avec les connaissances qui pr\'ec\g{e}dent, %elle requiert un peu de virtuosit\'e en mati\g{e}re de construction d'homotopie.
on peut utiliser le lemme \ref{abcd}

\par
A l'aide de ce r\'esultat, on retrouve facilement la simple connexit\'e des espaces contractiles, %
et on montre, par exemple, que l'espace $\mathbb{R}^2\setminus\{(0,0)\}$ admet comme groupe fondamental, d\'efini \g{a} isomorphisme pr\g{e}s, $\mathbb{Z}$.

\par
C'est aussi le cas de n'importe quelle couronne de $\mathbb{R}^2$.

\subsection{Groupes d'homotopie sup\'erieurs}


%Exemples : $z\mapsto z^2$ etc ?



%\subsubsection{Lacets en dimension $n$ et sph\g{e}res point\'ees}





%\section{Sph\g{e}res $\mathbb{S}^n$}%euclidiennes}

\subsubsection{Lacets en dimension $n$ et sph\g{e}res point\'ees}

La propri\'et\'e \ref{la_s1} se g\'en\'eralise aux dimensions sup\'erieures. %
Cela s'av\g{e}re utile dans la classification des fibr\'es principaux au-dessus des sph\g{e}res.
%
%On remarque que les groupes d'homotopie supérieurs sont naturellement associés aux classes d'homotopie d'applications continues depuis des sphères pointées.

\par
Soit en effet $n$ un entier naturel non nul.

\par
Le cube $[0,1]^n$, que l'on identifie à la boule-unité $B_n$ de l'espace normé $(\mathbb{R}^n,\| \|_{\infty})$ par une transformation affine, est homéomorphe à la boule-unité de $\mathbb{R}^n$ via l'application:\\
$\mathbb{R}^n \rightarrow\mathbb{R}^n : x \mapsto \frac{\|x\|_{\infty}}{\|x\|_2}x$, de changement de norme si l'on peut ainsi dire.

\par
On remarque que cette correspondance, entre deux espaces compacts, envoie la sphère $\partial{[0,1]^n}$ sur la sphère euclidienne de $\mathbb{R}^n$. %
On connait par ailleurs la bijection explicite :

\[b : B_n - \partial{B_n} \rightarrow \mathbb{R}^n : x \mapsto \frac{x}{(1-\|x\|^2)^{\frac{1}{2}}}\]

Si l'on compose cette fonction $b$ par la projection stéréographique inverse $\varphi_S^{-1}$ %
on associe, \g{a} un vecteur $(x_i)_{i \in [1,n]}$ le point de $\mathbb{S}^n$ dont les coordonnées s'écrivent :

\[((2x_i\sqrt{1-\|x\|^2})_{i \in [1,n]},2(1-\|x\|^2))\]

on remarque, comme dans le calcul avec les fibrations de Hopf dans la section précédente, %
que la même formule est définie sur la frontière $\partial{B_n}$ de $B_n$, et associe le pôle sud à chacun de ces points.

\par
Quitte \g{a} composer cette dernière application par une rotation de la sphère $\mathbb{S}^n$ %
nous avons défini une surjection continue $\beta$ entre les espaces $(B_n,\partial{B_n})$ et $\mathbb{S}^n - \{x_0\}$, %
où $x_0$ est un point quelconque de la sphère $\mathbb{S}^n$ à la place du pôle $S$.

\par
Cette surjection, continue entre deux compacts donc également ouverte, %
composée à celle que j'ai évoquée au début de ce paragraphe définit enfin un isomorphisme $\mathcal{C}_0$ %
entre les espaces $([0,1]^n,\partial{[0,1]^n})$ et $(\mathbb{S}^n,x_0)$.

%Ce résultat permet d'utiliser le calcul, difficile, des groupes d'homotopie supérieurs dans le cadre de la classification des fibrés principaux au-dessus des sphères, %
%à l'aide de l'invariant construit à la section suivante.
%\section{El\'ements de m\'ecanique quantique}
%\include{meca_q/meca_q}

\bibliography{fibib}
\bibliographystyle{alpha}

\end{document}
